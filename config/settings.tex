%---------------------------------------------------------------------------------------
%	Configuração a geometria do documento
%---------------------------------------------------------------------------------------
\geometry{
	paper=a4paper,
	top=3cm,
	bottom=2cm,
	left=3cm,
	right=2cm,
	headheight=15pt, % Header height
	footskip=1.5cm, % Space from the bottom margin to the baseline of the footer
	headsep=10pt,
}

%---------------------------------------------------------------------------------------
%	Configuração as referências cruzadas e links do documento
%---------------------------------------------------------------------------------------
\hypersetup{
	pdftoolbar=true,        % show Acrobat’s toolbar?
	pdfmenubar=true,        % show Acrobat’s menu?
	pdffitwindow=false,     % window fit to page when opened
	pdfstartview={FitH},    % fits the width of the page to the window
	pdftitle={Notas de Aula},           % title
	pdfauthor={Valdigleis S Costa},     % author
	pdfkeywords={Lógica, Computação}, % list of keywords
	pdfnewwindow=true,      % links in new PDF window
	colorlinks=true,        % false: boxed links; true: colored links
	linkcolor=blue,         % color of internal links
	citecolor=green,        % color of links to bibliography
	filecolor=cyan,         % color of file links
	urlcolor=red        % color of external links
}

%---------------------------------------------------------------------------------------
%	Configuração do espaçamento
%---------------------------------------------------------------------------------------
\renewcommand{\baselinestretch}{1.5} 

%\setlength{\parindent}{1.3cm}
% Controle do espaçamento entre um parágrafo e outro:
%\setlength{\parskip}{0.2cm} 

%---------------------------------------------------------------------------------------
%	Configuração das cores
%---------------------------------------------------------------------------------------
\definecolor{pupChapters}{RGB}{125, 50, 120}
\definecolor{myBlue}{RGB}{86, 210, 248}

% Usados no resultados teoricos
\definecolor{titleResults}{RGB}{5, 50, 120}
\definecolor{blueTheorem}{RGB}{10, 90, 180}
\definecolor{blueLemma}{RGB}{25, 150, 255}
\definecolor{blueCorollary}{RGB}{45, 200, 255}
\definecolor{blueBackground}{RGB}{240, 250, 250}

% Usado nos exemplos e exercicios
\definecolor{greenExample}{RGB}{19, 126, 10}

% Usado nas definições e observações
\definecolor{wine}{RGB}{150,85,110} 
\definecolor{wineBackground}{RGB}{255,230,240} 

\definecolor{chesse}{RGB}{243, 234, 156}
\definecolor{chesseBackground}{RGB}{133, 131, 116}

\definecolor{lemon}{RGB}{173, 248, 2}
\definecolor{lemonBackground}{RGB}{210, 229, 219}

% Usado nas observações, explicações e perguntas
\definecolor{snufflesColor}{RGB}{150, 185, 255}
\definecolor{rickColor}{RGB}{221, 196, 211}
\definecolor{mortyColor}{RGB}{145, 255, 195}

%---------------------------------------------------------------------------------------
%	Configuração do índice
%---------------------------------------------------------------------------------------
\contentsmargin{0cm} % Removes the default margin

% Part text styling (this is mostly taken care of in the PART HEADINGS section of this file)
\titlecontents{part}
[0cm] % Left indentation
{\addvspace{20pt}\large\sffamily\bfseries\bfseries} % Spacing and font options for parts
{}
{}
{}

% Chapter text styling
\titlecontents{chapter}
[1.25cm] % Left indentation
{\addvspace{12pt}\large\sffamily\bfseries} % Spacing and font options for chapters
{\color{myBlue!60}\contentslabel[\Large\thecontentslabel]{1.25cm}\color{myBlue}} % Formatting of numbered sections of this type
{\color{myBlue}} % Formatting of numberless sections of this type
{\color{myBlue!60}\normalsize\;\titlerule*[.5pc]{.}\;\thecontentspage} % Formatting of the filler to the right of the heading and the page number

% Section text styling
\titlecontents{section}
[1.25cm] % Left indentation
{\addvspace{3pt}\sffamily\bfseries} % Spacing and font options for sections
{\color{black}\contentslabel[\thecontentslabel]{1.25cm}\color{myBlue}} % Formatting of numbered sections of this type
{} % Formatting of numberless sections of this type
{\hfill\color{black}\thecontentspage} % Formatting of the filler to the right of the heading and the page number

% Subsection text styling
\titlecontents{subsection}
[1.25cm] % Left indentation
{\addvspace{1pt}\sffamily\small} % Spacing and font options for subsections
{\color{myBlue!60}\contentslabel[\thecontentslabel]{1.25cm}\color{myBlue}} % Formatting of numbered sections of this type
{} % Formatting of numberless sections of this type
{\ \titlerule*[.5pc]{.}\;\thecontentspage} % Formatting of the filler to the right of the heading and the page number

% Figure text styling
\titlecontents{figure}
[1.25cm] % Left indentation
{\addvspace{1pt}\sffamily\small} % Spacing and font options for figures
{\thecontentslabel\hspace*{1em}} % Formatting of numbered sections of this type
{} % Formatting of numberless sections of this type
{\ \titlerule*[.5pc]{.}\;\thecontentspage} % Formatting of the filler to the right of the heading and the page number

% Table text styling
\titlecontents{table}
[1.25cm] % Left indentation
{\addvspace{1pt}\sffamily\small} % Spacing and font options for tables
{\thecontentslabel\hspace*{1em}} % Formatting of numbered sections of this type
{} % Formatting of numberless sections of this type
{\ \titlerule*[.5pc]{.}\;\thecontentspage} % Formatting of the filler to the right of the heading and the page number

%----------------------------------------------------------------------------------------
%	MINI TABLE OF CONTENTS IN PART HEADS
%----------------------------------------------------------------------------------------

% Chapter text styling
\titlecontents{lchapter}
[0em] % Left indentation
{\addvspace{15pt}\large\sffamily\bfseries} % Spacing and font options for chapters
{\color{myBlue}\contentslabel[\Large\thecontentslabel]{1.25cm}\color{myBlue}} % Chapter number
{}  
{\color{myBlue}\normalsize\sffamily\bfseries\;\titlerule*[.5pc]{.}\;\thecontentspage} % Page number

% Section text styling
\titlecontents{lsection}
[0em] % Left indentation
{\sffamily\small} % Spacing and font options for sections
{\contentslabel[\thecontentslabel]{1.25cm}} % Section number
{}
{}

% Subsection text styling (note these aren't shown by default, display them by searchings this file for tocdepth and reading the commented text)
\titlecontents{lsubsection}
[.5em] % Left indentation
{\sffamily\footnotesize} % Spacing and font options for subsections
{\contentslabel[\thecontentslabel]{1.25cm}}
{}
{}

%----------------------------------------------------------------------------------------
%	HEADERS AND FOOTERS
%----------------------------------------------------------------------------------------
\pagestyle{fancy} % Enable the custom headers and footers

\renewcommand{\chaptermark}[1]{\markboth{\sffamily\normalsize\bfseries\chaptername\ \thechapter.\ #1}{}} 	% Styling for the current chapter in the header
\renewcommand{\sectionmark}[1]{\markright{\sffamily\normalsize\thesection\hspace{5pt}#1}{}} 				% Styling for the current section in the header

\fancyhf{} % Clear default headers and footers
\fancyhead[LE,RO]{\sffamily\normalsize\thepage} % Styling for the page number in the header
\fancyhead[LO]{\rightmark} % Print the nearest section name on the left side of odd pages
\fancyhead[RE]{\leftmark} % Print the current chapter name on the right side of even pages
%\fancyfoot[C]{\thepage} % Uncomment to include a footer

\renewcommand{\headrulewidth}{0.5pt} % Thickness of the rule under the header

\fancypagestyle{plain}{% Style for when a plain pagestyle is specified
	\fancyhead{}\renewcommand{\headrulewidth}{0pt}%
}

% Removes the header from odd empty pages at the end of chapters
\makeatletter
\renewcommand{\cleardoublepage}{
	\clearpage\ifodd\c@page\else
	\hbox{}
	\vspace*{\fill}
	\thispagestyle{empty}
	\newpage
	\fi}

%---------------------------------------------------------------------------------------
%	Configurações de teoremas, proposições, lemas, corolários, definições e métodos
%---------------------------------------------------------------------------------------
\newcommand{\intoo}[2]{\mathopen{]}#1\,;#2\mathclose{[}}
\newcommand{\ud}{\mathop{\mathrm{{}d}}\mathopen{}}
\newcommand{\intff}[2]{\mathopen{[}#1\,;#2\mathclose{]}}
\renewcommand{\qedsymbol}{$\square$}


% Estilo com caixa
\newtheoremstyle{BoxedStyle}%
{0pt}																									% Space above
{0pt}																									% Space below
{\normalfont}																							% Body font
{}																										% Indent amount
{\small\bf\sffamily\color{black}}																		% Theorem head font
{\;}																									% Punctuation after theorem head
{0.25em}																								% Space after theorem head
{\small\sffamily\color{titleResults}\thmname{#1}\nobreakspace\thmnumber{\@ifnotempty{#1}{}\@upn{#2}}		% Theorem text (e.g. Theorem 2.1)
	\thmnote{\nobreakspace\the\thm@notefont\sffamily\bfseries\color{black}---\nobreakspace#3.}} 		% Optional theorem note

% Estilo dos exemplos
\newtheoremstyle{exampleStyle}
{5pt}																									% Space above
{5pt}																									% Space below
{\normalfont}																							% Body font
{} 																										% Indent amount
{\small\bf\sffamily\color{greenExample}}																% Theorem head font
{\;}																									% Punctuation after theorem head
{0.25em}																								% Space after theorem head
{\small\sffamily{\tiny\ensuremath{\blacksquare}}\nobreakspace\thmname{#1}\nobreakspace\thmnumber{\@ifnotempty{#1}{}\@upn{#2}}% Theorem text (e.g. Theorem 2.1)
	\thmnote{\nobreakspace\the\thm@notefont\sffamily\bfseries---\nobreakspace#3.}}						% Optional theorem note

% Estilo da Definição
\newtheoremstyle{blacknumbox} % Theorem style name
{0pt}																							% Space above
{0pt}																							% Space below
{\normalfont}																					% Body font
{}																								% Indent amount
{\small\bf\sffamily}																			% Theorem head font
{\;}																							% Punctuation after theorem head
{0.25em}																						% Space after theorem head
{\small\sffamily\thmname{#1}\nobreakspace\thmnumber{\@ifnotempty{#1}{}\@upn{#2}}				% Theorem text (e.g. Theorem 2.1)
	\thmnote{\nobreakspace\the\thm@notefont\sffamily\bfseries---\nobreakspace#3.}}				% Optional theorem note

\makeatother


% Os contadores e comandos básicos
\newcounter{dummy} 
\numberwithin{dummy}{section}

% Os ambientes de teoremas/lema/corolário em caixa
\theoremstyle{BoxedStyle}
\newtheorem{theoremeT}{Teorema}[chapter]
\newtheorem{lemmaT}{Lema}[chapter]
\newtheorem{corollaryT}{Corolário}[chapter]
\newtheorem{proposition}{Proposição}[chapter]

\theoremstyle{exampleStyle}
\newtheorem{exampleT}{Exemplo}[chapter]
\newtheorem{exerciseT}{Exercício}[chapter]

\theoremstyle{blacknumbox}
\newtheorem{definitionT}{Definição}[chapter]
\newtheorem{remaT}{Observação}[chapter]
\newtheorem{noteT}{{\color{blue}\scriptsize $\textdbend$ }Nota}[chapter]

% Caixa Teorema
\newmdenv[skipabove=7pt,
skipbelow=7pt,
rightline=true,
leftline=true,
topline=false,
bottomline=false,
backgroundcolor=blueBackground,
linecolor=blueTheorem,
innerleftmargin=5pt,
innerrightmargin=5pt,
innertopmargin=5pt,
innerbottommargin=5pt,
leftmargin=0cm,
rightmargin=0cm,
linewidth=4pt]{theoremBox}

% Caixa Lema
\newmdenv[skipabove=7pt,
skipbelow=7pt,
rightline=true,
leftline=true,
topline=false,
bottomline=false,
backgroundcolor=blueBackground,
linecolor=blueLemma,
innerleftmargin=5pt,
innerrightmargin=5pt,
innertopmargin=5pt,
innerbottommargin=5pt,
leftmargin=0cm,
rightmargin=0cm,
linewidth=4pt]{lemmaBox}

% Caixa Corolário
\newmdenv[skipabove=7pt,
skipbelow=7pt,
rightline=true,
leftline=true,
topline=false,
bottomline=false,
backgroundcolor=blueBackground,
linecolor=blueCorollary,
innerleftmargin=5pt,
innerrightmargin=5pt,
innertopmargin=5pt,
innerbottommargin=5pt,
leftmargin=0cm,
rightmargin=0cm,
linewidth=4pt]{corollaryBox}

% Caixa da definição
\newmdenv[skipabove=7pt,
skipbelow=7pt,
rightline=true,
leftline=true,
topline=false,
bottomline=false,
linecolor=wine,
backgroundcolor=wineBackground,
innerleftmargin=5pt,
innerrightmargin=5pt,
innertopmargin=5pt,
innerbottommargin=5pt,
leftmargin=0cm,
rightmargin=0cm,
linewidth=1pt]{dBox}

% Caixa da observação
\newmdenv[skipabove=7pt,
skipbelow=7pt,
rightline=true,
leftline=true,
topline=true,
bottomline=true,
linecolor=black,
backgroundcolor=white,
innerleftmargin=5pt,
innerrightmargin=5pt,
innertopmargin=5pt,
innerbottommargin=5pt,
leftmargin=0cm,
rightmargin=0cm,
linewidth=0.5pt]{oBox}

\newmdenv[skipabove=7pt,
skipbelow=7pt,
rightline=false,
leftline=false,
topline=false,
bottomline=false,
innerleftmargin=5pt,
innerrightmargin=5pt,
innertopmargin=5pt,
innerbottommargin=5pt,
leftmargin=0cm,
rightmargin=0cm,
linewidth=0.5pt]{iBox}

% Caixa da observação - Usada
\newmdenv[skipabove=7pt,
skipbelow=7pt,
rightline=true,
leftline=true,
topline=false,
bottomline=false,
linecolor=chesseBackground,
backgroundcolor=chesse,
innerleftmargin=5pt,
innerrightmargin=5pt,
innertopmargin=5pt,
innerbottommargin=5pt,
leftmargin=0cm,
rightmargin=0cm,
linewidth=1pt]{rBox}


\newmdenv[skipabove=7pt,
skipbelow=7pt,
rightline=true,
leftline=true,
topline=false,
bottomline=false,
linecolor=lemon,
backgroundcolor=lemonBackground,
innerleftmargin=5pt,
innerrightmargin=5pt,
innertopmargin=5pt,
innerbottommargin=5pt,
leftmargin=0cm,
rightmargin=0cm,
linewidth=1pt]{nBox}


% O ambiente para o teorema
\newenvironment{theorem}{\begin{theoremBox}\begin{theoremeT}}{\end{theoremeT}\end{theoremBox}}
\newenvironment{lemma}{\begin{lemmaBox}\begin{lemmaT}}{\end{lemmaT}\end{lemmaBox}}
\newenvironment{corollary}{\begin{corollaryBox}\begin{corollaryT}}{\end{corollaryT}\end{corollaryBox}}

% O ambiente para o example
%\newenvironment{example}{\begin{exampleT}}{\hfill{\tiny\ensuremath{\blacksquare}}\end{exampleT}}
\newenvironment{example}{\begin{exampleT}}{\end{exampleT}}
\newenvironment{problem}{\begin{exerciseT}}{\end{exerciseT}}



% O ambiente para definição
\newenvironment{definition}{\begin{dBox}\begin{definitionT}}{\end{definitionT}\end{dBox}}
\newenvironment{remark}{\begin{rBox}\begin{remaT}}{\end{remaT}\end{rBox}}
\newenvironment{note}{\begin{nBox}\begin{noteT}}{\end{noteT}\end{nBox}}

%---------------------------------------------------------------------------------------
%	Configuração os ambientes de obserção
%---------------------------------------------------------------------------------------
\newtheorem*{snuffles*}{
	\includegraphics[scale=0.05]{figures/snuffles.png}
}

\newtheorem*{rick*}{
	\includegraphics[scale=0.07]{figures/rick.png}
}

\newtheorem*{morty*}{
	\includegraphics[scale=0.05]{figures/morty.png}
}


\newtheorem*{snufflesinfo*}{
	\includegraphics[scale=0.05]{figures/snuffles.png}
}

\newtheorem*{rickinfo*}{
	\includegraphics[scale=0.07]{figures/rick.png}
}

\newtheorem*{mortyinfo*}{
	\includegraphics[scale=0.05]{figures/morty.png}
}


\newenvironment{SnufflesInfo}{\begin{iBox}\begin{snufflesinfo*}}{\end{snufflesinfo*}\end{iBox}}
\newenvironment{RickInfo}{\begin{iBox}\begin{rickinfo*}}{\end{rickinfo*}\end{iBox}}		
\newenvironment{MortyInfo}{\begin{iBox}\begin{mortyinfo*}}{\end{mortyinfo*}\end{iBox}}

\newenvironment{Snuffles}{\begin{oBox}\begin{snuffles*}}{\end{snuffles*}\end{oBox}}
\newenvironment{Rick}{\begin{oBox}\begin{rick*}}{\end{rick*}\end{oBox}}		
\newenvironment{Morty}{\begin{oBox}\begin{morty*}}{\end{morty*}\end{oBox}}	

%---------------------------------------------------------------------------------------
%	Configuração da lista para questões dos exercícios
%---------------------------------------------------------------------------------------
\newlist{fieldsList}{enumerate}{1}
\setlist[fieldsList, 1]{label={\color{blue}(\roman*)}}

\newlist{exerList}{enumerate}{1}
\setlist[exerList,1]{label={\color{blue}(\alph*).}}

%---------------------------------------------------------------------------------------
%	Configuração do ambiente listings (para algoritmos "importados")
%---------------------------------------------------------------------------------------
\lstdefinestyle{mystyle}{
	backgroundcolor=\color{black!5},   
	commentstyle=\color{magenta},
	keywordstyle=\color{blue},
	numberstyle=\tiny\color{gray},
	stringstyle=\color{purple},
	basicstyle=\ttfamily\footnotesize,
	breakatwhitespace=false,         
	breaklines=true,                 
	captionpos=b,                    
	keepspaces=false,                 
	numbers=left,                    
	numbersep=5pt,                  
	showspaces=false,                
	showstringspaces=false,
	showtabs=false,                  
	tabsize=1,
	language=C
}

\lstset{style=mystyle}



