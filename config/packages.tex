%---------------------------------------------------------------------------------------
%	O pacote de caracteres
%---------------------------------------------------------------------------------------
\usepackage[utf8]{inputenc} 

%---------------------------------------------------------------------------------------
%	O pacote de codificação de 8 bits com 256 glifos
%---------------------------------------------------------------------------------------
\usepackage[T1]{fontenc}

%---------------------------------------------------------------------------------------
%	Os pacotes para cuida do hifém
%---------------------------------------------------------------------------------------
\usepackage[portuguese]{babel}
\usepackage{csquotes}

%---------------------------------------------------------------------------------------
%	O pacote para customizar as listas
%---------------------------------------------------------------------------------------
\usepackage{enumitem}

%---------------------------------------------------------------------------------------
%	O pacote para definir cores
%---------------------------------------------------------------------------------------
\usepackage{xcolor}

%---------------------------------------------------------------------------------------
%	O pacote para identar a primeira linha dos capítulos e seções
%---------------------------------------------------------------------------------------
\usepackage{indentfirst}

%---------------------------------------------------------------------------------------
%	O pacote para a geometria do documento
%---------------------------------------------------------------------------------------
\usepackage{geometry}

%---------------------------------------------------------------------------------------
%	O pacote para as fontes (texto e matemática) do documento
%---------------------------------------------------------------------------------------
\usepackage{avant}
\usepackage{mathptmx}

%---------------------------------------------------------------------------------------
%	O pacote para importar imagens
%---------------------------------------------------------------------------------------
\usepackage{graphicx} 

%---------------------------------------------------------------------------------------
%	O pacote para importar imagens como sendo o fundo do documento
%---------------------------------------------------------------------------------------
\usepackage{eso-pic}

%---------------------------------------------------------------------------------------
%	O pacote para pegar o tempo atual
%---------------------------------------------------------------------------------------
\usepackage[yyyymmdd,hhmmss]{datetime}

%---------------------------------------------------------------------------------------
%	O pacote para o índice
%---------------------------------------------------------------------------------------
\usepackage{titletoc} 

%---------------------------------------------------------------------------------------
%	O pacote para configurar os títulos e os rodapés
%---------------------------------------------------------------------------------------
\usepackage{fancyhdr}

%---------------------------------------------------------------------------------------
%	Os pacotes da matemática
%---------------------------------------------------------------------------------------
\usepackage{amsmath}
\usepackage{amsfonts}
\usepackage{amssymb}
\usepackage{amsthm}
\usepackage{mathtools}

%---------------------------------------------------------------------------------------
%	O pacote para adicionar o not antes das relações
%---------------------------------------------------------------------------------------
\usepackage{centernot}

%---------------------------------------------------------------------------------------
%	O pacote para criar ambientes de caixa nas definições e teoremas
%---------------------------------------------------------------------------------------
\RequirePackage[framemethod=default]{mdframed}

\usepackage{tcolorbox}

%---------------------------------------------------------------------------------------
%	O pacote para criar subimagens
%---------------------------------------------------------------------------------------
\usepackage{subfig}

%---------------------------------------------------------------------------------------
%	O pacote de desenho tikz
%---------------------------------------------------------------------------------------
\usepackage{tikz}
\usepackage{tikz-qtree}

\usetikzlibrary{positioning, calc, chains, fit, shapes, automata, trees}


%---------------------------------------------------------------------------------------
%	O pacote de desenho circuitikz
%---------------------------------------------------------------------------------------
\usepackage{circuitikz}

%---------------------------------------------------------------------------------------
%	O pacote para calculos 
%---------------------------------------------------------------------------------------
%\usepackage{calc}

%---------------------------------------------------------------------------------------
%	O pacote para multicoluna e multilinha
%---------------------------------------------------------------------------------------
\usepackage{multicol}
\usepackage{multirow}

%---------------------------------------------------------------------------------------
%	O pacote para cortar as cédulas de uma matriz
%---------------------------------------------------------------------------------------
\usepackage{slashbox}

%---------------------------------------------------------------------------------------
%	O pacote para setar espaçamento
%---------------------------------------------------------------------------------------
\usepackage{setspace}

%---------------------------------------------------------------------------------------
%	O pacote as referências cruzadas e links
%---------------------------------------------------------------------------------------
\usepackage{hyperref}

%---------------------------------------------------------------------------------------
%	O pacote para epígrafe
%---------------------------------------------------------------------------------------
\usepackage{epigraph}

%---------------------------------------------------------------------------------------
%	O pacote para adicionar a plaquinha
%---------------------------------------------------------------------------------------
\usepackage{manfnt}

%---------------------------------------------------------------------------------------
%	O pacote para criar os diagramas de bloco de prova
%---------------------------------------------------------------------------------------
\usepackage{logicproof}


%---------------------------------------------------------------------------------------
%	O pacote para criar os diagramas de fitch nas provas em lógica
%---------------------------------------------------------------------------------------
\usepackage{fitch}

%---------------------------------------------------------------------------------------
%	O pacote para criar as árvores de derivação nas provas em lógica
%---------------------------------------------------------------------------------------
\usepackage{proof}

%---------------------------------------------------------------------------------------
%	O pacote para escrever pseudo códigos
%---------------------------------------------------------------------------------------
\usepackage[portuguese,ruled,lined, linesnumbered]{algorithm2e}
\usepackage{algorithmic}

%---------------------------------------------------------------------------------------
%	O pacote para escrever e importar códigos de linguagens de programação
%---------------------------------------------------------------------------------------
\usepackage{listings}

%---------------------------------------------------------------------------------------
%	O pacote e configuração para a referência
%---------------------------------------------------------------------------------------
\usepackage{natbib}
%\usepackage[style=numeric,citestyle=numeric,sorting=nyt,sortcites=true,autopunct=true,hyperref=true,abbreviate=false,backref=true,backend=biber]{biblatex}