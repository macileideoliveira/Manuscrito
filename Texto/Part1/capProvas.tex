% !TeX spellcheck = <none>
\chapter{Métodos de Demonstração}\label{cap:Demonstracoes}

\begin{introduction}[Tópicos]
	\item Introdução
	\item Demonstrando Implicações
	\item Demonstrando por Absurdo
	\item Demonstrando por Casos
	\item Demonstrando Universalidade
	\item Demonstrando Existência
	\item Demonstrando Unicidade
\end{introduction}

\section{Introdução}\label{sec:Introducao-Demonstracoes}

No capítulo anterior, o leitor encontrou diversas demonstrações dentro da teoria intuitiva (ou Cantoriana) dos conjuntos. Para um leitor iniciante, talvez tenha sido um tanto quanto complicado entender a metodologia usada para construir tais demonstrações. E uma vez que, as demonstrações são figuras de interesse central no cotidiano dos matemáticos, cientistas da computação e engenheiros de software, em especial aqueles que trabalham com métodos formais, este texto irá fazer uma breve pausa no estudo da teoria dos conjuntos, para apresentar um pouco de teoria da prova ao leitor.

Este capítulo começa então com o seguinte questionamento: Do ponto de vista da ciência da computação qual a importância das demonstrações? Bem a resposta a essa pergunta pode ser dada de dois pontos de vista,  um teórico (purista) e um prático (aplicado ou de engenharia).

Na perspectiva de um cientista da computação puro, as demonstrações de teoremas são a principal ferramenta para investigar os limites dos diferentes modelos de computação propostos \cite{hopcroft2008, linz2006}, assim sendo é de suma importância que o estudante de graduação em ciência da computação receba em sua formação pelo menos o básico para dominar a ``arte'' de provar teoremas, sendo assim preparado para o estudo e a pesquisa pura em computação e(ou) matemática.

Já na visão prática, só existe uma forma segura de garantir que um \textit{software} está livre de erros, essa ``tecnologia'' é exatamente a demonstração das propriedades do \textit{software}. É claro que, mostrar que um \textit{software} não possui erros vai exigir que o \textit{software} seja visto através de um certo nível de formalismo e rigor matemático, mas após essa modelagem através de demonstrações pode-se garantir que um \textit{software} não apresentará erros, e assim se algo errado ocorrer foi por fatores externos, tais como defeito no \textit{hardware} por exemplo, e não por falha ou erros com a implementação. Este conceito é o cerne de uma área da engenharia de \textit{software} \cite{pressman2016}, chamada métodos  (ou especificações) formais, sendo essa área o ponto crucial no desenvolvimento de \textit{softwares} para sistemas críticos \cite{sommerville2011}. Isto já mostra a grande importância de programadores e engenheiros de \textit{software} terem em sua formação as bases para o domínio das técnicas de demonstração.

Nas próxima seções deste manuscrito serão descritas as principais técnicas de demonstração de interesse de matemáticos, cientistas da computação e engenheiros formais de \textit{software}. 

\begin{remark}
	Para o leitor que nunca antes teve contato com a lógica matemática recomenda-se que antes de estudar este capítulo, o leitor faça pelo menos um rápido estudo do Capítulo \ref{cap:IntroducaoLogica}.
\end{remark}

Para pode falar sobre métodos de demonstração e poder então descrever como os matemáticos, lógicos e cientistas da computação justificam asserções usando apenas a argumentação matemática, será necessário fixar algumas nomenclaturas e falar sobre alguns conceitos importantes.

\begin{definition}[Asserção]\label{def:Assercao}
	Uma \textbf{asserção} é qualquer frase declarativa que possa ser expressa na linguagem da lógica simbólica.
\end{definition}

\begin{remark}
	O leitor que conheça lógica nota facilmente que uma asserção é uma proposição ou predicado, para detalhes ver o Capítulo \ref{cap:IntroducaoLogica}.
\end{remark}

De forma geral a demonstração de uma asserção pode ser descrita em três passos:

\begin{enumerate}
	\item Expressar a asserção como uma fórmula da lógica simbólica (ver Capítulo \ref{cap:IntroducaoLogica}).
	\item Construir um rascunho que demonstra a asserção utilizando os métodos de demonstração.
	\item Produzir um texto formal a partir do rascunho obtido no passo anterior.
\end{enumerate}


Os métodos (ou estratégias) de demonstrações apresentadas neste manuscrito seguem as ideias e ordem  de apresentação similar ao que foi exposto em \cite{velleman2019comProvar}. Já o rascunho mencionando anteriormente, neste manuscrito será visto como sendo o ambiente em que será realizada as deduções necessárias para as demonstrações das asserções, isto é, o mesmo não é a prova em si, em vez disso, o mesmo é uma ferramenta para construir a prova. 

Diferente de \cite{velleman2019comProvar}, em que a estrutura do rascunho é dividida é muito similar ao que acontece em demonstrações usando provadores de teoremas como Coq \cite{coq2013} e Lean \cite{lean2015}, isto é, em que existe uma separação entre dados (hipótese) e os objetivos (em inglês \textit{Goal}) que se quer demonstrar. Neste manuscrito será adotado como rascunho de prova a contrução de provas em caixa ao estilo de Fitch \cite{broda2007}. A ideia de diagrama de caixa será introduzida gradativamente a cada método.

\section{Demonstrando Implicações}\label{sec:ProvandoImplicacao}

Este texto iniciar a apresentação dos métodos de demonstração pela estratégia conhecida como \textbf{prova direta de implicações}, ela é provavelmente o método de demonstração mais utilizado, e consiste em uma prova cujo objetivo é provar uma asserção do tipo: ``Se $\alpha$, então $\beta$''. Para facilitar a escrita será usado a simbologia da lógica $\alpha \Rightarrow \beta$ para representar as implicações.

\begin{method}[Prova direta de: $\alpha \Rightarrow \beta$]
	\begin{enumerate}
		\item Suponha que $\alpha$ é verdadeiro.
		\item Deduza a asserção $\beta$.
		\item Conclua a asserção: Se $\alpha$, então $\beta$.
	\end{enumerate}
\end{method}

Com respeito a ideia do rascunho, antes da aplicação do método de prova direta o rascunho iria conter na coluna de dados um conjunto finito (que pode ser vazio) de hipóteses verdadeiras e teria na coluna de objetivo a implicação $\alpha \Rightarrow \beta$, ou seja, o rascunho estaria da forma:

\begin{flushleft}
	\begin{tabular}{ll}
		Dados & Objetivos \\
		$\gamma_1, \cdots, \gamma_n$ & $\alpha\Rightarrow\beta$
	\end{tabular}
\end{flushleft}
\noindent após a aplicação do método de prova direta o rascunho deverá ser atualizado de forma que $\alpha$ se torne um dado e que $\beta$ passe a ser o novo objetivo a ser demonstrado, ou seja, o rascunho passa a ter a forma: 

\begin{flushleft}
	\begin{tabular}{ll}
		Dados & Objetivos \\
		$\gamma_1, \cdots, \gamma_n$ & $\beta$\\
		$\alpha$
	\end{tabular}
\end{flushleft}
\noindent assim fica claro que o método de prova direta tem ação imediata sobre a estrutura do rascunho de prova, no sentido de que tal método muda diretamente a forma do rascunho ao adicionar $\alpha$ com um dado e mudando o objetivo a ser demonstrado de $\alpha \Rightarrow \beta$ para $\beta$.

\begin{remark}
	O leitor mais experiente em lógica pode notar que na verdade o método de prova direta de implicações nada mais é que o uso da regra de introdução da implicação.
\end{remark}

Os exemplos a seguir esboçam ao leitor o uso da prova direta para provar resultados clássicos sobre os números inteiros.

\begin{problem}
	Demonstre a asserção: Dado $n \in \Int$. Se $n$ é impar, então $n^2 + n$ é um número par.
\end{problem}
