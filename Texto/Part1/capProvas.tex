\chapter{Métodos de Demonstração}\label{cap:Demonstracoes}

\begin{introduction}[Tópicos]
	\item Introdução
	\item Demonstrando Implicações
\end{introduction}

\section{Introdução}\label{sec:Introducao-Demonstracoes}

No capítulo anterior, o leitor encontrou diversas demonstrações dentro da teoria intuitiva (ou Cantoriana) dos conjuntos. Para um leitor iniciante, talvez tenha sido um tanto quanto complicado entender a metodologia usada para construir tais demonstrações. E uma vez que, as demonstrações são figuras de interesse central no cotidiano dos matemáticos, cientistas da computação e engenheiros de software, em especial aqueles que trabalham com métodos formais, este texto irá fazer uma breve pausa no estudo da teoria dos conjuntos, para apresentar um pouco de teoria da prova ao leitor.

Este capítulo começa então com o seguinte questionamento: Do ponto de vista da ciência da computação qual a importância das demonstrações? Bem a resposta a essa pergunta pode ser dada de dois pontos de vista,  um teórico (purista) e um prático (aplicado ou de engenharia).

Na perspectiva de um cientista da computação puro, as demonstrações de teoremas são a principal ferramenta para investigar os limites dos diferentes modelos de computação propostos \cite{hopcroft2008, linz2006}, assim sendo é de suma importância que o estudante de graduação em ciência da computação receba em sua formação pelo menos o básico para dominar a ``arte'' de provar teoremas, sendo assim preparado para o estudo e a pesquisa pura em computação e(ou) matemática.

Já na visão prática, só existe uma forma segura de garantir que um \textit{software} está livre de erros, essa ``tecnologia'' é exatamente a demonstração das propriedades do \textit{software}. É claro que, mostrar que um \textit{software} não possui erros vai exigir que o \textit{software} seja visto através de um certo nível de formalismo e rigor matemático, mas após essa modelagem através de demonstrações pode-se garantir que um \textit{software} não apresentará erros, e assim se algo errado ocorrer foi por fatores externos, tais como defeito no \textit{hardware} por exemplo, e não por falha ou erros com a implementação. Este conceito é o cerne de uma área da engenharia de \textit{software} \cite{pressman2016}, chamada métodos  (ou especificações) formais, sendo essa área o ponto crucial no desenvolvimento de \textit{softwares} para sistemas críticos \cite{sommerville2011}. Isto já mostra a grande importância de programadores e engenheiros de \textit{software} terem em sua formação as bases para o domínio das técnicas de demonstração.

Nas próxima seções deste manuscrito serão descritas as principais técnicas de demonstração de interesse de matemáticos, cientistas da computação e engenheiros formais de \textit{software}. 

\begin{remark}
	Para o leitor que nunca antes teve contato com a lógica matemática recomenda-se que antes de estudar este capítulo, o leitor faça pelo menos um rápido estudo do Capítulo \ref{cap:IntroducaoLogica}.
\end{remark}

Para pode falar sobre métodos de demonstração e poder então descrever como os matemáticos, lógicos e cientistas da computação justificam asserções usando apenas a argumentação matemática, será necessário fixar algumas nomenclaturas e falar sobre alguns conceitos importantes.

\begin{definition}[Asserção]\label{def:Assercao}
	Uma Asserção é qualquer frase declarativa que possa ser expressa na linguagem da lógica simbólica.
\end{definition}

\begin{remark}
	O leitor que conheça lógica nota facilmente que uma asserção é uma proposição ou predicado, para detalhes ver o Capítulo \ref{cap:IntroducaoLogica}.
\end{remark}

Para que uma demonstração de uma asserção possa ser aceita como correta, é comum exigir que a argumentação de tal demonstração deve conter um alto nível de rigor (formalismo) matemático, além disso, quanto maior for a riqueza\footnote{A riqueza de detalhes que uma demonstração tem pode variar, a depender para quem a prova se destina, por exemplo, uma prova escrita para um físico em geral não se preocupa com os por menores do linguajá matemático, diferentemente de uma prova escrita para matemáticos, onde os menores detalhes da linguagem matemática são considerados como informações relevantes.} de detalhes, mas fácil é de se entender as demonstrações. As ações usada para realizar demonstrações recebem o nome de métodos (ou estratégias) de demonstrações.

Os métodos de demonstração apresentadas neste manuscrito seguem as ideias, a forma e ordem similares as apresentadas em \cite{velleman2019comProvar}. Para que fique mais simples do leitor iniciante entender, neste manuscrito a tarefa de realizar demonstração acontecerá em duas etapas: (1) será realizada a construção de um rascunho\footnote{O leitor deve ter em mente que menos no rascunho é conveniente escrever o máximo das informações na linguagem da lógica de primeira ordem, isso torna mais simples a conversão do rascunho de prova para o texto formal da prova.} de prova, e (2) a partir do rascunho gerado no passo anterior será elaborado um texto formal que em sintase é a prova em si.

O rascunho é um ambiente na forma de uma tabela subdividida em três colunas, a primeira o marca número de passos (ou estados) da prova, a segunda coluna contém os dados disponíveis e a terceira coluna diz respeito aos objetivos (ou em inglês \textit{Goal}) que devem ser atingidos. Cada linha no rascunho por sua vez é um estado da prova\footnote{Esse tipo de visão é similar ao que acontece em provadores de teoremas como Coq \cite{coq2013} e Lean.}. Uma prova está terminada (ou completa) quando todos os objetivos tiverem sido eliminados (ou satisfeitos). 

\begin{remark}
	Os objetivos são modificados apenas de duas formas no rascunho, quando é aplicado alguma estratégia de demonstração que altere os objetivos ou quando os mesmo são obtidos na coluna de dados por meio de inferência, neste caso o mesmo deve ser removido da coluna de objetivos.
\end{remark}

Nas seções a seguir serão apresentados as diversas estratégias para realizar demonstrações, em cada uma delas será explicado como desenvolver o rascunho de prova de cada estratégia.

\section{Demonstrando Implicações}\label{sec:ProvandoImplicacao}

Este texto iniciar a apresentação dos métodos de demonstração pela estratégia conhecida como \textbf{prova direta de implicações}, ela é provavelmente o método de demonstração mais utilizado. O leitor mais experiente em lógica pode notar que na verdade tal estratégia nada mais é que o uso da regra de introdução da implicação.
