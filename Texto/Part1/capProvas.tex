\chapter{Métodos de Demonstração}\label{cap:Demonstracoes}

\begin{introduction}[Tópicos]
	\item Introdução
	\item Demonstrando Implicações
	\item Demonstrando por Absurdo
	\item Demonstrando por Casos
	\item Demonstrando Universalidade
	\item Demonstrando Existência
	\item Demonstrando Unicidade
\end{introduction}

\section{Introdução}\label{sec:Introducao-Demonstracoes}

No capítulo anterior, o leitor encontrou diversas demonstrações dentro da teoria intuitiva (ou Cantoriana) dos conjuntos. Para um leitor iniciante, talvez tenha sido um tanto quanto complicado entender a metodologia usada para construir tais demonstrações. E desde que, as demonstrações são figuras de interesse central no cotidiano dos matemáticos, cientistas da computação e engenheiros de software, em especial aqueles que trabalham com métodos formais, este texto irá fazer uma breve pausa no estudo da teoria dos conjuntos, para apresentar um pouco de teoria da prova ao leitor.

Este capítulo começa então com o seguinte questionamento: Do ponto de vista da ciência da computação qual a importância das demonstrações? Bem a resposta a essa pergunta pode ser dada de dois pontos de vista,  um teórico (purista) e um prático (aplicado ou de engenharia).

Na perspectiva de um cientista da computação puro, as demonstrações de teoremas são a principal ferramenta para investigar os limites dos diferentes modelos de computação propostos \cite{hopcroft2008, linz2006}, assim sendo é de suma importância que o estudante de graduação em ciência da computação receba em sua formação pelo menos o básico para dominar a ``arte'' de provar teoremas, sendo assim preparado para o estudo e a pesquisa pura em computação e(ou) matemática.

Já na visão prática, só existe uma forma segura de garantir que um \textit{software} está livre de erros, essa ``tecnologia'' é exatamente a demonstração das propriedades do \textit{software}. É claro que, mostrar que um \textit{software} não possui erros vai exigir que o \textit{software} seja visto através de um certo nível de formalismo e rigor matemático, mas após essa modelagem através de demonstrações pode-se garantir que um \textit{software} não apresentará erros (quando bem especificado), e assim se algo errado ocorrer foi por fatores externos, tais como defeito no \textit{hardware} por exemplo, e não por falha ou erros com a implementação. Este conceito é o cerne de uma área da engenharia de \textit{software} \cite{pressman2016}, chamada métodos  (ou especificações) formais, sendo essa área o ponto crucial no desenvolvimento de \textit{softwares} para sistemas críticos \cite{sommerville2011}. Isto já mostra a grande importância de programadores e engenheiros de \textit{software} terem em sua formação as bases para o domínio das técnicas de demonstração.

Nas próxima seções deste manuscrito serão descritas as principais técnicas de demonstração de interesse de matemáticos, cientistas da computação e engenheiros formais de \textit{software}. 

\begin{remark}
	Para o leitor que nunca antes teve contato com a lógica matemática recomenda-se que antes de estudar este capítulo, o leitor faça pelo menos um rápido estudo do Capítulo \ref{cap:IntroducaoLogica}.
\end{remark}

Para pode falar sobre métodos de demonstração e poder então descrever como os matemáticos, lógicos e cientistas da computação justificam propriedades usando apenas a argumentação matemática, será necessário fixar algumas nomenclaturas e falar sobre alguns conceitos importantes.

\begin{definition}[Asserção]\label{def:Assercao}
	Uma \textbf{asserção} é qualquer frase declarativa que possa ser expressa na linguagem da lógica simbólica.
\end{definition}

\begin{remark}
	O leitor que conheça lógica nota facilmente que uma asserção é uma proposição ou predicado, para detalhes ver o Capítulo \ref{cap:IntroducaoLogica}.
\end{remark}

Os métodos (ou estratégias) de demonstrações apresentadas neste manuscrito seguem as ideias e a ordem  de apresentação similar ao que foi exposto em \cite{velleman2019comProvar}. Em \cite{velleman2019comProvar} antes de apresentar as provas formais, erá necessário a construção de um rascunho de prova, este rascunho possui similaridades com as demonstrações em provadores de teoremas tais como Coq \cite{coq2013} e Lean \cite{lean2015}, isto é, existe uma separação clara entre dados (hipótese) e os objetivos (em inglês \textit{Goal}) que se quer demonstrar. 

Neste manuscrito por outro lado, não será utilizado a ideia de um rascunho de prova, em vez disso, será usado aqui a noção de \textbf{diagrama de blocos ao estilo Fitch} \cite{broda2007, joaoPavao2014, fitch1953}. Aqui tais diagramas serão encarados como as demonstrações em si, assim diferente de \cite{velleman2019comProvar} não haverá a necessidade de escrever um texto formal após o diagrama da prova ser completado.

Sobre o diagrama de blocos é conveniente explicar sua estrutura, ele consiste de uma série de linhas numeradas de $1$ até $m$, em cada linha está uma informação, sendo esta uma hipótese assumida como verdadeira ou deduzida a partir das informações anteriores a ela. Um diagrama de bloco representa uma prova apenas, porém, uma prova pode conter $n$ subprovas. Cada \textbf{prova} é delimitada no diagrama por um \textbf{bloco}, assim se existe uma subprova $p'$ em uma prova $p$, significa que o diagrama de bloco de $p'$ é interno ao diagrama de bloco de $p$.

Como o leitor poderá vê a seguir o rascunho é um ambiente muito similar a ideia de um programa imperativo em uma linguagem de programação estruturada (como Pascal ou C), no sentido de que, uma demonstração pode ser visto como a combinação de diversos blocos, em que os blocos respeito uma hierarquia e podem está aninhados entre si, a hierarquia dos bloco é determinar por uma indentação. 

Nas próximas seções serão apresentados cada um dos métodos de demonstração, e seguida será gradativamente apresentados exemplos para esboçar ao leitor como é usado o diagrama de blocos e relação a cada método de demonstração.

\section{Demonstrando Implicações}\label{sec:DemonstrandoImplicacoes}

Este manuscrito irá iniciar a apresentação dos métodos de demonstração a partir das estratégias usadas para demonstrar a implicação, isto é, as estratégias usadas para provar asserção da forma: ``se $\alpha$, então $\beta$''.

\begin{method}[Prova Direta]
	Dado uma asserção da forma: ``se $\alpha$, então $\beta$''. A metodologia de prova direta para tal asserção consiste em supor $\alpha$ como sendo verdade e a partir disto deduzir $\beta$.
\end{method}

\begin{remark}
	Obviamente ao fazer a demonstração é necessário identificar que são os componentes $\alpha$ e $\beta$ da implicação.
\end{remark}

Esta estratégia é provavelmente a técnicas mais famosa e usada dentro todos os métodos de demonstração que existem, um conhecedor de lógica pode notar facilmente que tal estratégia nada mais é do que a regra de dedução natural chamada de introdução da implicação\cite{joaoPavao2014}. 

No que diz respeito ao rascunho para tal estratégia o mesmo consistem em criar uma caixa, e dentro desta caixa a primeira linha irá conter a afirmação de que $\alpha$ está sendo assumido com uma hipótese verdadeira, isto é a primeira linha da caixa tem algo como ``suponha $\alpha$''. Em seguida nas próximas $n$ linhas irão acontecer as deduções necessárias até que na linha $n+1$ seja deduzido o $\beta$, assim será fechado a caixa do bloco e na linha $n + 2$ será adicionada a conclusão da hipótese da forma: \textbf{Portanto}, ``se $\alpha$, então $\beta$''.

A seguir serão apresentados exemplos do uso do método de demonstração direto para implicações, ao mesmo tempo será feito

\begin{example}
	Verifique a veracidade da asserção: Se $x$ é impar, então $x^2 + x$ é par.
\end{example}

\begin{solution}
	Em primeiro lugar a asserção não possui qualquer premissa assim pode-se seguir para a tarefa de prova a implicação em si.
	
	\begin{logicproof}{3}
		\begin{subproof}
			\hbox{\textbf{Suponha} } x \hbox{ é impar}, &  \\
			\hbox{assim } x = 2k + 1 \hbox{ com }k \in \Int, & \\
			\hbox{logo } x^2 + x = 2(2k^2 + 3k + 1),& \\
			\hbox{consequentemente } x^2 + x = 2i \hbox{ com } i = 2k^2 + 3k + 1, &  \\
			\hbox{\textbf{então} } x^2 + x \hbox{ é par}. & 
		\end{subproof}
		\hbox{\textbf{Portanto}, se $x$ é impar, então $x^2 + x$ é par} & 
	\end{logicproof}
\end{solution}