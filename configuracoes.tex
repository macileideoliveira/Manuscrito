% ===================================================
% Arquivos para configurações e novos comandos
% ===================================================

% ===================================================
% Informações básicas do documento para a capa
% ===================================================
\title{Computação Formal}
\subtitle{Um Compêndio dos Fundamentos Matemáticos da Computação}
\author{Valdigleis S. Costa}
\institute{Colegiado de Ciência da Computação (CCICOMP)}
%\date{\today}
%\version{1.0}
\bioinfo{Atenção}{Este manuscrito não deve ser vendido}
\extrainfo{Este manuscrito não possui qualquer vínculo editorial.}

% ===================================================
% Configuração dos arquivos da capa
% ===================================================
\logo{logo-blue.png}
\cover{cover.jpg}

% ===================================================
% Definição de cores
% ===================================================
\definecolor{customcolor}{RGB}{32,178,170}
\colorlet{coverlinecolor}{customcolor}

\definecolor{cinzaClaro}{RGB}{178, 189, 193}

% ===================================================
% Comandos e configurações para colocar epígrafes
% nas páginas de separação das partes do texto
% ===================================================
\titleformat{\part}[display]
{\filleft\fontsize{40}{40}\selectfont\scshape}
{\fontsize{90}{90}\selectfont\thepart}
{20pt}
{\thispagestyle{epigraph}}

\setlength\epigraphwidth{.8\textwidth}

\makeatletter
\xpatchcmd\epigraphhead
{\let\@evenfoot}
{\let\@oddfoot\@empty\let\@evenfoot}
{}{}
\makeatother



% ===================================================
% Configuração para o tabuleiro de prova
% ===================================================
\def\repls #1\endrepls{%
    \begingroup
    \replscriptwidth=0pt\relax
    \replgivenswidth=0pt\relax
    \replgoalswidth =0pt\relax
    % XXX: restorerepls... here, not restorerepl...
    \ifreplhidelineno\def\restorereplshidelineno{\global\replhidelinenotrue}%
    \else            \def\restorereplshidelineno{\global\replhidelinenofalse}%
    \fi
    \let\shownum=\replhidenum
    \let\hidenum=\replhidenum
    \def\maxproof##1: ##2; {\IfStrContains {##1} {m} {\setreplmaxscript{$##2$}} {\setreplmaxscript{##2}}}%
    \def\maxgiven##1: ##2; {\IfStrContains {##1} {t} {\setreplmaxgiven {##2}}   {\setreplmaxgiven {$##2$}}}%
    \def\maxgoal ##1: ##2; {\IfStrContains {##1} {t} {\setreplmaxgoal  {##2}}   {\setreplmaxgoal  {$##2$}}}%
    #1%
    \restorereplshidelineno
    \endgroup
}
