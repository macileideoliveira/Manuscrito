% ===================================================
% Arquivos para configurações e novos comandos
% ===================================================

% ===================================================
% Informações básicas do documento para a capa
% ===================================================
\title{Computação Formal}
\subtitle{Um Compêndio dos Fundamentos Matemáticos da Computação}
\author{Valdigleis S. Costa}
\institute{Colegiado de Ciência da Computação (CCICOMP)}
%\date{\today}
%\version{1.0}
\bioinfo{Atenção}{Este manuscrito não deve ser vendido}
\extrainfo{Este manuscrito não possui qualquer vínculo editorial.}

% ===================================================
% Configuração dos arquivos da capa
% ===================================================
\logo{logo-blue.png}
\cover{cover.jpg}

% ===================================================
% Definição de cores
% ===================================================
\definecolor{customcolor}{RGB}{32,178,170}
\colorlet{coverlinecolor}{customcolor}

\definecolor{cinzaClaro}{RGB}{178, 189, 193}

% ===================================================
% Comandos e configurações para colocar epígrafes
% nas páginas de separação das partes do texto
% ===================================================
\titleformat{\part}[display]
{\filleft\fontsize{40}{40}\selectfont\scshape}
{\fontsize{90}{90}\selectfont\thepart}
{20pt}
{\thispagestyle{epigraph}}

\setlength\epigraphwidth{.8\textwidth}

\makeatletter
\xpatchcmd\epigraphhead
{\let\@evenfoot}
{\let\@oddfoot\@empty\let\@evenfoot}
{}{}
\makeatother

% ===================================================
% Configuração alguns comando básicos
% ===================================================
\newcommand{\Nat}{\mathbb{N}}
\newcommand{\Int}{\mathbb{Z}}
\newcommand{\Rac}{\mathbb{Q}}
\newcommand{\Irr}{\mathbb{I}}
\newcommand{\Rea}{\mathbb{R}}
\newcommand{\Com}{\mathbb{C}}

% ===================================================
% Configuração do ambiente para os métodos de prova
% ===================================================
%\newtheorem{method}{Método de Demonstração}[chapter]