\chapter{Métodos de Demonstração}\label{cap:Demonstracoes}

\epigraph{Mais um colchão, mais uma demonstração}{Paul Erdös}

\epigraph{Um matemático é uma máquina que transforma café em teoremas.}{Paul Erdös}

\section{Introdução}\label{sec:Introducao-Demonstracoes}

No capítulo anterior, o leitor encontrou diversas demonstrações dentro da teoria intuitiva (ou Cantoriana) dos conjuntos. Para um leitor iniciante talvez tenha sido um tanto quanto complicado entender a metodologia usada para construir tais demonstrações. E desde que, as demonstrações são figuras de interesse central no cotidiano dos matemáticos, cientistas da computação e engenheiros de software, em especial aqueles que trabalham com métodos formais, este texto irá fazer uma breve pausa no estudo da teoria dos conjuntos, para apresentar um pouco de teoria da prova ao leitor.

Este capítulo começa então com o seguinte questionamento: Do ponto de vista da ciência da computação qual a importância das demonstrações? Bem a resposta a essa pergunta pode ser dada de dois pontos de vista,  um teórico (purista) e um prático (aplicado ou de engenharia).

Na perspectiva de um cientista da computação puro, as demonstrações de teoremas, proposições, lema, corolários e propriedades são a principal ferramenta para investigar os limites dos diferentes modelos de computação propostos \cite{hopcroft2008, linz2006}, assim sendo é de suma importância que o estudante de graduação em ciência da computação receba em sua formação pelo menos o básico para dominar a ``arte'' de provar teoremas, sendo assim preparado para o estudo e a pesquisa pura em computação e(ou) matemática.

Já na visão prática, só existe uma forma segura de garantir que um \textit{software} está livre de erros, essa ``tecnologia'' é exatamente a demonstração das propriedades do \textit{software}. É claro que, mostrar que um \textit{software} não possui erros vai exigir que o \textit{software} seja visto através de um certo nível de formalismo e rigor matemático, mas após essa modelagem através de demonstrações pode-se garantir que um \textit{software} não apresentará erros (quando bem especificado), e assim se algo errado ocorrer foi por fatores externos, tais como defeito no \textit{hardware} por exemplo, e não por falha ou erros com a implementação. Este conceito é o cerne de uma área da engenharia de \textit{software} \cite{pressman2016}, chamada métodos  (ou especificações) formais, sendo essa área o ponto crucial no desenvolvimento de \textit{softwares} para sistemas críticos \cite{sommerville2011}. Isto já mostra a grande importância de programadores e engenheiros de \textit{software} terem em sua formação as bases para o domínio das técnicas de demonstração.

Nas próxima seções deste manuscrito serão descritas as principais técnicas de demonstração de interesse de matemáticos, cientistas da computação e engenheiros formais de \textit{software}. 

\begin{remark}
	Para o leitor que nunca antes teve contato com a lógica matemática recomenda-se que antes de estudar este capítulo, o leitor faça pelo menos um rápido estudo do Capítulo \ref{cap:IntroducaoLogica}.
\end{remark}

Para pode falar sobre métodos de demonstração e poder então descrever como os matemáticos, lógicos e cientistas da computação justificam propriedades usando apenas a argumentação matemática, será necessário fixar algumas nomenclaturas e falar sobre alguns conceitos importantes.

\begin{definition}[Asserção]\label{def:Assercao}
	Uma \textbf{asserção} é qualquer frase declarativa que possa ser expressa na linguagem da lógica simbólica.
\end{definition}

\begin{remark}
	O leitor que conheça lógica nota facilmente que uma asserção é uma proposição ou predicado, para detalhes ver o Capítulo \ref{cap:IntroducaoLogica}.
\end{remark}

Os métodos (ou estratégias) de demonstrações apresentadas neste manuscrito seguem as ideias e a ordem  de apresentação similar ao que foi exposto em \cite{velleman2019comProvar}. Em \cite{velleman2019comProvar} antes de apresentar as provas formais, erá necessário a construção de um rascunho de prova, este rascunho possui similaridades com as demonstrações em provadores de teoremas tais como Coq \cite{coq2013} e Lean \cite{lean2015}, isto é, existe uma separação clara entre dados (hipótese) e os objetivos (em inglês \textit{Goal}) que se quer demonstrar. 

Neste manuscrito por outro lado, não será utilizado a ideia de um rascunho de prova, em vez disso, será usado aqui a noção de \textbf{diagrama de blocos} \cite{broda2007}. Aqui tais diagramas serão encarados como as demonstrações em si, assim diferente de \cite{velleman2019comProvar} não haverá a necessidade de escrever um texto formal após o diagrama da prova ser completado.

Sobre o diagrama de blocos é conveniente explicar sua estrutura, ele consiste de uma série de linhas numeradas de $1$ até $m$, em cada linha está uma informação, sendo esta uma hipótese assumida como verdadeira ou deduzida a partir das informações anteriores a ela ou ainda um resultado (ou definição) válido(a) conhecido(a). Um diagrama de bloco representa uma prova, porém, uma prova pode conter $n$ subprovas. Cada \textbf{prova} é delimitada no diagrama por um \textbf{bloco}, assim se existe uma subprova $p'$ em uma prova $p$, significa que o diagrama de bloco de $p'$ é interno ao diagrama de bloco de $p$. Na linha abaixo de todo bloco sempre estará a conclusão que se queria demonstrar, isto é, abaixo de cada bloco está a asserção que tal bloco demonstra.

Cada linha no diagrama começa com algum termo reservado (em um sentido similar ao de palavra reservada de linguagem de programação \cite{aho2007, cooper2017}) escrito em negrito\footnote{A escrita dos termos reservados em negrito em geral será usada para que o leitor consiga identificar o que é informação último da prova e o que é apenas um artificio textual para dá melhor entendimento a demonstração.}, esses termos reservados tem três naturezas distintas: inicialização de bloco, ligação e conclusão de blocos. Tais palavras podem variar a depender do material sobre demonstrações que o leitor possa encontrar na literatura neste manuscrito serão usando os seguintes conjuntos de palavras:

\begin{itemize}
	\item Termos de inicialização de bloco: \textbf{Suponha}, \textbf{Deixe ser}, \textbf{Assuma} e \textbf{Considere};
	\item Termos de ligação: \textbf{mas}, \textbf{tem-se que}, \textbf{então}, \textbf{assim}, \textbf{logo}, \textbf{além disso}, \textbf{desde que} e \textbf{dessa forma};
	\item Termos de conclusão de bloco: \textbf{Portanto}, \textbf{Dessa forma}, \textbf{Consequentemente} e \textbf{Logo por contra positiva}.
\end{itemize}  

%O exemplo a seguir ilustra a construção de um diagrama de bloco para uma demonstração de uma famosa asserção.

\begin{example}\label{exe:DiagramaProva1}
	Demonstre a asserção: Dado $A = \{x \in \mid x = 2i, i \in \mathbb{Z}\}$ e $B = \{x \in \mid x = 2j - 2, j \in \mathbb{Z}\}$ tem-se que $A = B$.
	{\scriptsize
	\begin{logicproof}{3}
		\begin{subproof}
			\text{\textbf{Assuma} que } x \text{ é um elemento qualquer, }&\\ 
			\begin{subproof}
				\text{\textbf{Suponha} que } x \in A, &\\
				\text{\textbf{assim }} x = 2i, i \in \mathbb{Z}, &\\
				\text{\textbf{desde que} existe } k \in \mathbb{Z} \text{ tal que } k = j - 1 \text{ e fazendo } i = k, &\\
				\text{\textbf{tem-se que} } x = 2i = 2k = 2(j -1) = 2j - 2.&\\
				\text{\textbf{Então} } x \in B. &
			\end{subproof}
			\text{\textbf{Portanto}, Se } x \in A, \text{ então } x \in B.&  
		\end{subproof}
		\text{\textbf{Consequentemente}, } \forall x.[\text{ Se } x \in A, \text{ então } x \in B].&\\
		\begin{subproof}
			\text{\textbf{Assuma} que } x \text{ é um elemento qualquer, }&\\ 
			\begin{subproof}
				\text{\textbf{Suponha} que } x \in B. &\\
				\text{\textbf{assim }} x = 2j -2, j \in \mathbb{Z}, &\\
				\text{\textbf{desde que} } j - 1 \in \mathbb{Z}, \text{ pode-se fazer } i = j - 1&\\
				\text{\textbf{logo }} x = 2i. &\\
				\text{\textbf{Então} } x \in A. &
			\end{subproof}
			\text{\textbf{Portanto}, Se } x \in B, \text{ então } x \in A.&  
		\end{subproof}
		\text{\textbf{Consequentemente}, } \forall x.[\text{ Se } x \in B, \text{ então } x \in A].&\\
		\text{\textbf{Portanto,} } A \subseteq B \text{ e } B \subseteq A \text{ assim por definição } A = B.&
	\end{logicproof}
	}
\end{example}

Neste momento o exemplo anterior serve apenas para esboçar a ideia de um diagrama de bloco para uma demonstração, note que fica evidente que a depender da situação alguns termos de ligação são melhores que outros, e o mesmo também é válido para os termos de inicialização e conclusão de bloco. 

Aqui não será detalhando a aplicação dos métodos de demonstração usado na demonstração do Exemplo \ref{exe:DiagramaProva1}, mas nas próximas seções serão apresentados cada um dos métodos de demonstração, e seguida será gradativamente apresentados exemplos para esboçar ao leitor como é usado o diagrama de blocos e relação a cada método de demonstração..

Como o leitor pode ter notado pelo diagrama de bloco no Exemplo \ref{exe:DiagramaProva1}, é possível enxergar o diagrama como ambiente muito similar a ideia de um programa imperativo em uma linguagem de programação estruturada (como Pascal ou C), no sentido de que, uma demonstração pode ser visto como a combinação de diversos blocos, em que os blocos respeito uma hierarquia e podem está aninhados entre si, a hierarquia dos bloco é determinar por uma indentação\footnote{Indentação é um termo utilizando em código fonte de um programa, serve para ressaltar ou definir a estrutura do algoritmo. Na maioria das linguagens de programação, a indentação é empregada com o objetivo de ressaltar a estrutura do algoritmo, aumentando assim a legibilidade do código, porém a linguagem de programação em que a indentação é parte da própria da gramática da linguagem.}. 

\section{Demonstrando Implicações}\label{sec:DemonstrandoImplicacoes}

Este manuscrito irá iniciar a apresentação dos métodos de demonstração a partir das estratégias usadas para demonstrar a implicação, isto é, as estratégias usadas para provar asserção da forma: ``se $\alpha$, então $\beta$''.

\begin{definition}[Prova Direta (PD)]
	Dado uma asserção da forma: ``se $\alpha$, então $\beta$''. A metodologia de prova direta para tal asserção consiste em supor $\alpha$ como sendo verdade e a partir disto deduzir $\beta$.
\end{definition}

\begin{remark}
	Obviamente ao fazer a demonstração é necessário identificar que são os componentes $\alpha$ e $\beta$ da implicação.
\end{remark}

Esta estratégia é provavelmente a técnicas mais famosa e usada dentre todos os métodos de demonstração que existem, um conhecedor de lógica pode notar facilmente que tal estratégia nada mais é do que a regra de dedução natural chamada de introdução da implicação\cite{joaoPavao2014}. 

No que diz respeito ao diagrama tal estratégia consistem em: (1) criar um bloco, e dentro deste bloco na primeira linha irá conter a afirmação de que $\alpha$ está sendo assumido com uma hipótese verdadeira; (2) nas próximas $n$ linhas irão acontecer as deduções necessárias até que na linha $n+2$ seja deduzido o $\beta$ e o bloco é fechado e (3) na linha $n + 3$ será adicionada a conclusão do bloco. A seguir serão apresentados exemplos do uso do método de demonstração direto para implicações e seu uso junto com o diagrama de bloco.

\begin{example}\label{exe:DiagramaProva2}
	Para demonstrar da asserção \textbf{``Se $x$ é ímpar, então $x^2 + x$ é par''} deve-se usar o método de demonstração \textbf{PD}, assim a prova começa abrindo um bloco e inserido na primeira linha a hipótese de que o antecedente \textbf{$x$ é um número ímpar} é verdadeira, ou seja, tem-se:
	{\scriptsize
	\begin{logicproof}{2}
		\begin{subproof}
			\text{\textbf{Suponha} que } x \in \mathbb{N}, &
		\end{subproof}
	\end{logicproof}
	}
	\noindent em seguida  pode-se na linha 2 deduzir a forma que $x$, mudando o diagrama para:
	{\scriptsize
		\begin{logicproof}{2}
			\begin{subproof}
				\text{\textbf{Suponha} que } x \in \mathbb{N}, &\\
				\text{\textbf{logo} } x = 2k + 1, k  \in \mathbb{N},&
			\end{subproof}
		\end{logicproof}
	}
	\noindent  agora nas próximas duas linhas pode-se deduzir respectivamente as formas (ou valores) de $x^2$ e $x^2 + x$, assim o diagrama é atualizado para:
	{\scriptsize
		\begin{logicproof}{2}
			\begin{subproof}
				\text{\textbf{Suponha} que } x \text{ é um número ímpar}, &\\
				\text{\textbf{logo} } x = 2k + 1, k  \in \mathbb{Z},&\\
				\text{\textbf{assim} } x^2 = 4k^2 + 4k + 1, k \in \mathbb{Z},&\\
				\text{\textbf{dessa forma} } x^2 + x= 2((2k^2 + 2k) + k + 1), k \in \mathbb{Z}.&
			\end{subproof}
		\end{logicproof}
	}
	\noindent agora note que $x^2 + x= 2((2k^2 + 2k) + k + 1)$ pode ser reescrito (por substituição) como $x^2 + x= 2j$ com $j = (2k^2 + 2k) + k + 1$, essa dedução é inserida na linha de número $5$ atualizando o diagrama para:
	{\scriptsize
		\begin{logicproof}{2}
			\begin{subproof}
				\text{\textbf{Suponha} que } x \text{ é um número ímpar}, &\\
				\text{\textbf{logo} } x = 2k + 1, k  \in \mathbb{Z},&\\
				\text{\textbf{assim} } x^2 = 4k^2 + 4k + 1, k \in \mathbb{Z},&\\
				\text{\textbf{dessa forma} } x^2 + x= 2((2k^2 + 2k) + k + 1), k \in \mathbb{Z}.&\\
				\text{\textbf{logo} } x^2 + x= 2j \text{ com } j = (2k^2 + 2k) + k + 1, k \in \mathbb{Z}.&
			\end{subproof}
		\end{logicproof}
	}
	\noindent agora na sexta linha do diagrama pode-se então deduzir a partir da informação na linha de número $5$ que $x^2 + x$ é um número par, assim o diagrama muda para a forma:
	{\scriptsize
		\begin{logicproof}{2}
			\begin{subproof}
				\text{\textbf{Suponha} que } x \text{ é um número ímpar}, &\\
				\text{\textbf{logo} } x = 2k + 1, k  \in \mathbb{Z},&\\
				\text{\textbf{assim} } x^2 = 4k^2 + 4k + 1, k \in \mathbb{Z},&\\
				\text{\textbf{dessa forma} } x^2 + x= 2((2k^2 + 2k) + k + 1), k \in \mathbb{Z}.&\\
				\text{\textbf{logo} } x^2 + x = 2j \text{ com } j = (2k^2 + 2k) + k + 1, k \in \mathbb{Z},&\\
					\text{\textbf{então} } x^2 + x \text{ por definição é um número par}.&
			\end{subproof}
		\end{logicproof}
	}
	\noindent note porém que a informação na deduzida na linha de número $6$ é exatamente o consequente da implicação que se queria deduzir. Portanto, o objetivo interno ao bloco foi atingido, pode-se então fechar o bloco introduzindo abaixo dele a conclusão do bloco, ou seja, na linha de número $7$ é escrito que o antecedente de fato implica no consequente, assim o diagrama fica da forma:
	{\scriptsize
		\begin{logicproof}{2}
			\begin{subproof}
				\text{\textbf{Suponha} que } x \text{ é um número ímpar}, &\\
				\text{\textbf{logo} } x = 2k + 1, k  \in \mathbb{Z},&\\
				\text{\textbf{assim} } x^2 = 4k^2 + 4k + 1, k \in \mathbb{Z},&\\
				\text{\textbf{dessa forma} } x^2 + x= 2((2k^2 + 2k) + k + 1), k \in \mathbb{Z}.&\\
				\text{\textbf{logo} } x^2 + x = 2j \text{ com } j = (2k^2 + 2k) + k + 1, k \in \mathbb{Z},&\\
				\text{\textbf{então} } x^2 + x \text{ por definição é um número par}.&
			\end{subproof}
			\text{\textbf{Portanto}, Se } x \text{ é ímpar, então } x^2 + x \text{ é par.} &
		\end{logicproof}
	}
	\noindent assim o objetivo a ser demonstrado foi atingido e, portanto, a prova está completa.
\end{example}

\begin{note}
	Na demonstração apresentada no Exemplo \ref{exe:DiagramaProva2} as justificativas da evolução do diagrama fora apresentadas passo a passo e separadas do diagrama, isso foi adotado nesse primeiro exemplo para detalhar a evolução da demonstração ao leitor, entretanto, isso não é o padrão, o normal (que será adotado) é que a justificativa (caso necessário\footnote{Quando a justificativa for trivial, ela não será inserida na prova.}) da dedução de uma linha seja inserida a direita da informação deduzida, separada desta pelo símbolo ---. A partir deste ponto até o fim deste capítulo será usado esta forma de escrita.
\end{note}

\begin{remark}
	Nas justificativas das provas as palavras definição, associatividade, comutatividade serão  abreviadas para DEF, ASS, COM respetivamente
\end{remark}

\begin{example}\label{exe:DiagramaProva3}
	Demonstração da asserção: Se $n$ é múltiplo de 4, então também é múltiplo de 2.
	
	{\scriptsize
		\begin{logicproof}{2}
			\begin{subproof}
				\text{\textbf{Suponha} que } n \text{ é múltiplo de } 4, & --- Hipótese\\
				\text{\textbf{logo} } x = 4k, k  \in \mathbb{Z},& --- DEF de múltiplo de $4$\\
				\text{\textbf{assim} } x = (2 \cdot 2)k, k \in \mathbb{Z},& --- Reescrita da linha $2$\\
				\text{\textbf{dessa forma} } x= 2(2k), k \in \mathbb{Z}.& --- ASS da multiplicação\\
				\text{\textbf{logo} } x= 2i, \text{ com } i = 2k, k \in \mathbb{Z}.& --- Reescrita da linha $4$\\
				\text{\textbf{então} } x \text{ é múltiplo de } 2.& --- DEF de múltiplo de 2
			\end{subproof}
			\text{\textbf{Portanto}, Se } n \text{ é múltiplo de } 4, \text{ então também é múltiplo de 2}. & --- Conclusão da PD (linhas 1-7)
		\end{logicproof}
	}
\end{example}

O leitor deve ter notado nos Exemplos \ref{exe:DiagramaProva2} e \ref{exe:DiagramaProva3} as demonstrações sempre iniciam das hipótese que estão sendo assumidas, isto é, os antecedentes das implicações, isso ocorrer por que nenhuma informação adicional (necessária) é apresentada como premissa, há caso entretanto, que as premissas são importantes para o desenvolvimento da prova, como será visto no próximo exemplo.

\begin{example}\label{exe:DiagramaProva4}
	Demonstração da asserção: Dado $m$ inteiro maior ou igual que 5 e $n$ um número ímpar positivo. Se $n$ é ímpar, então $m+n$ é um par maior ou igual que 6.
	
	{\scriptsize
		\begin{logicproof}{2}
			m \geq 5, m \in \mathbb{Z} & --- Premissa\\
			n = 2k + 1, k \in  \mathbb{Z}^+ & --- Premissa\\
			\begin{subproof}
				\text{\textbf{Suponha} que } m \text{ é ímpar}, & --- Hipótese\\
				\text{\textbf{logo} } m = 2j +1, j  \in \mathbb{Z},& --- DEF  de número ímpar\\
				\text{\textbf{assim} } m + n = 2(j + k + 1)&\\
				\text{\textbf{desde que} } m \geq 5 \text{ tem-se que } j \geq 2, &\\
				\text{\textbf{mas} como } n \in \mathbb{Z}_+ \text{ tem-se que } k \geq 0, &\\
				\text{\textbf{assim }} j + k + 1 \geq 3, & --- Direto das linhas $6$ e $7$\\
				\text{\textbf{logo} } 2(j + k + 1) \geq 6 &\\
				\text{\textbf{então} } m + n \geq 6. & --- Reescrita da linha 9 
			\end{subproof}
			\text{\textbf{Portanto}, Se } n \text{ é ímpar}, \text{ então } m + n \text{ é um par maior ou igual } 6. & --- Conclusão da PD (linhas 3-10)
		\end{logicproof}
	}
\end{example}

%\begin{remark}
%	No Exemplo \ref{exe:DiagramaProva4} as linhas $6,7$ e $8$ da demonstração foram destacada na cor purpura por que juntas elas dizem respeito a um ferramenta da lógica que será explicado mais adiante neste capítulo.
%\end{remark}
%	{\color{pupChapters}


\begin{example}\label{exe:DiagramaProva5}
	Demonstração da asserção: Dado $m,n \in \mathbb{R}$ e $3m + 2n \leq 5$. Se $m > 1$, então $n < 1$.
	
	{\scriptsize
		\begin{logicproof}{2}
			m, n \in \mathbb{R} & --- Premissa\\
			3m + 2n \leq 5 & --- Premissa\\
			\begin{subproof}
				\text{\textbf{Suponha} que } m > 1, & --- Hipótese\\
				\text{\textbf{assim} } 3m > 3, &\\
				\text{\textbf{desde que }} 3m \leq 5 - 2n, &\\
				\text{\textbf{tem-se que }} 3 < 3m  \leq  5 - 2n& --- Direto das linhas $4$ e $5$\\
				\text{\textbf{logo} } 3 < 5 - 2n,&\\
				\text{\textbf{assim} } 3 + 2n < 5,&\\
				\text{\textbf{então} } n < 1.&
			\end{subproof}
			\text{\textbf{Portanto}, Se } n > 1, \text{ então }  n < 1. & --- Conclusão da PD (linhas 3-10)
		\end{logicproof}
	}
\end{example}

Além do método de prova direta asserções que são implicações podem ser provadas por um segundo método, chamado método da contra positiva (ou contraposição). Como dito em \cite{menezes2010MD}, o método da contra positiva se baseia na equivalência semântica\footnote{De forma mais rigorosa o que de fato sustenta o método de demonstração por contra positiva é a corretude e a completude da lógica de primeira ordem, e não simplesmente uma questão semântica.} (ver Capítulo \ref{cap:LogicaProposicional}) da expressão ``Se $\alpha$, então $\beta$'' com a expressão ``Se não $\beta$, então não $\alpha$''. Formalmente o método de demonstração por contra positiva é como se segue.

\begin{definition}[Prova por Contra Positiva (PCP)]
	Dado uma asserção da forma: ``se $\alpha$, então $\beta$''. A metodologia de prova por contra positiva para tal asserção consiste em demonstrar usando PD a asserção ``se não $\beta$, então não $\alpha$'', em seguida concluir (ou enunciar) que a veracidade de ``se $\alpha$, então $\beta$'' segue da veracidade de ``se não $\beta$, então não $\alpha$''.
\end{definition}

\begin{remark}
	Como pode ser visto no Capítulo \ref{cap:IntroducaoLogica} a asserção não $\alpha$ representa a negação da asserção $\alpha$.
\end{remark}

Agora em termos do diagrama de blocos o método PCP apresenta o seguinte raciocínio de construção do diagrama: (1) abrir um bloco com a primeira linha em braco; (2) realizar em um bloco (interno) a  demonstração  de que ``se não $\beta$, então não $\alpha$'' e (3) após a conclusão deste segundo bloco, o primeiro bloco é fechado, e sua conclusão consiste na informação ``se $\alpha$, então $\beta$'' e a justificativa de tal informação é simplesmente a conclusão PCP das linhas $i$-$j$, onde $i$-$j$ diz respeito ao intervalo contendo as  linhas do bloco e da conclusão da demonstração de ``se não $\beta$, então não $\alpha$''.

\begin{note}
	Vale salientar que a linha em branco no início é apenas um fator estético adotado neste manuscrito, para tornar a leitura do diagrama da demonstração mais agradável. A depender da situação do diagrama seu uso pode ser desconsiderado´.
\end{note}

\begin{example}\label{exe:DiagramaProva6}
	Demonstração da asserção: Se $n! > (n+1)$, então $n > 2$.
	{\scriptsize
		\begin{logicproof}{3}
			\begin{subproof}
				&  \\
				\begin{subproof}
					\text{\textbf{Suponha} que } n \leq 2, & --- Hipótese\\
					\text{\textbf{assim} } n = 0, n = 1 \text{ ou } n = 2 & --- Direto da linha $2$\\
					\text{\textbf{logo} } n! = 1 \text{ ou } n! = 2, & --- Da linha $3$ e da DEF de fatorial\\
					\text{\textbf{então} } n! \leq (n + 1) \text{ com } n \leq 2 .& --- Direto das linhas $3$ e $4$
				\end{subproof}
				\text{\textbf{Portanto}, Se } n \leq 2, \text{ então }  n! \leq (n + 1). & --- Conclusão da PD (linhas 2-5)
			\end{subproof}
			\text{\textbf{Logo por contra positiva}, Se }n! > (n+1), \text{ então }n > 2. & --- Conclusão da PCP (linhas 2-6)
		\end{logicproof}
	}
\end{example}

\begin{example}\label{exe:DiagramaProva7}
	Demonstração da asserção: Se $n \neq 0$, então $n + c \neq c$.
	{\scriptsize
		\begin{logicproof}{3}
			\begin{subproof}
				&  \\
				\begin{subproof}
					\text{\textbf{Suponha} que } n + c = c, & --- Hipótese\\
					\text{\textbf{assim} } n + c - c = c -c  & \\
					\text{\textbf{logo}, } n + 0 = 0, &\\
					\text{\textbf{então} } n = 0 .&
				\end{subproof}
				\text{\textbf{Portanto}, Se } n + c = c, \text{ então }  n  = 0. & --- Conclusão da PD (linhas $2$-$5$)
			\end{subproof}
			\text{\textbf{Logo por contra positiva}, Se }n \neq 0, \text{ então } n + c \neq c. & --- Conclusão da PCP (linhas $2$-$6$)
		\end{logicproof}
	}
\end{example}

\begin{example}\label{exe:DiagramaProva8}
	Demonstração da asserção: Dado três números $x, y, z \in \mathbb{R}$ com $x > y$. Se $xz \leq yz$, então $z \leq 0$.
	{\scriptsize
		\begin{logicproof}{3}
			x, y, z \in \mathbb{R}, & --- Premissa\\
			x > y, & --- Premissa\\
			\begin{subproof}
				&  \\
				\begin{subproof}
					\text{\textbf{Suponha} que } z > 0, & --- Hipótese\\
					\text{\textbf{então} } xz > yz,  & --- Das linhas $2$ e $4$ e da monotonicidade da multiplicação em $\mathbb{R}$
				\end{subproof}
				\text{\textbf{Portanto}, Se } z > 0, \text{ então }  xz > yz. & --- Conclusão da PD (linhas $2$-$5$)
			\end{subproof}
			\text{\textbf{Logo por contra positiva}, Se }xz \leq yz, \text{ então } y \leq 0. & --- Conclusão da PCP (linhas $3$-$6$)
		\end{logicproof}
	}
\end{example}

\begin{example}\label{exe:DiagramaProva9}
	Demonstração da asserção: Se $n^2$ é par, então $n$ é par.
	{\scriptsize
		\begin{logicproof}{3}
			\begin{subproof}
				&  \\
				\begin{subproof}
					\text{\textbf{Suponha} que } n \text{ não é par}, & --- Hipótese\\
					\text{\textbf{logo} } n = 2k + 1 \text{ com } k \in \mathbb{Z},  & --- DEF de paridade\\
					\text{\textbf{assim} } n^2 = 4k^2 + 4k + 1 \text{ com } k \in \mathbb{Z},&\\
					\text{\textbf{dessa forma }} n^2 = 2j + 1 \text{ com } j = 2k^2 + 2k,& --- Reescrita da linha $4$\\
					\text{\textbf{então} } n^2 \text{ não é par}& --- DEF de paridade
				\end{subproof}
				\text{\textbf{Portanto}, Se } n  \text{ não é par}, \text{ então } n^2  \text{ não é par}. & --- Conclusão da PD (linhas $2$-$6$)
			\end{subproof}
			\text{\textbf{Logo por contra positiva}, Se }n^2 \text{ é par}, \text{ então } n \text{ é par}. & --- Conclusão da PCP (linhas $2$-$6$)
		\end{logicproof}
	}
\end{example}

\section{Demonstração por Absurdo}\label{sec:DemonstracaoAbsurdo}

O método de demonstração por redução ao absurdo\footnote{\textit{Reductio ad absurdum} em latim.} (ou por contradição) tem por objetivo provar que a asserção $\alpha$ junto com as premissas (se houverem) é verdadeira a partir da prova de que a suposição de que a asserção ``não $\alpha$'' seja verdadeira junto das mesmas premissas (mencionadas anteriormente) gera um absurdo (ou contradição). O fato deste absurdo seja gerado, permite concluir que suposição de que a asserção ``não $\alpha$'' seja verdadeira é ridícula, ou seja, ``não $\alpha$'' tem que ser falsa e, portanto, a asserção $\alpha$ tem que ser verdadeira. Esse argumento é garantido pelos princípios da não contradição e do terceiro excluído, ambos mencionados na Seção \ref{sec:Tipos-Logica-Aplicacoes} do Capítulo \ref{cap:IntroducaoLogica}.

%Em uma aspecto mais formal, isto é, em termos da linguagem de lógica (ver a Parte \ref{Parte:logica} desde manuscrito), a prova por redução ao absurdo de uma asserção $\alpha$ ser verdadeira é garantido pela prova de que $\Gamma\vdash \neg \alpha \Rightarrow \bot$, com $\Gamma$ sendo o conjunto de premissas.

\begin{definition}[Prova por Redução ao Absurdo (RAA)]
	A metodologia para uma demonstração por redução ao absurdo de uma asserção $\alpha$, consiste em supor que não $\alpha$ é uma hipótese verdadeira, então deduzir um absurdo (ou contradição). Em seguida concluir que dado que a partir de não $\alpha$ foi produzido um absurdo pode-se afirma que $\alpha$ é verdadeiro.
\end{definition} 

Em termos do diagrama de blocos o método RAA consiste nos seguintes passo: (1) abrir um bloco  cuja primeira linha é vazia; (2) abri um bloco interno em que na primeira linha deste bloco o termo de inicialização do bloco (já listados anteriormente) é seguida da expressão ``por absurdo'' e da asserção não $\alpha$; (3) em seguida nas próximas $n$ linhas irão acontecer as deduções necessárias até que na linha $n+2$ seja deduzido o absurdo (ou uma contradição) e o bloco é fechado, inserido na linha $n +3$ a informação de que ``Se não $\alpha$, então $\bot$'' e é fechado o bloco externo e (4) na linha $n + 4$ será adicionada a conclusão do bloco externo, contendo algo como `` Portanto, $\alpha$ é verdadeiro''.

\begin{remark}
	Como explicado na Parte \ref{Parte:logica} deste manuscrito, o símbolo $\bot$ é usado para denotar o absurdo.
\end{remark}

\begin{example}\label{exe:DiagramaProva10}
	Demonstração da asserção: $\sqrt{2} \not\in \mathbb{Q}$.
	{\scriptsize
		\begin{logicproof}{3}
			\begin{subproof}
				&  \\
				\begin{subproof}
					\text{\textbf{Assuma por absurdo} que } \sqrt{2} \in \mathbb{Q}, & --- Hipótese \\
					\text{\textbf{logo} existem } a,b \in \mathbb{Z} \text{ tal que }  \sqrt{2} = \frac{a}{b} \text{ sendo } b \neq 0, mdc(a, b) = 1 &\\
					\text{\textbf{logo} } a^2 = 2b^2, \text{ ou seja, } a^2 \text{ é par},  &\\
					\text{\textbf{dessa forma} } a = 2i \text{ com } i \in \mathbb{Z},& --- Pela linha $4$ e o Exemplo \ref{exe:DiagramaProva9}\\
					\text{\textbf{logo} } b^2 = 2i^2 \text{ com } i \in \mathbb{Z},&\\
					\text{\textbf{dessa forma} } b = 2j \text{ com } j \in \mathbb{Z},& --- Pela linha $6$ e o Exemplo \ref{exe:DiagramaProva9}\\
					\text{\textbf{assim} } mdc(a, b) \geq 2,& --- Direto das linhas $5$ e $7$\\
					\text{\textbf{mas} } mdc(a, b) = 1  \text{ e } mdc(a, b) \geq 2 \text{ é um absurdo}.& --- Direto das linhas $3$ e $8$
				\end{subproof}
				\text{\textbf{Portanto}, Se } \sqrt{2} \in \mathbb{Q}, \text{ então }  \bot. & --- Conclusão da PD (linhas $2$-$10$)
			\end{subproof}
			\text{\textbf{Consequentemente}, } \sqrt{2} \notin \mathbb{Q}. & --- Conclusão da RAA (linhas $2$-$11$)
		\end{logicproof}
	}
\end{example}

\begin{note}
	O termo $mdc$ que aparece na demonstração no Exemplo \ref{exe:DiagramaProva10} é a abreviação para o conceito de máximo divisor comum.
\end{note}

\begin{remark}
	Note que internamente a prova por redução ao absurdo de uma asserção $\alpha$, deve ser demonstrado a asserção: não $\alpha \Rightarrow \bot$.
\end{remark}

\begin{example}\label{exe:DiagramaProva11}
	Demonstração da asserção: Não existe solução inteira positiva não nula para a equação diofantina\footnote{Equações diofantinas são equações polinomiais, que permite a duas ou mais variáveis assumirem apenas valores inteiros.} $x^2 - y^2 = 1$.
	{\scriptsize
		\begin{logicproof}{3}
			\begin{subproof}
				&  \\
				\begin{subproof}
					\text{\textbf{Assuma por absurdo} que } \exists x, y \in \mathbb{Z}_+^* \text{ tal que } x^2 - y^2 = 1 , & --- Hipótese\\
					\text{\textbf{logo }}  x, y \in \mathbb{Z}_+^* \text{ tem-se que } min(x, y) = 1 \text{ e tem-se que } (x-y)(x+y) = 1,&\\
					\text{\textbf{assim} } x - y = 1 \text{ e } x + y = 1 \text{ ou }  x - y = -1 \text{ e } x + y = -1, & --- Por $x, y \in \mathbb{Z}_+^*$\\
					\text{\textbf{mas} se } x - y = 1 \text{ e } x + y = 1 \text{ pode-se assumir } x = 1 \text{ e } y = 0,&\\
					\text{\textbf{assim} } min(x, y) \neq 1,& --- Direto da linha $5$\\
					\text{\textbf{mas} se }  x - y = -1 \text{ e } x + y = -1 \text{ pode-se assumir } x = -1 \text{ e } y = 0, &\\
					\text{\textbf{assim} } min(x, y) \neq 1,& --- Direto da linha $7$\\
					\text{\textbf{mas} } min(x, y) = 1  \text{ e } min(x, y) \neq 1 \text{ é um absurdo}.&--- Direto das linhas $3, 6$ e $8$.
				\end{subproof}
				\text{\textbf{Portanto}, Se } \exists x, y \in \mathbb{Z}_+^* \text{ tal que } x^2 - y^2 = 1, \text{ então }  \bot. & --- Conclusão da PD (linhas $2$-$10$)
			\end{subproof}
			\text{\textbf{Consequentemente}, não } \exists x, y \in \mathbb{Z}_+^* \text{ tal que } x^2 - y^2 = 1. & --- Conclusão da RAA (linhas $2$-$10$)
		\end{logicproof}
	}
\end{example}

\begin{remark}
	Equações diofantinas tem papel central para computação, assim vale mencionar aqui que um importante resultado sobre essas equações que possui forte impacto na lógica, computabilidade e teoria dos números  foi demonstrado pela combinação dos trabalhos de Julia Robinson (1919--1985) e Yuri Matiyasevich(1947--.) \cite{yuri1993hilbert}. Tal resultado é a prova do problema de número dez da famosa lista de Hilbert\footnote{A lista de Hilbert é um lista inicialmente composta por 10 problemas e depois expandida para 23, que foi apresentada pelo matemático alemão David Hilbert (1862-1943) como uma forma de guia a atenção dos matemáticos no século XX.}, de forma sucinta a prova do resultado diz que não existe um algoritmo (ou método) universal para determinar se uma equação diofantina tem raízes inteiras.
\end{remark}

\begin{example}\label{exe:DiagramaProva12}
	Demonstração da asserção: Não existe um programa $P$ jogador de xadrez que sempre vença.
	{\scriptsize
		\begin{logicproof}{3}
			\begin{subproof}
				&  \\
				\begin{subproof}
					\text{\textbf{Assuma por absurdo} que $P$ é um programa jogador de xadrez que sempre vence}, & --- Hipótese\\
					\text{\textbf{logo } pode-se instanciar duas execuções de }  P \text{ denotadas por } P_1 \text{ e } P_2,&\\
					\text{\textbf{mas} se } P_1 \text{ joga contra } P_2 \text{ e vence},&\\
					\text{\textbf{tem-se que} } P \text{ não é um programa que sempre vence},& --- Direto da linha $4$\\
					\text{\textbf{além disso} se ocorre o contrário e } P_2 \text{ joga contra } P_1 \text{ e vence},&\\
					\text{\textbf{tem-se que} } P \text{ não é um programa que sempre vence},& --- Direto da linha $6$\\
					\text{\textbf{mas} } P \text{ perder é um absurdo, já que por hipótese } P \text{ é um programa que sempre vence}.&--- Direto das linhas $3, 5$ e $7$.
				\end{subproof}
				\text{\textbf{Portanto}, Se } P \text{ é um programa de jogar xadrez que sempre vence}, \text{ então }  \bot. & --- Conclusão da PD (linhas $2$-$8$)
			\end{subproof}
			\text{\textbf{Consequentemente}, Não existe um programa $P$ jogador de xadrez que sempre vença.}. & --- Conclusão da RAA (linhas $2$-$9$)
		\end{logicproof}
	}
\end{example}

\begin{note}
	O Exemplo \ref{exe:DiagramaProva12} apresenta a demonstração de uma clássica asserção diretamente ligada a computação, tal exemplo foi apresentado para mostrar ao leitor iniciante que argumentos válidos não necessariamente usam a notação matemática.
\end{note}

\begin{example}\label{exe:DiagramaProva13}
	Demonstração da asserção: Se $3n + 2$ é ímpar, então $n$ é ímpar.
	{\scriptsize
		\begin{logicproof}{3}
			\begin{subproof}
				&  \\
				\begin{subproof}
					\text{\textbf{Suponha por absurdo} que }3n + 2 \text{ é ímpar e } n \text{ é par}, & --- Hipótese\\
					\text{\textbf{logo} } n = 2k, k \in \mathbb{Z},& --- DEF de paridade\\
					\text{\textbf{dessa forma} } 3n + 2 = 2(3k + 1), k \in \mathbb{Z},&\\ 
					\text{\textbf{assim} } 3n + 2 \text{ é par},& --- Da linha $4$ e da DEF de paridade\\
					\text{\textbf{mas} } 3n + 2 \text{ ser ímpar e } 3n + 2 \text{ ser par, é um absurdo}.& --- Das linhas $2$ e $5$
				\end{subproof}
				\text{\textbf{Portanto}, se } 3n + 2 \text{ é ímpar e } n \text{ é par}, \text{ então }  \bot. & --- Conclusão da PD (linhas $2$-$7$)
			\end{subproof}
			\text{\textbf{Consequentemente}, se $3n + 2$ é ímpar, então $n$ é ímpar}. & --- Conclusão da RAA (linhas $2$-$8$)
		\end{logicproof}
	}
\end{example}

\section{Demonstrando  Generalizações}\label{sec:DemonstracaoGeneralizacao}

Antes de falar sobre o método usado para demonstrar generalizações deve-se primeiro reforçar ao leitor o que é são generalizações. Uma generalização é qualquer asserção que contenha em sua formação expressões das formas: 
\begin{itemize}
	\item[(a)] Para todo \underline{\ \ \ \ \ \ \ \ \ \ \ \ \ }.
	\item[(b)] Para cada \underline{\ \ \ \ \ \ \ \ \ \ \ \ \ }.
	\item[(c)] Para qualquer \underline{\ \ \ \ \ \ \ \ \ \ \ \ \ }.
\end{itemize}

\begin{example}
	A seguintes asserções são generalizações.
	\begin{itemize}
		\item[(a)] Todos os cachorros tem quatro patas.
		\item[(b)] Todos os números inteiros possuem um inverso aditivo.
		\item[(c)] Todos os times de futebol pernambucanos são times brasileiros.
	\end{itemize}
\end{example}

Nos termos da lógica uma asserção é uma generalização sempre que o quantificador universal é o quantificador mais externo a da asserção.

\begin{remark}
	Neste texto sempre que possível será usado a escrita da lógica de primeira ordem nas asserções universais e existenciais, para detalhes ver Capítulos \ref{cap:IntroducaoLogica} e \ref{cap:LogicaPredicados}. 
\end{remark}

Agora que o leitor está a par do que é uma generalização, pode-se prosseguir o texto deste manuscrito apresentando formalmente o método de demonstração para generalizações.

\begin{definition}[Prova de Generalizações (PG)]
	Para provar uma asserção da forma, ``$(\forall x)[P(x)]$'', em que $P(x)$ é uma asserção acerca da variável $x$. Deve-se assumir que a variável $x$ assume como valor um objeto qualquer no universo do discurso de que trata a  generalização, em seguida, provar que a asserção $P(x)$ é verdadeira, usando as propriedades disponível de forma genérica para os objetos do universo do discurso.
\end{definition}

Em termos do diagrama, a prova de uma generalização começa inserido na primeira linha de um bloco a informação de que $x$ é um objeto genérico (ou qualquer) do discurso, em seguida deve ser provado $P(x)$ é verdadeiro, caso seja necessário deve ser é aberto um novos blocos para as subprovas, após demonstrar que $P(x)$ é verdadeiro para um $x$ genérico do discurso, o bloco externo (aberto para a prova da generalização) é fechado e  pode-se apresentar a conclusão de que todo objeto $x$ do discurso $P(x)$ é verdadeiro. Note que esse raciocínio de demonstração garante (com explicado em \cite{velleman2019comProvar}) que a asserção $P$ é universal sobre o universo do discurso, ou seja, garante a universalidade da asserção $P$.

\begin{example}\label{exe:DiagramaProva14}
	Demonstração da asserção: $(\forall x \in \{4n \mid n \in \mathbb{N} \})$[$x$ é par].
	{\scriptsize
		\begin{logicproof}{3}
			\begin{subproof}
				\text{\textbf{Assuma} que } x \text{ é um elemento qualquer de } \{4n \mid n \in \mathbb{N} \}&  --- Hipótese\\
				\begin{subproof}
					\text{\textbf{dessa forma} } x = 4n, n \in \mathbb{N}, & --- DEF do discurso\\
					\text{\textbf{logo} } x = (2\cdot 2)n, n \in \mathbb{N},& --- Reescrita da linha $2$\\
					\text{\textbf{dessa forma} } x = 2(2n), n \in \mathbb{N},& --- ASS da multiplicação \\ 
					\text{\textbf{assim} } x = 2k, k \in \mathbb{N},& \\
					\text{\textbf{então} } x \text{ é par}.& --- DEF de paridade
				\end{subproof}
				\text{\textbf{Portanto}, quando } x \in \{4n \mid n \in \mathbb{N} \}, \text{ tem-se então que }  x \text{ é par}. & --- Conclusão  das linhas $2$-$6$
			\end{subproof}
			\text{\textbf{Consequentemente},} (\forall x \in \{4n \mid n \in \mathbb{N} \})\text{[$x$ é par]}. & --- Conclusão da PG (linhas $1$-$7$)
		\end{logicproof}
	}
\end{example}

\begin{example}\label{exe:DiagramaProva15}
	Demonstração da asserção: $(\forall X, Y \subseteq \mathbb{U})$[se $X \neq \emptyset$, então $(X \cup Y) \neq \emptyset$].
	{\scriptsize
		\begin{logicproof}{3}
			\begin{subproof}
				\text{\textbf{Considere} dois conjuntos quaisquer } X, Y \subseteq \mathbb{U}&  --- Hipótese\\
				\begin{subproof}
					\text{\textbf{Suponha} que } X \neq \emptyset, & --- Hipótese\\
					\text{\textbf{logo} existe pelo menos um } x \in X,&\\
					\text{\textbf{desde que} } x \in X \text{ tem-se que } x \in (X \cup Y),&\\ 
					\text{\textbf{então} } (X \cup Y) \neq \emptyset.& 
				\end{subproof}
				\text{\textbf{Portanto}, Se } X \neq \emptyset, \text{ então }  X \cup Y \neq \emptyset. & --- Conclusão  das PD (linhas $2$-$5$)
			\end{subproof}
			\text{\textbf{Consequentemente},} (\forall X, Y \subseteq \mathbb{U})\text{[se $X \neq \emptyset$, então $(X \cup Y) \neq \emptyset$]}. & --- Conclusão da PG (linhas $1$-$6$)
		\end{logicproof}
	}
\end{example}

Um erro que muitos iniciantes frequentemente cometem ao tentar provar enunciados de generalização é utilizar uma (ou mais) propriedade(s) de um elemento genérico $x$  para provar $P(x)$, entretanto esta(s) propriedade(s) usada(s) não é (são) compartilhada(s) por todos os elementos de $\mathbb{U}$, isto é, apenas um subconjunto de $\mathbb{U}$ apresenta a(s) propriedade(s) usadas, para mais detalhes sobre este tipo de erro podem ser consultados em \cite{velleman2019comProvar}.

\begin{example}\label{exe:DiagramaProva16}
	Demonstração da asserção: $(\forall n \in \mathbb{Z})$[se $n > 2$, então $n^2 > n + n$].
	{\scriptsize
		\begin{logicproof}{3}
			\begin{subproof}
				\text{\textbf{Assuma} que } n \text{ é um número inteiro,}&  --- Hipótese\\
				\begin{subproof}
					\text{\textbf{Suponha} que } n > 2, & --- Hipótese\\
					\text{\textbf{logo} } n \cdot n > 2x,& --- Monotonicidade da multiplicação em $\mathbb{Z}$\\
					\text{\textbf{então} } n^2 > n + n.& --- Reescrita da linha $3$
				\end{subproof}
				\text{\textbf{Dessa forma}, se } n > 2, \text{ então } x^2 > n + n. & --- Conclusão  das PD (linhas $2$-$4$)
			\end{subproof}
			\text{\textbf{Portanto},} (\forall n \in \mathbb{Z})\text{[se $n > 2$, então $x^2 > n + n$]}. & --- Conclusão da PG (linhas $1$-$5$)
		\end{logicproof}
	}
\end{example}

\begin{example}\label{exe:DiagramaProva17}
	Demonstração da asserção: $(\forall n \in \mathbb{Z})$[$3(n^2 + 2n + 3) - 2n^2$ é um quadrado perfeito].
	{\scriptsize
		\begin{logicproof}{3}
			\begin{subproof}
				\text{\textbf{Assuma} que } x \text{ é um número inteiro,}&  --- Hipótese\\
				\begin{subproof}
					\text{\textbf{Desde que} } 3(n^2 + 2n + 3) - 2n^2 = 3n^2 + 6n + 9 - 2n^2, &\\
					\text{\textbf{mas} } 3n^2 + 6n + 9 - 2n^2 =  n^2 +6n + 9,&\\
					\text{\textbf{assim} } 3(n^2 + 2n + 3) - 2n^2 = n^2 +6n + 9& --- Direto das linhas $2$ e $3$\\
					\text{\textbf{mas} } n^2 +6n + 9 = (n + 3)^2,&\\
					\text{\textbf{logo} } 3(n^2 + 2n + 3) - 2n^2 = (n + 3)^2,& 
				\end{subproof}
				\text{\textbf{Dessa forma},  } 3(n^2 + 2n + 3) - 2n^2 \text{ é um quadrado perfeito.}&  --- Conclusão  das linhas $2$-$6$
			\end{subproof}
			\text{\textbf{Portanto},} (\forall n \in \mathbb{Z})\text{[$3(n^2 + 2n + 3) - 2n^2$ é um quadrado perfeito]}. & --- Conclusão da PG (linhas $1$-$7$)
		\end{logicproof}
	}
\end{example}

\section{Demonstrando Existência e Unicidade}\label{sec:DemonstrandoExistencia}

Antes de falar sobre o método de demonstração existencial deve-se primeiro reforçar ao leitor o que é um enunciado existencial. Um enunciado de uma sentença do tipo existencial é qualquer asserção que inicia usando as expressões das forma seguir:
\begin{itemize}
	\item[(a)] Existe um(a) $\underline{\ \ \ \ \ \ \ \ \ \ \ \ }$.
	\item[(b)] Há um(a) $\underline{\ \ \ \ \ \ \ \ \ \ \ \ }$.
\end{itemize} 

Agora sobre a metodologia para demonstrar (provar) a existência de um objeto com um determinada propriedade, ou seja, provar que um certo objeto $x$ satisfaz uma propriedade $P$,  é especificada pela definição a seguir.

\begin{definition}[Prova de existência (PE)]
	Para provar uma asserção da forma ``$(\exists x)[P(x)]$'', em que $P(x)$ é uma asserção sobre a variável x. Deve-se exibir um elemento específico ``$a$'' pertencente ao universo do discurso, e mostrar que a asserção $P(x)$ é verdadeira quando $x$ é instanciado como sendo exatamente o elemento $a$, ou seja, deve-se mostrar que $P(a)$ é verdadeira.
\end{definition}

Em relação ao diagrama de bloco, uma demonstração de existência, isto é, uma prova de uma asserção $(\exists x).[P(x)]$,  irá se comportar de forma muito semelhante a uma demonstração de generalidade, as única mudanças significativas é que tal método inicia seu bloco com a declaração de que será atribuído um objeto \textbf{específico} em vez de considerar a variável genérica a $x$, ou seja, é realizado uma instanciação de um elemento. Além disso, a conclusão do bloco externo deve ser exatamente a $(\exists x).[P(x)]$, ou seja, a conclusão deverá ser a asserção existencial.

\begin{example}\label{exe:DiagramaProva18}
	Demonstração da asserção: $(\exists m, n \in \mathbb{I})[m^n \in \mathbb{Q}]$.
	{\scriptsize
		\begin{logicproof}{3}
			\begin{subproof}
				\text{\textbf{Deixe ser} } a = \sqrt{2} \text{ e } b = \sqrt{2}&  --- Instanciação existencial\\
				\text{\textbf{logo} } a,b \in \mathbb{I}, & --- Pelo Exemplo \ref{exe:DiagramaProva10}\\
				\begin{subproof}
					\text{\textbf{Se} } \sqrt{2}^{\sqrt{2}} \in  \mathbb{Q},&\\
					\text{\textbf{então} não há mais nada a ser demonstrado.}&
				\end{subproof}
				\text{\textbf{Consequentemente}, se } \sqrt{2}^{\sqrt{2}} \in  \mathbb{Q}, \text{ então } a^b \in Q.& --- PD das linhas $3$-$4$\\
				\begin{subproof}
					\text{\textbf{Se} } \sqrt{2}^{\sqrt{2}} \notin  \mathbb{Q},&\\
					\text{\textbf{logo} } \sqrt{2}^{\sqrt{2}} \in  \mathbb{I},&\\
					\text{\textbf{assim} fazendo } c = a^b \text{ tem-se que } c \in \mathbb{I},&\\
					\text{\textbf{então} } c^b = (\sqrt{2}^{\sqrt{2}})^{\sqrt{2}} = 2.&
				\end{subproof}
				\text{\textbf{Consequentemente}, se } \sqrt{2}^{\sqrt{2}} \notin \mathbb{Q}, \text{ então } c^b \in \mathbb{Q} \text{ com } c \in \mathbb{I}, c = a^b.& --- PD das linhas $6$-$9$
			\end{subproof}
			\text{\textbf{Portanto},} (\exists m, n \in \mathbb{I})[m^n \in \mathbb{Q}]. & --- Conclusão da PE (linhas $1$-$10$)
		\end{logicproof}
	}
\end{example}

\begin{example}\label{exe:DiagramaProva19}
	Demonstração da asserção: $(\exists n \in \mathbb{N})[n = n^2]$.
	{\scriptsize
		\begin{logicproof}{3}
			\begin{subproof}
				\text{\textbf{Deixe ser} } n = 1 &  --- Instanciação existencial\\
				\text{\textbf{logo} } n \cdot 1 = 1 \cdot 1, & --- Monotonicidade da multiplicação\\
				\text{\textbf{assim} } n = 1^2, & --- Reescrita da linha $2$\\
				\text{\textbf{logo} } 1 = 1^2, & --- Das linhas $1$ e $3$\\
				\text{\textbf{então} } n = n^2,& --- Direto das linha $1, 3$ e $4$
			\end{subproof}
			\text{\textbf{Portanto},} (\exists n \in \mathbb{N})[n = n^2]. & --- Conclusão da PE (linhas $1$-$5$)
		\end{logicproof}
	}
\end{example}

\begin{example}\label{exe:DiagramaProva20}
	Demonstração da asserção: $(\exists X \subseteq \mathbb{U})[(\forall Y \subseteq \mathbb{U})[X \cup Y = Y]]$.
	{\scriptsize
		\begin{logicproof}{5}
			\begin{subproof}
				\text{\textbf{Deixe ser} } X = \emptyset, &  --- Instanciação existencial\\
				\begin{subproof}
					\text{\textbf{Assuma} que } Y \subseteq \mathbb{U}&  --- Hipótese\\
					\text{\textbf{logo} } y \in Y \text{ tem-se que } y \in (X \cup Y)& --- DEF de união\\
					\text{\textbf{assim} } Y \subseteq (X \cup Y), &\\
					\begin{subproof}
						&\\
						\begin{subproof}
							\text{\textbf{Suponha por absurdo} que } (X \cup Y) \not\subseteq Y&  --- Hipótese\\
							\text{\textbf{assim} tem-se que existe } z \in (X \cup Y) \text{ e } z \notin Y, &\\
							\text{\textbf{dessa forma} } z \in X,& --- Direto da linha $7$\\
							\text{\textbf{desde que} } X = \emptyset \text{ é um absurdo que } z \in X,& --- Da linha $1$ e da DEF de conjunto vazio 
						\end{subproof}
						\text{\textbf{Portanto}, se }(X \cup Y) \not\subseteq Y, \text{ então } \bot. & --- Conclusão da PD (linhas $6$-$9$)
					\end{subproof}
					\text{\textbf{Consequentemente}, } (X \cup Y) \subseteq Y.& --- Conclusão RAA (linhas $6$-$10$)
				\end{subproof}
				\text{\textbf{Dessa forma}, } (X \cup Y) = Y.& --- Direto das linhas $4$ e $11$
			\end{subproof}
			\text{\textbf{Portanto},} (\exists X \subseteq \mathbb{U})[(\forall Y \subseteq \mathbb{U})[X \cup Y = Y]]. & --- Conclusão da PE (linhas $1$-$12$)
		\end{logicproof}
	}
\end{example}

\begin{remark}
	O leitor que leu com  atenção o Capítulo \ref{cap:Conjuntos}, ou que tenha domínio sobre a teoria dos conjuntos sabe que $(\emptyset \cup X) = X$, para qualquer conjunto $X$, assim poderia escrever uma prova bem mais curta (fica como exercício) do que a demonstração mostrada no Exercício \ref{exe:DiagramaProva20}.
\end{remark}

Agora vale ressaltar uma importante questão, a prova de existência não garante que um único elemento do discurso satisfaça uma determinada propriedade, note que no Exemplo \ref{exe:DiagramaProva19} poderia ser substituído $1$ pelo número natural $0$ sem haver qualquer perca para a demonstração. De fato, o que a prova de existência garante é que \textbf{pelo menos um} elemento dentro do discurso satisfaz a propriedade que está sendo avaliada. Uma demonstração que garante que \textbf{um e apenas um} elemento em todo discurso satisfaz uma certa propriedade é chamada de demonstração de unicidade.

Antes de falar sobre o método de demonstração de unicidade deve-se primeiro reforçar ao leitor o que é um enunciado existencial de unicidade. Basicamente tal tipod e enunciado consiste de um enunciado de existência que adiciona os termos ``único'' ou ``apenas um'' na uma sentença do tipo existencial ficando da formas:
\begin{itemize}
	\item[(a)] Existe apenas um(a) $\underline{\ \ \ \ \ \ \ \ \ \ \ \ }$.
	\item[(b)] Há apenas um(a) $\underline{\ \ \ \ \ \ \ \ \ \ \ \ }$.
\end{itemize} 
ou ainda,
\begin{itemize}
	\item[(a)] Existe um(a) único(a) $\underline{\ \ \ \ \ \ \ \ \ \ \ \ }$.
	\item[(b)] Há  um(a) único(a) $\underline{\ \ \ \ \ \ \ \ \ \ \ \ }$.
\end{itemize} 

\begin{note}
	Deste ponto em diante sempre que possível será substituido a escrita ``se $P$, então $Q$'' pela notação da lógica simbólica $P \Rightarrow Q$.
\end{note}

\begin{definition}[Prova de unicidade (PU)]\label{def:ProvaUnicidade}
	Uma prova de unicidade consiste em provar uma asserção da forma ``$(\exists x)[P(x) \land (\forall y)[P(y) \Rightarrow x = y]]$'', em que $P$ é uma asserção sobre os elementos do discurso. Para tal primeiro deve-se demonstrar que a asserção ``$(\exists x)[P(x)]$'' é verdadeira, e depois prova que a generalização $(\forall y)[P(y) \Rightarrow x = y]$ também é verdadeira.
\end{definition}

\begin{remark}
	Pela noção de contrapositiva pode-se substituir na Definição \ref{def:ProvaUnicidade} a generalização $(\forall y)[P(y) \Rightarrow x = y]$ pela asserção da forma $(\forall y)[x \neq y \Rightarrow \neg P(y)]$.
\end{remark}

Com respeito ao diagrama de blocos, uma demonstração de unicidade apresenta um diagrama similar ao de uma prova de existência, entretanto, internamente ao bloco da demonstração irá existir uma subprova para a asserção $(\forall y)[P(y) \Rightarrow x = y]$, sendo está subprova responsável pro mostrar a unicidade. Por fim após fechar o bloco mais externo deve-se enunciar a conclusão.

\begin{example}\label{exe:DiagramaProva21}
	Demonstração da asserção: $(\exists! x \in  \mathbb{N})[x + x = x \land (\forall y \in  \mathbb{N})[y + y = y \Rightarrow x = y]]$.
	{\scriptsize
		\begin{logicproof}{3}
				\begin{subproof}
					\text{\textbf{Deixe ser} } x = 0, &  --- Instanciação existencial\\
					\text{\textbf{logo} } x + 0 = 0 + 0,&\\
					\text{\textbf{assim} } x + 0 = 0, &\\
					\text{\textbf{dessa forma} } x + x = x. & --- Da linha $1$ e da reescrita da linha $3$\\
					\begin{subproof}
						\text{\textbf{Suponha } que } y \in \mathbb{N}, &  --- Hipótese\\
						\begin{subproof}
							\text{\textbf{Assuma } que } y + y = y, &  --- Hipótese\\
							\text{\textbf{logo} } y = y - y,&\\
							\text{\textbf{assim} } y = 0,&\\
							\text{\textbf{então} } y = x,& --- Reescrita da linha $8$
						\end{subproof}
						\text{\textbf{Consequentemente},} y + y = y \Rightarrow x = y. & --- Conclusão da PD (linhas $6$-$9$)
					\end{subproof}
					\text{\textbf{Portanto},} (\forall y \in  \mathbb{N})[y + y = y \Rightarrow x = y], & --- Conclusão da PG (linhas $5$-$10$)\\
					\text{\textbf{logo} } x + x = x \land (\forall y \in  \mathbb{N})[y + y = y \Rightarrow x = y].& --- Direto das linhas $4$ e $11$
				\end{subproof}
			\text{\textbf{Portanto},} (\exists x \in  \mathbb{N})[x + x = x \land (\forall y \in  \mathbb{N})[y + y = y \Rightarrow x = y]]. & --- Conclusão da PU (linhas $2$-$13$)
		\end{logicproof}
	}
\end{example}

\begin{remark}
	Obviamente uma aserção de unicidade pode muito bem ser escrita usando a abreviação $(\exists! x)[P(x)]$, para mais detalhes sobre isto veja o Capítulo \ref{cap:IntroducaoLogica} da Parte \ref{Parte:logica} deste manuscrito.
\end{remark}

\begin{example}\label{exe:DiagramaProva22}
	Demonstração da asserção: $(\forall x \in  \mathbb{Z})[(\exists! y \in  \mathbb{Z})[x + y = 0]]$.
	{\scriptsize
		\begin{logicproof}{4}
			\begin{subproof}
				\text{\textbf{Assuma} que } x \in \mathbb{Z}, &  --- Hipótese\\
				\begin{subproof}
					\text{\textbf{Deixe ser} } y = -x, &  --- Instanciação existencial\\
					\text{\textbf{logo} } x + y = x + (-x),&\\
					\text{\textbf{mas} } x + (-x) = 0,&\\
					\text{\textbf{então} } x + y = 0.&
				\end{subproof}
				(\exists y \in  \mathbb{Z})[x + y = 0].& --- Conclusão da PE (linhas $2$-$5$)\\
				\begin{subproof}
					\text{\textbf{Assuma} que } z \in \mathbb{Z}, &  --- Hipótese\\
					\begin{subproof}
						& \\
						\begin{subproof}
							\text{\textbf{Suponha por absurdo} que } x + z = 0 \text{ e } z \neq y, &  --- Hipótese\\
							\text{\textbf{desde que} } x + y = 0 \text{ tem-se que } x + z = x + y,&\\
							\text{\textbf{mas} assim } z = y, \text{ o que contradiz a hipótese e, portanto, é um absurdo}.&
						\end{subproof}
						\text{\textbf{Consequentemente}, se } x + z = 0 \text{ e } z \neq -x, \text{ então } \bot.& --- Conclusão da PD (linhas $8$-$10$)
					\end{subproof}
					\text{\textbf{Portanto}, se } x + z = 0, \text{ então } x = y. & --- Conclusão da RAA (linhas $8$-$11$)
				\end{subproof}
				\text{\textbf{Dessa forma} } (\forall z \in  \mathbb{Z})[x + z = 0 \Rightarrow z = y]. & --- Conclusão da PG (linhas $7$-$13$)
			\end{subproof}
			\text{\textbf{Portanto},} (\forall x \in  \mathbb{Z})[(\exists! y \in  \mathbb{Z})[x + y = 0]]. & --- Conclusão da PU (linhas $2$-$13$)
		\end{logicproof}
	}
\end{example}

\section{Demonstração Guiada por Casos}

Para realizar uma demonstração guiada por casos (ou simplesmente demonstração por casos) a estratégia emprega consiste em demonstrar cobrindo todos os casos possíveis que as premissas  $\alpha_i$ em um enunciado podem assumir, formalmente esta metodologia de demonstração é definida como se segue.

\begin{definition}[Prova por Casos (PPC)]\label{metodo:PorCasos}
	Uma prova por caso, consiste em provar um enunciado da forma: Se $\alpha_1$ ou $\cdots$ ou $\alpha_n$, então $\beta$. Para isso é realizado os seguintes passos:
	\begin{itemize}
		\item Supor $\alpha_1$ (e apenas ela) verdadeira, e demonstrar $\beta$.
		
		$\vdots$
		
		\item Supor $\alpha_n$ (e apenas ela) verdadeira, e demonstrar $\beta$.
	\end{itemize}
\end{definition}

A justificativa da validade  da metodologia da prova por casos é que um enunciado que tenha a forma $(\alpha_1 \lor \cdots \lor \alpha_n) \Rightarrow \beta$ será verdadeiro quando a conjunção da forma $(\alpha_1 \Rightarrow \beta) \land \cdots \land (\alpha_n \Rightarrow \beta)$ for verdadeira,  e para isso deve-se provar a validade de $(\alpha_i \Rightarrow \beta)$ para todo $1 \leq i \leq n$. Dessa forma o leitor pode notar facilmente que uma prova por casos nada mais é do que provar uma série de $n$ implicações (se for necessário releia a Seção \ref{sec:DemonstrandoImplicacoes}).

\begin{remark}
	A justificativa descrita acima pode ser facilmente constatada utilizando o conceito de equivalência lógica, para detalhes veja a Seção \ref{sec:SistemaSemantico} que trata dos sistemas semânticos para da lógica proposicional.
\end{remark}
	
Com respeito ao diagrama de blocos uma prova por casos consiste de um diagrama que possui em seu interior $n$ provas da forma $\alpha_i \Rightarrow \beta$ com $1 \leq i \leq n$, após todas as sub-provas serem apresentadas a última linha no diagrama mais externo irá expressar uma sentença da forma $(\alpha_1 \Rightarrow \beta) \land \cdots \land (\alpha_n \Rightarrow \beta)$, então o diagrama será fechado e será escrita a conclusão do diagrama. O exemplo a seguir ilustram esse procedimento.
	
\begin{example}\label{exe:DiagramaProva23}
	Demonstração da asserção: Se $x \in  \mathbb{Z}$, então $x^2$ tem a mesma paridade de $x$.
	{\scriptsize
		\begin{logicproof}{4}
			\begin{subproof}
				&\\
				\begin{subproof}
					\text{\textbf{Assuma} que } x = 2i \text{ com } i \in \mathbb{Z}, &  --- Hipótese\\
					\text{\textbf{logo} } x^2 = 2(2i^2), \text{ com } i \in \mathbb{Z} &\\
					\text{\textbf{então} } x \text{ é par} & --- DEF de paridade
				\end{subproof}
				\text{\textbf{Portanto}, se } x \text{ é par, então }, x^2 \text{ é par} & --- Conclusão da PD (linhas 2-4)\\
				\begin{subproof}
					\text{\textbf{Assuma} que } x = 2i + 1\text{ com } i \in \mathbb{Z}, &  --- Hipótese\\
					\text{\textbf{logo} } x^2 = 2(2i^2) + 1, \text{ com } i \in \mathbb{Z} & \\
					\text{\textbf{assim} } x^2 = 2j + 1, \text{ com } j = (2i^2), i \in \mathbb{Z} & --- Reescrita\\
					\text{\textbf{então} } x \text{ é par} & --- DEF de paridade
				\end{subproof}
				\text{\textbf{Portanto}, se } x \text{ é ímpar, então }, x^2 \text{ é par} & --- Conclusão da PD (linhas 6-9)
			\end{subproof}
			\text{\textbf{Consequentemente}, se } x \in \mathbb{Z}, \text{então } x^2 \text{ tem a mesma paridade que } x. & --- Conclusão da PPC (linhas 1-10)
		\end{logicproof}
	}
\end{example}

\begin{remark}
	Note que os casos que guiam a prova do Exemplo \ref{exe:DiagramaProva23} são o caso do $x$ ser par e o caso do $x$ ser ímpar.
\end{remark}

\begin{example}\label{exe:DiagramaProva24}
	Demonstração da asserção: Dado $n \in \mathbb{N}$. Se $n \leq 2$, então $n! \leq n + 1$.
	{\scriptsize
		\begin{logicproof}{4}
			\begin{subproof}
				n \in \mathbb{N} & --- Premissa\\
				\begin{subproof}
					\text{\textbf{Assuma} que } n = 0, &  --- Hipótese\\
					\text{\textbf{desde que} } 0! = 1 &\\
					\text{\textbf{assim} } 0! \leq 1 &\\
					\text{\textbf{mas} } 1 = n + 1&\\
					\text{\textbf{então} } n! \leq n + 1 &
				\end{subproof}
				\text{\textbf{Portanto}, se } n = 0, \text{então } n! \leq n + 1 & --- Conclusão da PD (linhas 2-5)\\
				\begin{subproof}
					\text{\textbf{Assuma} que } n = 1, &  --- Hipótese\\
					\text{\textbf{desde que} } n! = 1 &\\
					\text{\textbf{assim} } n! < 2 &\\
					\text{\textbf{mas} } 2 = n + 1&\\
					\text{\textbf{então} } n! < n + 1 &
				\end{subproof}
				\text{\textbf{Portanto}, se } n = 1, \text{então } n! < n + 1 & --- Conclusão da PD (linhas 7-11)\\
				\begin{subproof}
					\text{\textbf{Assuma} que } n = 2, &  --- Hipótese\\
					\text{\textbf{desde que} } n! = 2 &\\
					\text{\textbf{logo} } n! < 3 &\\
					\text{\textbf{mas} } 3 = n + 1&\\
					\text{\textbf{então} } n! < n + 1 &
				\end{subproof}
				\text{\textbf{Portanto}, se } n = 2, \text{então } n! < n + 1 & --- Conclusão da PD (linhas 7-11)
			\end{subproof}
			\text{\textbf{Consequentemente}, Dado $n \in \mathbb{N}$. Se } n \leq 2, \text{ então }n! \leq n + 1. & --- Conclusão da PPC (linhas 1-19)
		\end{logicproof}
	}
\end{example}

\section{Outras Formas de Representação de Provas}\label{sec:OutrasFormasProvas}

Durante este capítulo foram apresentadas diversas metodologias para se realizar demonstrações, e para representar as provas (demonstrações) usando tais metodologias foi empregado o uso de representação por diagrama de blocos. Este manuscrito utilizou-se dessa representação por ela ser mais amigável ao leitor iniciante na tarefa de provar teoremas.

Existem diversas outras formas de representar a demonstração de um teorema, por exemplo, o professor Thanos Tsouanas em seu livro \cite{fmcbook}, usa o conceito de tabuleiro do ``jogo''da demonstração para representar as demonstrações. Por fim, vale destacar a representação das demonstrações por meio de texto formal, que consiste basicamente em descrever a prova usando um texto utilizando o máximo de formalismo matemático possível, o exemplo a seguir ilustra a representação em texto formal.

\begin{example}
	A representação por texto formal da demonstração da asserção: ``Se $n$ é par, então $n^2$ é par'', pode ser da seguinte forma.
	\begin{proof}
		Suponha que $n$ é par, logo $n = 2k$ para algum $k \in \mathbb{Z}$, dessa forma tem-se que $n^2 = n \cdot n = 2k \cdot 2k = 4k^2 = 2(2k^2)$, mas desde que a multiplicação e potenciação são fechadas em $\mathbb{Z}$ tem-se que existe $r \in \mathbb{Z}$ tal que $r = 2k^2$ e, portanto, $n^2 = 2r$, consequentemente,  $n^2$ é par.
	\end{proof}
\end{example}

A representação por texto formal é em geral a maneira utilizada de fato no meio acadêmico, para mais exemplos dessa representação veja \cite{valdi2016master, valdi2020phd, annax2019phd, thadeu2021phd, rui2019phd} e com texto em inglês é sugerido a leitura de \cite{vania2019phd}.

\begin{remark}
	A partir deste ponto do manuscrito será adotado a escrita de demonstração em texto formal, ficando assim a representação por bloco ``confinada'' neste capítulo.
\end{remark}

\section{Refutação por Contraexemplo}

Escrever futuramente sobre contraexemplos\footnote{A grafia contra-exemplos era usando no AO de 1945 e não é mais usada!}...!

\section{Questionário}\label{sec:Questionario2part1}

\begin{problem}\label{prob:Demosntracoes1}
	Demonstre as seguintes asserções.
\end{problem}

\begin{exerList}
	\item Dado $a, b, \in \mathbb{R}$. Se $a < b < 0$, então $a^2 > b^2$.
	\item Dado $a, b, \in \mathbb{R}$. Se $0 < a < b$, então $\frac{1}{b} < \frac{1}{a}$.
	\item Dado $a, \in \mathbb{R}$. Se $a^3 > a$, então $a^5 > a$.
	\item Sejam $(A - B) \subseteq (C \cap D)$ e $x \in A$. Se $x \notin D$, então $x \in B$.
	\item Sejam $a, b \in \mathbb{R}$. Se $a < b$, então $\frac{a + b}{2} < b$.
	\item Dado $x \in \mathbb{R}$ e $x \neq 0$. Se $\frac{\sqrt[3]{x} + 5}{x^2 + 6} = \frac{1}{x}$, então $x \neq 8$.
	\item Sendo $a, b, c, d \in \mathbb{R}$ com $0 < a < b$ e $d > 0$. Se $ac \geq bd$, então $c > d$.
	\item Dado $x, y \in \mathbb{R}$ e $3x + 2y \leq 5$. Se $x > 1$, então $y < 1$.
	\item Sejam $x, y \in \mathbb{R}$. Se $x^2 + y = -3$ e $2x - y = 2$, então $x = -1$.
	\item Se $n \in \mathbb{Z}$ e $4 \leq n \leq 12$, então $n$ é a soma de dois números primos.
	\item Dado $n \in \mathbb{N}$. Se $n \leq 3$, então $n! \leq 2^n$.
	\item Dado $n \in \mathbb{N}$. Se $2 \leq n \leq 4$, então $n^2 \geq 2^n$. 
	\item Se $n$ é um inteiro par, então $n^2 - 1$ é ímpar.
	\item Seja $n_0 \in \mathbb{N}$ e $n_1 = n_0 + 1$. Tem-se que $n_0n_1$ é par.
	\item Se $n \in \mathbb{Z}$, então $n^2 + n$ é par.
	\item Se $n \in \mathbb{Z}$ e $n$ é par, então $n^2$ é divisível por $4$. 
	\item Para todo $n \in \mathbb{Z}$ o número $3(n^2 + 2n + 3) - 2n^2$ é um quadrado perfeito.
	\item Dado $n \in \mathbb{Z}$. Se $x > 0$, então $x + 1 > 0$.
	\item Se $n$ é ímpar, então $n$ é a diferença de dois quadrados.
	\item Se $3n + 5 = 6k + 8$ com $k \in \mathbb{Z}$, então $n$ é ímpar.
	\item Se $n$ é par, então $3n + 2 = 6k + 2$ com $k \in \mathbb{Z}$.
	\item Se $x^2 + 2x - 3 = 0$, então $x \neq 2$.
	\item Dado $n, n_0, n_1 \in \mathbb{Z}$. Se $n_0$ e $n_1$ são ambos divisíveis por $n$, então $n_0 + n_1$ é também divisível por $n$.
	\item Dado $x, y \in \mathbb{Z}$. Se $xy$ não é divisível por $n$ tal que $n \in \mathbb{Z}$, então $x + y$ é divisível por $n$.
	\item Dado $m, n, p \in \mathbb{Z}$. Se $m$ é divisível por $n$ e $n$ é divisível por $p$, então $m$ é divisível por $p$.
	\item Se $x$ é ímpar, então $x^2 - x$ é par.
\end{exerList}

\begin{problem}\label{prob:Demosntracoes2}
	Prove que se $A$ e $(B - C)$ são disjuntos, então $(A \cap B) \subseteq C$.
\end{problem}

\begin{problem}\label{prob:Demosntracoes3}
	Prove que se $A \subseteq (B - C)$, então $A$ e $C$ são disjuntos.
\end{problem}

\begin{problem}\label{prob:Demosntracoes4}
	Dado $x \in \mathbb{R}$ prove que:
\end{problem}

\begin{exerList}
	\item Se $x \neq 1$, então existe $y \in \mathbb{R}$ tal que $\frac{y+1}{y-2} = x$.
	\item Se existe um $y \in \mathbb{R}$ tal que  $\frac{y+1}{y-2} = x$, então $x \neq 1$.
\end{exerList}

\begin{problem}\label{prob:Demosntracoes5}
	Dado um conjunto $A$ e $\mathcal{G}$ uma família demonstre que:
\end{problem}

\begin{exerList}
	\item Se $A \in \mathcal{G}$, então $A \subseteq \mathcal{G}_{\cup}$.
	\item Se $A \in \mathcal{G}$, então $\mathcal{G}_{\cap} \subseteq A$.
\end{exerList}

\begin{problem}\label{prob:Demosntracoes6}
	Se $B$ é um conjunto, $\mathcal{G}$ uma família não vazia e $(\forall A \in \mathcal{G})[B \subseteq A]$, então $B \subseteq \mathcal{G}_{\cap}$.
\end{problem}