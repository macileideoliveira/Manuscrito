\chapter{Introdução}\label{cap:LinguagemFormais}

%\epigraph{``O gênio vê a resposta antes da pergunta.''}{Robert Oppenheiner.}

\epigraph{``Nós só podemos ver um pouco do futuro, mas o suficiente para perceber que há muito a fazer.''}{Alan M. Turing}

\section{Sobre as Linguagens Formais}\label{sec:Linguagens}

\section{Noções Fundamentais}\label{sec:Nocoesbasicas}

Neste primeiro momento para o estudo da teoria das linguagens formais e dos autômatos finitos serão apresentados alguns conceitos fundamentais de extrema importância para o desenvolvimento das próximas seções e capítulos.

\begin{definition}[Alfabetos e Palavras]\label{def:AlfabetoPalavra}
	Qualquer conjunto finito e não vazio $\Sigma$ será chamado de alfabeto. Qualquer sequência finita de símbolos na forma $a_1\cdots a_n$ com $a_i \in \Sigma$ para todo $1 \leq i \leq n$ será chamada de palavra sobre o alfabeto $\Sigma$.
\end{definition}

\begin{example}
	Os conjuntos $\{0, 1, 2, 3\}, \{a, b, c\}, \{\heartsuit,\spadesuit, \Diamond, \clubsuit\}$ e $\{n \in \mathbb{N} \mid n \leq 25\}$ são todos alfabetos, já os conjuntos $\mathbb{N}$ e $\mathbb{R}$ não são alfabetos.
\end{example}

\begin{example}
	Dado o alfabeto $\Sigma = \{0, 1, 2, 3\}$ tem-se que as sequências 0123, 102345, 1 e 0000 são todas palavras sobre $\Sigma$.
\end{example}

\begin{definition}[Comprimento das palavras]\label{def:ComprimentoPalavra}
	Seja $w$ uma palavra qualquer sobre um certo alfabeto $\Sigma$, o comprimento\footnote{Por conta desta notação em alguns texto é usado o termo módulo em vez de comprimento.} de $w$, denotado por $|w|$, corresponde ao número de símbolos existentes em $w$.
\end{definition}

\begin{example}
	Dado o alfabeto $\Sigma = \{a, b, c, d\}$ e as palavras $abcd, aacbd, c$ e $ddaacc$ tem-se que: $|abcd| = 4, |aa| = 2, |c| = 1$ e $|ddaacc| = 6$.
\end{example}

\begin{remark}
	Em especial quando $|w| = 1$, é dito que $w$ é uma palavra unitária, isto é, a mesma contém apenas um único símbolo do alfabeto.
\end{remark}

Como muito bem explicado em \cite{benjaLivro2010, hopcroft2008, linz2006}, pode-se definir uma série de operações sobre palavras, sendo a primeira delas  a noção de concatenação.

\begin{definition}[Concatenação de palavras]\label{def:Concatenacao}
	Sejam $w_1 = a_1\cdots a_m$ e $w_2 = b_1\cdots b_n$ duas palavras quaisquer, tem-se que a concatenação de $w_1$ e $w_2$, denotado por $w_1w_2$, corresponde a uma sequência iniciada com os símbolos que forma $w_1$ imediatamente seguido dos símbolos que forma $w_2$, ou seja, $w_1w_2 = a_1\cdots a_mb_1\cdots b_n$.
\end{definition}

\begin{remark}
	O leitor deve ficar atento ao fato de que a concatenação apenas combina duas palavras em uma nova palavra, sendo que, não a qualquer tipo de exigência sobre os alfabeto sobre os quais as palavras usadas na concatenação estão definidas.
\end{remark}

\begin{example}\label{exe:Concatenacao}
	Dado duas palavras $w_1 = abra$ e $w_2 = cadabra$ tem-se que $w_1w_2 = abracadabra$ e $w_2w_1 = cadabraabra$.
\end{example}

Note que o Exemplo \ref{exe:Concatenacao} estabelece que a operação de concatenação entre duas palavras não é comutativa, isto é, a ordem com que as palavras aparecem é responsável pela forma da palavra resultante.

\begin{theorem}[Associativade da Concatenação]\label{teo:AssociatividaeConcatenacao}
	Para quaisquer $w_1, w_2$ e $w_3$ tem-se que $(w_1w_2)w_3 = w_1(w_2w_3)$.
\end{theorem}

\begin{proof}
	Dado três palavras quaisquer $w_1 = a_1\cdots a_i, w_2 = b_1\cdots b_j$ e $w_3 = c_1\cdots c_k$ tem-se que,
	\begin{eqnarray*}
		(w_1w_2)w_3 & = & (a_1\cdots a_ib_1\cdots b_j)c_1\cdots c_k\\
		& = & a_1\cdots a_ib_1\cdots b_jc_1\cdots c_k\\
		& = & a_1\cdots a_i(b_1\cdots b_jc_1\cdots c_k)\\
		& = & w_1(w_2w_3)
	\end{eqnarray*}
	o que conclui a prova.
\end{proof}

Sobre qualquer alfabeto $\Sigma$ sempre é definida uma palavra especial chamada \textbf{palavra vazia} \cite{hopcroft2008, linz2006}, esta palavra especial não possui nenhum símbolo, e em geral é usado o símbolo $\lambda$ para denotar a palavra vazia \cite{benjaLivro2010, valdi2016master}. Como mencionado em \cite{benjaLivro2010, valdi2020phd} sobre a palavra vazia é importante destacar que:

\begin{eqnarray}
	w\lambda & = & \lambda w = w\\
	|\lambda| & = &  0
\end{eqnarray}

Isto é, a palavra vazia é neutra para a operação de concatenação, além disso, a mesma apresenta comprimento nulo.

\begin{definition}[Potência das palavras]\label{def:PotenciaPalavras}
	Seja $w$ uma palavra sobre um alfabeto $\Sigma$ a potência de $w$ é definida recursivamente para todo $n \in \mathbb{N}$ como sendo:
	\begin{eqnarray}
		w^0 & = & \lambda\\
		w^{n+1} & = & ww^{n}
	\end{eqnarray}
\end{definition}

\begin{example}
	Sejam $w_1 = ab, w_2 = bac$ e $w_3 = cbb$ palavras sobre $\Sigma = \{a, b, c\}$ tem-se que:
	\begin{itemize}
		\item[(a)] $w_1^3 = w_1w_1^2 = w_1w_1w_1^1 = w_1w_1w_1w_1^0 = w_1w_1w_1\lambda = ababab$.
		\item[(b)] $w_2^2 = w_2w_2^1 = w_2w_2w_2^0 = w_2w_2\lambda = w_2w_2 = bacbac$.
	\end{itemize} 
\end{example}

\begin{example}
	Seja $u = 01$ e $v = 231$ tem-se que: 
	$$uv^3 = uvv^2 = uvvv^1 = uvvv\lambda = uvvv = 01231231231$$
	e também 
	$$u^2v = uu^1v = uu\lambda v = uuv = 0101231$$
\end{example}

\begin{proposition}
	Para toda palavra $w$ e todo $m,n \in \mathbb{N}$ tem-se que:
	\begin{itemize}
		\item[(i)] $(w^m)^n = w^{mn}$.
		\item[(ii)] $w^mw^n = w^{m+n}$.
	\end{itemize}
\end{proposition}

\begin{proof}
	Direto das Definições \ref{def:Concatenacao} e \ref{def:PotenciaPalavras}, e portanto, ficará como exercício ao leitor.
\end{proof}

Outro importante conceito existente sobre a ideia de palavra é a noção de palavra reversa (ou inversa) formalmente definida como se segue.

\begin{definition}[Palavra Reversa]\label{def:PalavraInversa}
	\cite{valdi2016master} Seja $w = a_1\cdots a_n$ uma palavra qualquer, a palavra reversa de $w$ denotada por $w^r$, é tal que $w^r = a_n\cdots a_1$. 
\end{definition}

\begin{example}
	Dado as palavras $u = aba, v = 011101$ e $w = 3021$ tem-se que $u^r = aba, v^r = 101110$ e $w^r = 1203$.
\end{example}

\begin{remark}
	Com respeito a noção de palavra reversa tem-se em particular que vale a seguinte igualdade $\lambda^r = \lambda$.
\end{remark}

Além das palavras, pode-se também formalizar uma série de operações sobre a própria noção de alfabeto. Em primeiro lugar, uma vez que,  alfabetos são conjuntos, obviamente todas operações usuais de união, interseção, complemento, diferença e diferença simétrica (ver Capítulo \ref{cap:Conjuntos}) também são válidas sobre alfabetos. Além dessas operações, também esta definida a operação de potência e os fechos positivo e de Kleene sobre alfabetos.

\begin{definition}[Potência de um alfabeto]\label{def:PotenciaAlfabeto}
	\cite{benjaLivro2010} Seja $\Sigma$ um alfabeto a potência de $\Sigma$ é definida recursivamente para todo $n \in \mathbb{N}$ como:
	\begin{eqnarray}
		\Sigma^0 & = & \{\lambda\}\\
		\Sigma^{n+1} & = & \{aw \mid a \in \Sigma, w \in \Sigma^{n}\}
	\end{eqnarray}
\end{definition} 

\begin{example}
	Dado $\Sigma = \{a, b\}$ tem-se que $\Sigma^3 = \{aaa, aab, aba, baa, abb, bab, bba, bbb\}$ e $\Sigma^1 = \{a, b\}$
\end{example}

\begin{example}
	Seja $\Sigma = \{0, 1, 2\}$ tem-se que $\Sigma^2 = \{00, 01, 02, 10, 11, 12, 20, 21, 22\}$ e $\Sigma^{0} = \{\lambda\}$.
\end{example}

O leitor mais atencioso e maduro matematicamente pode notar que para qualquer que seja $n \in \mathbb{N}$ o conjunto potência tem a propriedade de que todo $w \in \Sigma^n$ é tal que $|w| = n$, além disso, é claro que todo $\Sigma^n$ é sempre finito\footnote{Essa afirmação é facilmente verificável, uma vez que, a mesma nada mais é do que um exemplo de arranjo com repetição.}.

\begin{definition}[Fecho Positivo e de Kleene]\label{def:FechoPositivoKleene}
	Seja $\Sigma$ um alfabeto o fecho positivo e o fecho de Kleene de $\Sigma$, denotados respectivamente por $\Sigma^+$ e $\Sigma^*$, correspondem aos conjuntos:
	\begin{eqnarray}
		\Sigma^+ & = & \bigcup_{i = 1}^\infty \Sigma^i
	\end{eqnarray}
	e
	\begin{eqnarray}
		\Sigma^* & = & \bigcup_{i = 0}^\infty \Sigma^i
	\end{eqnarray}
\end{definition}

Obviamente como dito em \cite{benjaLivro2010}, o fecho de positivo pode ser reescrito em função do fecho de Kleene usando a operação de diferença de conjunto, isto é, o fecho positivo corresponde a seguinte identidade, $\Sigma^+ = \Sigma^* - \{\lambda\}$. Sobre o fecho de Kleene com destacado em \cite{valdi2020phd} o mesmo corresponde ao monoide livremente\footnote{Relembre que uma álgebra é livremente gerada quando todo elemento possui fatoração única (a menos de isomorfismo).} gerado pelo conjunto $\Sigma$ munida da operação de concatenação.

\begin{definition}[Prefixos e Sufixos]\label{def:PrefixoSufixo}
	Uma palavra $u \in \Sigma^*$ é um prefixo de outra palavra $w \in \Sigma^*$, denotado por $u \preceq_p w$, sempre que $w = uv$, com $v \in \Sigma^*$. Por outro lado, uma palavra $u$ é um sufixo de outra palavra $w$, denotado por $u \preceq_s w$, sempre que $w = vu$.
\end{definition}

\begin{example}
	Seja $w = abracadabra$ tem-se qu~e as palavras $ab$ e $abrac$ são prefixos de $w$, por outro, lado $cadabra$ e $bra$ são sufixos de $w$, e a palavra $abra$ é prefixo e também sufixo. Já a palavra $cada$ não é prefixo e nem sufixo de $w$.
\end{example}

\begin{definition}[Conjunto dos Prefixos e Sufixos]\label{def:ConjuntoPrefixoSufixo}
	Seja $w \in \Sigma^*$ o conjunto de todos os prefixos de $w$ corresponde ao conjunto:
	\begin{eqnarray}
		PRE(w) = \{w' \in \Sigma^* \mid w' \preceq_p w\}
	\end{eqnarray}
	e o conjunto de todos os sufixos de $w$ corresponde ao conjunto:
	\begin{eqnarray}
		SUF(w) = \{w' \in \Sigma^* \mid w' \preceq_s w\}
	\end{eqnarray}
\end{definition}

\begin{example}
	Seja $w = univasf$ tem-se que:
	\begin{eqnarray*}
		PRE(w) = \{\lambda, u, un, uni, univ, univa, univas, univasf\}
	\end{eqnarray*}
	e
	\begin{eqnarray*}
		SUF(w) = \{\lambda, f, sf, asf, vasf, ivasf, nivasf, univasf \}
	\end{eqnarray*}
\end{example}

\begin{example}
	A seguir é apresentado alguns exemplos de palavras e seus conjuntos de prefixos e sufixos.
	\begin{itemize}
		\item[(a)] Se $w = ab$, então $PRE(w) = \{\lambda, a, ab\}$ e  $SUF(w) = \{\lambda, b, ab\}$.
		\item[(b)] Se $w = 001$, então $PRE(w) = \{\lambda, 0, 00, 001\}$ e  $SUF(w) = \{\lambda, 1, 01, 001\}$.
		\item[(c)] Se $w = \lambda$, então $PRE(w) = \{\lambda\}$ e  $SUF(w) = \{\lambda\}$
		\item[(d)] Se $w = a$, então $PRE(w) = \{\lambda, a\}$ e $SUF(w) = \{\lambda, a\}$.
	\end{itemize}
\end{example}

Com respeito a cardinalidade dos conjuntos de prefixos e sufixos, os mesmo apresentam as propriedades descritas pelo teorema a seguir.

\begin{theorem}\label{teo:CardinalidadePrefixoSufixo}
	Para qualquer que seja $w \in \Sigma^*$ as seguintes asserções são verdadeiras.
	\begin{itemize}
		\item[(i)] $\# PRE(w) = |w| + 1$.
		\item[(ii)] $\#PRE(w) = \#SUF(w)$.
		\item[(iii)] $\#(PRE(w) \cap SUF(w)) \geq 1$.
	\end{itemize}
\end{theorem}

\begin{proof}
	Dado uma palavra $w$ tem-se que:
	\begin{itemize}
		\item[(i)] Sem perda de generalidade assumindo que $w = a_1\cdots a_n$ logo $w \in \Sigma^n$ (o caso quando $w = \lambda$ é trivial e não será demonstrado aqui) logo $|w| = n$ para algum $n \in \mathbb{N}$, assim existem exatamente $n$ palavras da forma $a_1 \cdots a_i$ com $1 \leq i \leq n$ tal que $a_1 \cdots a_i \preceq_p w$, portanto, para todo $1 \leq i \leq n$ tem-se que $a_1 \cdots a_i \in PRE(w)$, além disso, é claro que $w = \lambda w$, e portanto, $\lambda \in PRE(w)$, consequentemente, $\#PRE(w) = n + 1 = |w| + 1$.
		\item[(ii)] É suficiente mostrar que $\# SUF(w) = |w| + 1$, para isso como antes sem perda de generalidade assuma que $w = a_1\cdots a_n$ e assim tem-se que $w \in \Sigma^n$ logo $|w| = n$ com $n \in \mathbb{N}$, dessa forma existem exatamente $n$ palavras da forma $a_i \cdots a_n$ com $1 \leq i \leq n$ tal que $a_i \cdots a_n \preceq_s w$, portanto, para todo $1 \leq i \leq n$ tem-se que $a_i \cdots a_n \in SUF(w)$, além disso, é claro que $w = w\lambda$, logo $\lambda \in SUF(w)$, consequentemente, $\#SUF(w) = n + 1 = |w| + 1$, e portanto, $\#PRE(w) = \#SUF(w)$. O caso $w = \lambda$ é trivial e não será demonstrado aqui.
		\item[(iii)] Trivial, pois $\lambda \in (PRE(w) \cap SUF(w))$, e portanto, tem-se que $\#(PRE(w) \cap SUF(w)) \geq 1$.
	\end{itemize}
\end{proof}

\begin{corollary}
	Toda palavra tem pelo menos um prefixo e um sufixo.
\end{corollary}

\begin{proof}
	Direto do item $(iii)$ do Teorema \ref{teo:CardinalidadePrefixoSufixo}.
\end{proof}

Seguindo com o texto deste manuscrito pode-se finalmente formalizar o pilar fundamental (a ideia de linguagem) necessário para desenvolver o estuda da computabilidade neste e nos próximo capítulos.

\begin{definition}[Linguagem]\label{def:Linguagem}
	Dado um alfabeto $\Sigma$, qualquer subconjunto $L \subseteq \Sigma^*$ será chamado de linguagem.
\end{definition}

\begin{example}
	Seja $\Sigma = \{0, 1\}$ tem-se que os conjuntos a seguir são todos linguagens sobre $\Sigma$.
	\begin{itemize}
		\item[(a)] $\Sigma^*$.
		\item[(b)] $\{0^nb^n \mid n \in \mathbb{N}\}$.
		\item[(c)] $\{\lambda, 0, 1\}$.
		\item[(d)] $\Sigma^{22}$.
		\item[(e)] $\emptyset$.
	\end{itemize}
\end{example}

Similarmente ao que ocorre com os alfabetos, as linguagens por serem conjuntos ``herdam'' as operações básicas da teoria dos conjuntos \cite{lipschutz1978-TC, lipschutz2013-MD, abe1991-TC}, isto é, estão definidas sobre as linguagens as operações de união, interseção, completo, diferença e diferença simétrica. E como par aos alfabetos novas operações são definidas.

\begin{definition}[Concatenação de Linguagens]\label{def:ConcatenacaoLinguagem}
	Sejam $L_1$ e $L_2$ duas linguagens, a concatenação de $L_1$ com $L_2$, denotado por $L_1L_2$, corresponde a seguinte linguagem:
	\begin{eqnarray}
		L_1L_2 = \{xy \in (\Sigma_1 \cup \Sigma_2)^* \mid x \in L_1, y \in L_2\}
	\end{eqnarray}
\end{definition}

\begin{example}\label{exe:ConcatenacaoLinguagem} 
	Dado as três linguagens $L_1 = \{\lambda, ab, bba\}, L_2 =\{0^{2n}1 \mid n \in \mathbb{N}\}$ e $L_3 = \{a^p \mid p \text{ é primo}\}$ tem-se que:
	\begin{itemize}
		\item[(a)] $L_1L_2 = \{w \mid w = 0^{2n}1 \text{ ou } w = ab0^{2n}1 \text{ ou } w = bba0^{2n}1 \text{ com } n \in \mathbb{N}\}$.
		\item[(b)] $L_3L1 = \{w \mid w = a^p \text{ ou } w = a^{p+1}b \text{ ou } a^pbba \text{ onde } p \text{ é um número primo}\}$.
		\item[(c)] $L_2L_3 = \{0^{2n}1a^p \mid n \in \mathbb{N}, p \text{ é um número primo}\}$.
	\end{itemize}
\end{example}

\begin{definition}[Linguagem Reversa]\label{def:LinguagemReversa}
	Seja $L$ uma linguagem, a linguagem reversa de $L$, denotada por $L^r$, corresponde ao conjunto $\{w^r \mid w \in L\}$.
\end{definition}

\begin{example}
	Considerando as linguagens $L_1, L_2$ e $L_3$ do Exemplo \ref{exe:ConcatenacaoLinguagem} e a linguagem $\{01, 011\}$ tem-se que:
	\begin{itemize}
		\item[(a)] $L_1^r = \{\lambda, ba, abb\}$.
		\item[(b)] $L_2^r = \{10^{2n} \mid n \in \mathbb{N}\}$.
		\item[(c)] $L_3^r = \{a^p \mid n \in \mathbb{N}, p \text{ é um número primo}\}$.
		\item[(d)] $\{01, 011\}^r = \{10, 110\}$.
	\end{itemize}
\end{example}

\begin{remark}
	O leitor mais atento pode perceber que a propriedade involutiva da operação reversa sobre palavras é ``herdada'' para a reversão sobre linguagens, isto é, para qualquer linguagem $L$ tem-se que $(L^r)^r = L$. 
\end{remark}

\begin{definition}[Linguagem Potência]\label{def:PotenciaLinguagens}
	Seja $L$ uma linguagem, a linguagem potência de $L$, denotada por $L^n$, é definida recursivamente para todo $n \in \mathbb{N}$ como:
	\begin{eqnarray}
		L^0 & = &\{\lambda\}\\
		L^{n+1} & = &  LL^{n}
	\end{eqnarray}
\end{definition}

Utilizando o conceito de linguagem potência a seguir é apresentado a formalização para os fechos positivo e de Kleene sobre linguagens.

\begin{definition}[Fecho positivo e Fecho de Kleene de Linguagens]\label{def:FechoPositivoKleeneLinguagem}
	Seja $L$ uma linguagem, o fecho positivo $(L^+)$ e o fecho de Kleene $(L^*)$ de $L$ são dados pelas equações a seguir.
	\begin{eqnarray}
		L^+ & = & \bigcup_{i = 1}^\infty L^i\\
		L^* & = & \bigcup_{i = 0}^\infty L^i
	\end{eqnarray}
\end{definition}

Por fim, esta seção irá apresentar a noção de linguagem dos prefixos e sufixos.

\begin{definition}[Linguagem de Prefixos e Sufixos]\label{def:LinguagemPrefixosSufixos}
	Seja $L$ uma linguagem, a linguagem dos prefixos e dos sufixos de $L$, respectivamente $PRE(L)$ e $SUF(L)$, são exatamente os seguintes conjuntos:
	\begin{eqnarray*}
		PRE(L) & = & \{w' \in \Sigma^* \mid w' \preceq_p w, w \in L\}\\
		SUF(L) & = & \{w' \in \Sigma^* \mid w' \preceq_s w, w \in L\}
	\end{eqnarray*}
\end{definition}

\begin{example}
	Considere a linguagem $\{0, 10, 11010\}$ tem-se que:
	\begin{eqnarray*}
		PRE(\{0, 10, 11010\}) & = & \{\lambda, 0, 1, 10, 11, 110, 1101, 11010\}\\
		SUF(\{0, 10, 11010\}) & = & \{\lambda, 0, 10, 010, 1010, 11010\}
	\end{eqnarray*}
\end{example}

Nos próximos capítulos deste manuscrito irão ser apresentadas as formalizações da ideia de linguagens formais na visão ``mecânica'' de Turing \cite{turing1937}, entretanto, em vez de apresentar de forma direta os conceitos ligados as máquinas de Turing e as computações por elas realizadas, este manuscrito opta por fazer um estudo seguindo a ideia dos livros texto de linguagens formais \cite{benjaLivro2010, linz2006, menezes1998LFA}, assim sendo, este parte do manuscrito irá apresentar as linguagens formais da mais simples para a mais complexas seguindo a hierarquia de Chomsky \cite{chomsky1956}, ou seja, serão aqui estudadas as linguagens formais na seguintes ordem: regulares, livres do contexto, recursivas e recursivamente enumeráveis. Nas próximas seções será apresentado de forma superficial a ideia de autômatos e gramáticas.

\section{Sobre Autômatos Finitos}\label{sec:Automato}

Como dito em \cite{valdi2016master, valdi2020phd} uma definição informal do conceito de autômato finito (ou máquina de estado finita) e que tais dispositivos podem ser vistos como sendo máquinas (ou computadores) com dois componentes fundamentais: (1) Um conjunto finito de memórias\footnote{Na literatura também é usado o termo fita em vez de memória \cite{menezes1998LFA}.}, estas sendo subdivididas em células, cada uma das quais capaz de comportar um único símbolo por vez e (2) Uma unidade de controle\footnote{Também chamada de unidade central de processamento (UCP).} que  administra o estado atual do autômato e é responsável por executar as instruções (programa) da máquina.

Com respeito as memórias é comum assumir a existência de um \textbf{dispositivo de leitura e (ou) escrita}\footnote{Também é usado a nomenclatura cabeçote \cite{valdi2020phd, valdi2016master}.} que é capaz de acessar uma única célula por vez, e assim pode lê e (ou) escrever na célula. A depender do tipo de autômato podem existir vários dispositivos de leitura/escrita ou apenas um \cite{benjaLivro2010}.

A(s) memória(s) de um autômato finito serve(m) para guarda dados (os símbolos) usados durante o funcionamento do autômato. O funcionamento de um autômato por sua vez, pode ser descrito em tempo discreto \cite{valdi2016master, valdi2020phd}, assim sendo, em qualquer momento  no tempo $t$, a \textbf{unidade de controle} do autômato estará sempre em algum \textbf{estado} interno possível e a(s) \textbf{unidade(s) de leitura/escrita} tem acesso a alguma(s)  \textbf{célula(s)} da(s) memória(s).

\begin{figure}[ht]
	\centering
	\begin{tikzpicture}
		\tikzstyle{every path}=[very thick]
		
		\edef\sizetape{0.7cm}
		\tikzstyle{tmtape}=[draw,minimum size=\sizetape]
		\tikzstyle{tmhead}=[arrow box,draw,minimum size=.5cm,arrow box
		arrows={east:.25cm, west:0.25cm}]
		
		%% Fita
		\begin{scope}[start chain=1 going right,node distance=-0.15mm]
			\node [on chain=1,tmtape] (input) {$y_1$};
			\node [on chain=1,tmtape] (raiz1) {$\ldots$};
			\node [on chain=1,tmtape] (alvo)  {$y_i$};
			\node [on chain=1,tmtape] (raiz2) {$\ldots$};
			\node [on chain=1,tmtape] (output){$y_n$};
			\node [on chain=1,xshift=0.3cm]        (descr) {\textbf{Memória do autômato}};
		\end{scope}
		
		%% Unidade de Controle
		\begin{scope}
			[shift={(1.5cm,-3.7cm)},start chain=circle placed {at=(-\tikzchaincount*30:1.5)}]
			%\foreach \i in {q_0,p_1,q_2,q_3, q_4,\ddots,q_n}
			\foreach \i in {q_0,q_1, q_2,q_3, q_4,q_5, q_6,q_7, \cdots,\ddots, q_{n-1},q_n}
			\node [on chain] {$\i$};
			
			% Seta para estado corrente
			\node (center) {};
			\draw[->] (center) -- (circle-2);
			
			\node[rounded corners,draw=black,thick,fit=(circle-1) (circle-2) (circle-3) 
			(circle-4) (circle-5) (circle-6) (circle-7) (circle-8) 
			(circle-9) (circle-10) (circle-11) (circle-12),
			label=right:\textbf{Unidade de Controle}] (fsbox)
			{};
		\end{scope}
		
		%% Draw TM head below (input) tape cell
		\node [draw=white, thick, yshift=-.3cm] at (alvo.south)   (head3) {};
		
		
	\end{tikzpicture}
	\caption{Conceito informal de autômato finito com uma única memória retirado de \cite{valdi2020phd}.}
	\label{fig:AF-Informal}
\end{figure}

Formalmente pode-se dizer como apontado em \cite{valdi2020phd}, que a teoria dos autômatos finitos, ou simplesmente teoria dos autômatos, teve seu desenvolvimento inicial entre os anos de 1940 e 1960 sendo este início os trabalhos de McCulloch e Pitts \cite{mcculloch1943}, Kleene \cite{kleene1951}, Mealy \cite{mealy1955}, Moore \cite{moore1956}, Rabin e Scott \cite{rabin1963, rabin1959}. De forma geral os autômatos finitos são os mais simples modelos abstratos de máquinas de computação \cite{farias2017}, sendo eles máquinas de Turing limitadas. 

\section{Sobre Gramática Formais}\label{sec:GramaticaFormal}

Agora que foram introduzidos os conceitos fundamentais para a teoria dos autômatos, este prossegue apresentado formalmente o conceito de estrutura geradora ou gramática formal, o leitor mais atento e com maior conhecimento sobre lógica de primeira ordem e teoria da prova \cite{avigad1998, buss1998} pode notar que gramáticas formais são na verdade outro nome para  sistemas de reescrita \cite{ayala2014}.

\begin{definition}[Gramática formal]\label{def:GramaticaFormal}
	Uma gramática formal é uma estrutura com a seguinte forma $G = \langle V, \Sigma, S, P \rangle$ onde $V$ é um conjunto não vazio de símbolos chamados variáveis tal que $V \cap \Sigma = \emptyset$, $\Sigma$ é um alfabeto, $S \in V$ é uma variável destacada chamada de \textbf{variável inicial} e $P$ é um conjunto de regras de reescrita\footnote{Tamém é comum encontrar na literatura a nomenclatura regras de produção \cite{benjaLivro2010, linz2006}.}\footnote{Um leitor com o mínimo de maturidade matemática pode observar que o símbolo $P$ é na verdade o rótulo de uma relação binária, sendo claramente $P \subseteq (V \cup \Sigma)^+ \times (V \cup \Sigma)^*$.} da forma $w \rhd w'$ onde $w \in (V \cup \Sigma)^+$ e $w' \in (V \cup \Sigma)^*$.
\end{definition}

\begin{example}\label{exe:GramaticaFormal1} 
	A estrutura $G = \langle \{A, B\}, \{a\}, A, P \rangle$ em que $P$ é formado pelas regras $A \rhd aABa, A \rhd B$ e $B \rhd \lambda$ é uma gramática formal.
\end{example}

Qualquer gramática então pode ser visto com um sistema para a geração de palavras através de um mecanismo chamado derivação descrito a seguir.

\begin{definition}[Derivação de palavras]
	Dado uma gramática $G = \langle V, \Sigma, S, P \rangle$, a palavra $XwY$ deriva a palavra $Xw'Y$  na gramática $G$, denotado por $XwY \vdash_G Xw'Y$, sempre que existe uma regra forma $w \rhd w' \in P$.
\end{definition}

\begin{example}
	Dado a gramática do Exemplo \ref{exe:GramaticaFormal1} tem-se que $aABa \vdash_G aaABaBa$, pois existe em $P$ a regra $A \rhd aABa$.
\end{example}

Rigorosamente $\vdash_G$ na verdade é uma relação entre $(V \cup \Sigma)^+$ e $(V \cup \Sigma)^*$, e assim $\vdash_G^*$ denota o fecho transitivo e reflexivo de  $\vdash_G$, além disso, sempre que não causar confusão é comum eliminar a escrita do rótulo da gramática, ou seja, são escritos respectivamente $\vdash^*$ e $\vdash$ em vez de $\vdash_G^*$ e $\vdash_G$.

\begin{example}
	Considerando ainda a gramática exibida no Exemplo \ref{exe:GramaticaFormal1} tem-se que $aABa \vdash^* aaaABaBaBa$, uma vez que, $aABa \vdash aaABaBa \vdash aaaABaBaBa$.
\end{example}

\begin{example}
	A estrutura $G = \langle \{A, B, S\}, \{0,1\}, S, P \rangle$ em que $P$ é formado pelas regras $S \rhd 11A, A \rhd B0$ e $B \rhd 000$ é uma gramática formal e assim $11A \vdash^* 110000$, pois tem-se que, $11A \vdash^* 11B0 \vdash^* 110000$.
\end{example}

Como dito em \cite{benjaLivro2010}, dado uma gramática formal $G$ tem-se que sempre houver uma sequencia de derivações $w_1 \vdash w_2 \vdash \cdots \vdash w_n$ acontecer, as palavras $w_1, w_2, \cdots, w_n$ são chamadas de formas sentenciais, ou simplesmente, sentenças. Assim uma derivação nada mais é do que uma sequência finita de formas sentenciais.

\begin{definition}[Igualdade de Derivações]\label{def:IgualdadeDerivacaoGramatica}
	Dado uma gramática $G = \langle V, \Sigma, S, P \rangle$ e duas derivações $S \vdash w_1 \vdash^* w_n$ e $S \vdash w'_1 \vdash^* w'_n$ sobre $G$, será dito que estas derivações são iguais sempre que $w_i = w'_i$ para todo $1 \leq i \leq n$.
\end{definition}

Desde que $\vdash^*$ é de fato uma relação tem-se que a igualdade entre derivações nada mais é do que a igualdade entre tuplas ordenadas.

\begin{definition}[Linguagem de uma gramática]\label{def:LinaugemGramatica}
	Dado uma gramática $G = \langle V, \Sigma, S, P \rangle$ a linguagem gerada por $G$, denotada por $\mathcal{L}(G)$, corresponde ao conjunto formado por todas as palavras sobre $\Sigma$ que são deriváveis a partir do variável inicial da gramática, ou seja, $\mathcal{L}(G) = \{w \in \Sigma^* \mid S \vdash^* w\}$.
\end{definition}

\begin{example}
	Não é difícil verificar que a gramática do Exemplo \ref{exe:GramaticaFormal1} gera a linguagem $\{w \in \{a\}^* \mid |w| = 2k, k \in \mathbb{N}\}$.
\end{example}

O leitor pode ter notado que como gramática formais possuem um conjunto finito de regras, as linguagens por elas geradas nada mais são do que conjuntos indutivamente gerados.

\section{Questionário}\label{sec:Questionario1part4}

\begin{problem}\label{prob:Linguagem1}
	Dado o alfabeto $\Sigma = \{a, b, c\}$ e as palavras $u = aabcab, v = bbccabac$ e $w = ccbabbaaca$ determine o que é solicitado.
\end{problem}

\begin{exerList}
	\item A palavra $uv^r$.
	\item A palavra $(w^r)^2u$.
	\item A palavra $((u^r)^2v^0)^rv$.
	\item A palavra $uu^2v^rw$.
	\item A palavra $((wuv)^r)^2u$.
	\item O valor da expressão $|w^3| + 2|v^2u| - |u|$.
	\item O valor da expressão $2|w^r| - |uv|$.
	\item O valor da expressão $|w^raaw| - |w|$.
	\item O valor da expressão $|uv^r| - 4$.
	\item O valor da expressão $\frac{|(w^r)^2u|}{2} - \frac{|u|}{6}$.
\end{exerList}

\begin{problem}\label{prob:Linguagem2}
	Demonstre para quaisquer palavras $u$ e $v$ e para todo $n \in \mathbb{N}$ as asserções a seguir.
\end{problem}

\begin{exerList}
	\item Se $u$ é um prefixo de $v$, então $|u| \leq |v|$.
	\item $|u^n| = n|u|$.
	\item $|(uv)^r| = |vu|$.
	\item Se $|u| = n$, então $n \leq |uv|$.
\end{exerList}

\begin{problem}\label{prob:Linguagem3}
	Considere a linguagem $L = \{\lambda, abb, a, abba\}$ e determine o que é solicitado a seguir.
\end{problem}

\begin{exerList}
	\item $L^r - \{\lambda, a\}$.
	\item $L^3$.
	\item $PRE(L)$.
	\item $SUF(L^2)$.
	\item todos os $w$'s tais que $|w| = \bigsqcup \{|w'| \mid w' \in L^3\}$.
\end{exerList}

\begin{problem}\label{prob:Linguagem4}
	Prove que para qualquer linguagem $L$ e quaisquer $m,n \in \mathbb{N}$ as seguintes asserções.
\end{problem}

\begin{exerList}
	\item $(L^m)^n = L^{mn}$.
	\item $L^mL^n = L^{m+n}$.
	\item $(L^r)^n = (L^n)^r$.
	\item $\overline{L}^r = \overline{L^r}$.
	\item $PRE(L) = (SUF(L^r))^r$.
\end{exerList}

\begin{problem}\label{prob:Linguagem5}
	Dado duas linguagens quaisquer $L_1$ e $L_2$ demostre que:
\end{problem}

\begin{exerList}
	\item Se $L_1 \cap L_2 = \emptyset$, então $PRE(L_1) \cap PRE(L_2) \neq \emptyset$.
	\item Se $L_1 \subseteq L_2$, então $SUF(L_1) \cap SUF(L_2) = \emptyset$.
	\item Se $L_1 \subseteq L_2$, então $L_1^r \subseteq L_2^r$.
	\item Se $L_1 \subseteq L_2$, então para todo $L$ tem-se que $LL_1 \subseteq LL_2$. 
\end{exerList}

\begin{problem}\label{prob:Linguagem6}
	Demonstre ou refute o predicado $(\forall L \subseteq \Sigma^*)[(\forall n \in \mathbb{N})[\overline{L}^n = \overline{L^n}]]$.
\end{problem}

\begin{problem}\label{prob:Linguagem7}
	Dado uma linguagem finita e não vazia $L$, demonstre por indução $L^n$ é sempre finita para qualquer $n \in \mathbb{N}$.
\end{problem}

\begin{problem}\label{prob:Linguagem8}
	Esboce formalmente em que condições a igualdade $PRE(L) = (SUF(L))^r$ é verdadeira.
\end{problem}