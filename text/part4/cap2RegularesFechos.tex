\chapter{Álgebra das Linguagens Regulares}\label{cap:AlgebrasRegulares}

%\epigraph{Sem o fogo do Entusiamo, não há o calor da vitória.}{Provérbio Chinês}

%\epigraph{A Vida. Pode-se odiá-la ou ignorá-la, mas nunca gostar dela!}{Douglas Noël Adams, Série -- O Guia do Mochileiro das Galáxias.}

\epigraph{Você acha que VOCÊ tem problemas? Experimente ser um robô maníaco depressivo...}{Douglas Noël Adams, O Guia do Mochileiro das Galáxias.}

\section{Operadores Booleanos}

Neste capítulo serão estudadas  as operações que fecham algebricamente o conjunto (ou classe) de todas as linguagens regulares $\mathcal{L}_{Reg}$, isto é, serão apresentados aqui o operadores de fecho para as linguagens regulares. A maior parte dos resultados demonstrados aqui foi provado inicialmente no seminal \textit{paper} \cite{rabin1959}.

\begin{theorem}[Fecho do Complemento]\label{teo:FechoComplementoRegular}
	Se $L$ é uma linguagem regular, então $\overline{L}$ também é uma linguagem regular.
\end{theorem}

\begin{proof}
	Assuma que $L$ seja uma linguagem regular, assim por definição existe um AFD $A = \langle Q, \Sigma, \delta, q_0, F\rangle$ tal que $\mathcal{L}(A) = L$, agora defina um no novo AFD $\overline{A} = \langle Q, \Sigma, \delta, q_0, \overline{F} \rangle$ tal que $\overline{F} = Q - F$, agora para todo $w \in \Sigma^*$ tem-se que:
	\begin{eqnarray*}
		w \in \mathcal{L}(\overline{A}) & \Longleftrightarrow & \widehat{\delta}(q_0, w) \in \overline{F}\\
		& \Longleftrightarrow & \widehat{\delta}(q_0, w) \notin F\\
		& \Longleftrightarrow & w \notin \mathcal{L}(A)
	\end{eqnarray*}
	consequentemente, $\mathcal{L}(\overline{A}) = \overline{L}$, e desde que $\overline{A}$ é um AFD, tem-se que $\overline{L}$ também é uma linguagem regular, concluindo assim a prova.
\end{proof}

O AFD construído na demonstração do Teorema \ref{teo:FechoComplementoRegular} é chamado de \textbf{autômato complementar}, o exemplo a seguir mostra a construção de um autômato complementar de um AFD dado.

\begin{example}\label{exe:FechoAFD1}
	Considere o AFD $A$ representado na Figura \ref{fig:FechoAFD1} que reconhece a linguagem de todas as palavras com a quantidade par de a's e uma quantidade qualquer de b's.
	
	\begin{figure}[ht]
		\centering
		\begin{tikzpicture}[>=stealth, shorten >=1pt, node distance=5.0cm, on grid, auto, state/.append style={minimum size=3em}, thick ]
			\node[state, initial, accepting]   (A)		  {$q_0$};
			\node[state] (B) [right of=A]  						 {$q_1$};
			
			\path[->] (A) +(-1,0) edge (A)
			
			%Transições:
			%(Partida) edge [tipo da seta] node {simbolo lido} (Destino)
			(A) edge [bend left]  				node {$a$}     		(B)
			(B) edge [bend left]  				node {$a$}     		(A)
			(A) edge [loop above]			 node {$b$}          ()
			(B) edge [loop above]			 node {$b$}          ();
		\end{tikzpicture}
		\caption{AFD $A$ para a linguagem $\{w \in \{a, b\}^* \mid |w_a| = 2n \text{ com } n, |w|_b \in \mathbb{N}\}$.}
		\label{fig:FechoAFD1}
	\end{figure}

	O AFD complementar ao AFD da Figura \ref{fig:FechoAFD1}  pode ser visto na Figura \ref{fig:FechoAFD2}.
	 
	\begin{figure}[ht]
		\centering
		\begin{tikzpicture}[>=stealth, shorten >=1pt, node distance=5.0cm, on grid, auto, state/.append style={minimum size=3em}, thick ]
			\node[state, initial]   (A)		  {$q_0$};
			\node[state,  accepting] (B) [right of=A]  						 {$q_1$};
			
			\path[->] (A) +(-1,0) edge (A)
			
			%Transições:
			%(Partida) edge [tipo da seta] node {simbolo lido} (Destino)
			(A) edge [bend left]  				node {$a$}     		(B)
			(B) edge [bend left]  				node {$a$}     		(A)
			(A) edge [loop above]			 node {$b$}          ()
			(B) edge [loop above]			 node {$b$}          ();
		\end{tikzpicture}
		\caption{AFD $A$ para a linguagem $\{w \in \{a, b\}^* \mid |w_a| = 2n+1 \text{ com } n, |w|_b \in \mathbb{N}\}$.}
		\label{fig:FechoAFD2}
	\end{figure}
\end{example}

\begin{example}\label{exe:FechoAFD2}
	Seguindo a estratégia exposta na prova do Teorema \ref{teo:FechoComplementoRegular} o AFD complementar ao AFD da Figura \ref{fig:AFD2} corresponde exatamente ao AFD representado na Figura \ref{fig:FechoAFD3}, e é claro que a linguagem de tal AFD corresponde exatamente a linguagem $\{0,1\}^*$
	
	\begin{figure}[h]
		\centering
		\begin{tikzpicture}[>=stealth, shorten >=1pt, node distance=2.5cm, on grid, auto, state/.append style={minimum size=3em}, thick ]
			\node[state, initial,  accepting]   			(A)               {$s_0$};
			\node[]		  						(C) [right of=A]  {};
			\node[state,  accepting] 						(B) [above of=C]  {$s_1$};
			\node[state,  accepting] 						(D) [below of=C]  {$s_2$};
			
			\path[->] (A) +(-1,0) edge (A)
			
			%Transições:
			%(Partida) edge [tipo da seta] node {simbolo lido} (Destino)
			(B) edge [loop right]               node {$1$}           ( )
			(A) edge			  				node {$0$}     		 (B)
			(A) edge			  				node {$1$}     		 (D)
			(B) edge [bend left]  				node {$0$}     		 (D)
			(D) edge [bend left]  				node [right] {$0, 1$}(B);
		\end{tikzpicture}
		\caption{AFD complementar ao AFD da Figura \ref{fig:AFD2}.}
		\label{fig:FechoAFD3}
	\end{figure}
\end{example}

Para os dois próximos resultados considere sempre que as linguagens são definidas sobre o mesmo alfabeto, isto é, $L_1, L_2 \subseteq \Sigma^*$. 

\begin{theorem}[Fecho do União]\label{teo:FechoUniaoRegular}
	Se $L_1$ e $L_2$ são linguagens regulares, então $L_1 \cup L_2$ também é uma linguagem regular.
\end{theorem}

\begin{proof}
	Assuma que $L_1$ e $L_2$ são linguagens regulares, logo existem $A_1 = \langle Q, \Sigma, \delta_1, q_0,  F_1\rangle$ e $A_2 = \langle S, \Sigma, \delta_2, s_0,  F_2\rangle$ com $A_1$ e $A_2$ sendo AFD e, além disso, $\mathcal{L}(A_1) = L_1$ e $\mathcal{L}(A_2) = L_2$, sem perda de generalidade assuma que $Q \cap S = \emptyset$, agora construa um $\lambda$-AFN $A = \langle Q \cup S \cup \{q_{init}\}, \Sigma, \underline{\delta_N}, q_{init}, F\rangle$ tal que $q_{init} \notin Q \cup S$ e $F = F_1 \cup F_2$, além disso, tem-se que:
	\begin{eqnarray*}
		\underline{\delta_N}(q, a) & = & \left\{\begin{array}{ll}	\{\delta_1(q, a)\}, & \hbox{se } q \in Q, a \in \Sigma \\	\{\delta_2(q, a)\},  & \hbox{se } q \in S, a \in \Sigma \\ \{q_0, s_0\}, &   \hbox{se } q = q_{init}, a = \lambda \\ \emptyset, & \hbox{qualquer outro caso} \end{array}\right.
	\end{eqnarray*}
	agora para todo $w \in \Sigma^*$ tem-se que, 
	\begin{eqnarray*}
		w \in \mathcal{L}(A) & \Longleftrightarrow & \widehat{\underline{\delta_N}}(q_{init}, w) \cap F \neq \emptyset
	\end{eqnarray*}
	mas pela construção de $\underline{\delta_N}$ tem-se que,
	\begin{eqnarray*}
		\widehat{\underline{\delta_N}}(q_{init}, w) \cap F \neq \emptyset & \Longleftrightarrow & \{\widehat{\delta_1}(q_{0}, w)\} \cup  \{\widehat{\delta_2}(s_{0}, w)\} \neq \emptyset\\
		& \Longleftrightarrow & \{\widehat{\delta_1}(q_{0}, w)\} \neq \emptyset \text{ ou } \{\widehat{\delta_2}(s_{0}, w)\} \neq \emptyset \\
		& \Longleftrightarrow & (\exists q_i \in F_1)[\widehat{\delta_1}(q_{0}, w) = q_i] \text{ ou }  (\exists s_j \in F_2)[\widehat{\delta_2}(s_{0}, w) = s_j]\\
		& \Longleftrightarrow & w \in \mathcal{L}(A_1) \text{ ou } w \in \mathcal{L}(A_2) \\
		& \Longleftrightarrow & w \in \mathcal{L}(A_1) \cup \mathcal{L}(A_2) \\
		& \Longleftrightarrow & w \in L_1 \cup L_2
	\end{eqnarray*}
	assim $\mathcal{L}(A) = L_1 \cup L_2$, e desde que $A$ é um $\lambda$-AFN pelo Corolário \ref{col:RegularLAFN} tem-se que $L_1 \cup L_2$ será uma linguagem regular, o completa a prova.
\end{proof}

\begin{theorem}[Fecho do Interseção]\label{teo:FechoIntersecaoRegular}
	Se $L_1$ e $L_2$ são linguagens regulares, então $L_1 \cap L_2$ também é uma linguagem regular.
\end{theorem}

\begin{proof}
	Assuma que $L_1$ e $L_2$ são duas linguagens regulares, assim existem dois AFD $A_1$ e $A_2$ com $\mathcal{L}(A_1) = L_1$ e $\mathcal{L}(A_2) = L_2$, pelo Teorema \ref{teo:FechoComplementoRegular} tem-se que existem dois dois AFD $A'_1$ e $A'_2$ tais que $\mathcal{L}(A'_1) = \overline{\mathcal{L}(A_1)}$ e $\mathcal{L}(A'_2) = \overline{\mathcal{L}(A_2)}$, agora pelo Teorema \ref{teo:FechoUniaoRegular} tem-se que $\overline{\mathcal{L}(A_1)} \cup \overline{\mathcal{L}(A_2)}$ é uma linguagem regular, logo existe um AFD $A_0$ tal que $\mathcal{L}(A_0) = \overline{\mathcal{L}(A_1)} \cup \overline{\mathcal{L}(A_2)}$, e novamente pelo Teorema \ref{teo:FechoComplementoRegular} tem-se que $\overline{\overline{\mathcal{L}(A_1)} \cup \overline{\mathcal{L}(A_2)}}$ é uma linguagem regular, mas pelas operações sobre conjunto tem-se que $\overline{\overline{\mathcal{L}(A_1)} \cup \overline{\mathcal{L}(A_2)}} = \mathcal{L}(A_1) \cap \mathcal{L}(A_2)$, consequentemente, $L_1 \cap L_2$ é regular.
\end{proof}

Note que o Teorema \ref{teo:FechoIntersecaoRegular} determinar que a interseção de duas linguagens regulares ainda é uma linguagem regular, mas na prova do mesmo não é mostrado um autômato que reconhece tal linguagem, o que contrasta com a prova do Teorema \ref{teo:FechoUniaoRegular}, apesar disso, através do Teorema \ref{teo:FechoIntersecaoRegular} pode-se concluir o resultado a seguir.

\begin{corollary}
	Dado duas linguagem regulares $L_1$ e $L_2$, existe um AFD $A$ tal que $\mathcal{L}(A) = L_1 \cap L_2$.
\end{corollary}

\begin{proof}
	Assuma que $L_1$ e $L_2$ são duas linguagens regulares, assim por definição existem dois AFD $A_1 = \langle Q, \Sigma, \delta_1, q_0, F_1 \rangle$ e $A_2 = \langle S, \Sigma, \delta_2, s_0, F_2 \rangle$ tal que $\mathcal{L}(A_1) = L_1$ e $\mathcal{L}(A_2) = L_2$, sem perda de generalidade assuma que $Q \cap S = \emptyset$, agora construa o autômato $A = \langle Q \times S, \Sigma, \delta, (q_0, s_0), F_1 \times F_2\rangle$ onde para todo $(q, s) \in Q \times S$ e $a \in \Sigma$ tem-se que, 
	\begin{eqnarray*}
		\delta((q, s), a) = (\delta_1(q, a), \delta_2(s, a))
	\end{eqnarray*}
	desde que $\delta_1$ e $\delta_2$ são funções totais, tem-se que $\delta$ também será total, além disso, como $F_1 \times F_2 \subseteq Q \times S$, pode-se concluir que $A$ é um autômato bem definido. Agora desde que $A_1$ e $A_2$ são ambos AFD e pela construção de $\delta$ é claro que $A$ também será um AFD, além disso, por indução sobre o tamanho das palavras $w$ tem-se para todo $(q, s) \in Q \times S$ que, 
	\begin{eqnarray*}
		\widehat{\delta}((q, s), w) = (\widehat{\delta_1}(q, w), \widehat{\delta_2}(s, w))
	\end{eqnarray*}
	por fim note que para todo $w \in \Sigma^*$ tem-se que,
	\begin{eqnarray*}
		w \in \mathcal{L}(A) & \Longleftrightarrow & \widehat{\delta}((q_0, s_0), w) \in F_1 \times F_2\\
		& \Longleftrightarrow & (\widehat{\delta_1}(q_0, w), \widehat{\delta_2}(s_0, w)) \in F_1 \times F_2\\
		& \Longleftrightarrow & \widehat{\delta_1}(q_0, w) \in F_1 \text{ e } \widehat{\delta_2}(s_0, w) \in F_2\\
		& \Longleftrightarrow & w \in \mathcal{L}(A_1) \text{ e } w \in \mathcal{L}(A_2)\\
		& \Longleftrightarrow & w \in \mathcal{L}(A_1) \cap \mathcal{L}(A_2)
	\end{eqnarray*}
	portanto, tem-se que $\mathcal{L}(A) = L_1 \cap L_2$, o que conclui a prova.
\end{proof}

Agora desde que os operadores conjuntistas são todos fechados em $\mathcal{L}_{Reg}$ pode-se concluir o resultado que se segue.

\begin{corollary}[Álgebra Booleana Regular]
	O conjunto $\mathcal{L}_{Reg}$ é uma álgebra booleana.
\end{corollary}

\begin{proof}
	Direto desde que $\mathcal{L}_{Reg}$ é fechado para união, interseção e complemento.
\end{proof}

\section{Operadores Não Booleanos}

Nesta seção dando continuidade ao estudo dos operadores de fecho sobre as linguagens regulares, serão estudados os operadores não booleanos, isto é, aqueles operadores que não são naturalmente visto dentro da teoria ingenua (ou intuitiva) de conjunto \cite{abe1991-TC, lipschutz1978-TC}. Assim o próximo operador de fecho para as linguagens regulares que será investigado é o operador de concatenação de linguagem, que foi definido anteriormente (Definição \ref{def:ConcatenacaoLinguagem}) no Capítulo \ref{cap:LinguagemFormais}.

\begin{theorem}[Fecho do Concatenação]\label{teo:FechoConcatenacaoRegular}
	Se $L_1$ e $L_2$ são linguagens regulares, então $L_1L_2$ também é uma linguagem regular.
\end{theorem}

\begin{proof}
	Assuma que $L_1$ e $L_2$ são linguagens regulares, assim pelo Teorema \ref{teo:GRD-AFD} existem duas GLD $G_1 = \langle V_1, \Sigma, S_1, P_1 \rangle$ e $G_2 = \langle V_2, \Sigma, S_2, P_2 \rangle$, sem perda de generalidade assuma que $V_1 \cap V_2 = \emptyset$. Assim pode-se construir uma nova GLD $G = \langle V_1 \cup V_2 \cup \{S\}, \Sigma, S, P\rangle$ com $P = P_1 \cup P_2 \cup \{S \rhd S_1S_2\}$, agora para todo $w \in \Sigma^*$ tem-se que,
	\begin{eqnarray*}
		w \in \mathcal{L}(G) & \Longleftrightarrow & S \vdash^*_G w\\
		& \Longleftrightarrow & S \vdash S_1S_2 \vdash^*_G w\\
		& \Longleftrightarrow & (\exists u, v \in \Sigma)[w = uv \land S_1 \vdash^*_{G_1} u \land S_2 \vdash^*_{G_2} v]\\
		& \Longleftrightarrow & (\exists u, v \in \Sigma)[w = uv \land u \in \mathcal{L}(G_1) \land v \in \mathcal{L}(G_2)]\\
		& \Longleftrightarrow &  w \in \mathcal{L}(G_1) \mathcal{L}(G_2)\\
		& \Longleftrightarrow &  w \in L_1L_2
	\end{eqnarray*}
	portanto, $\mathcal{L}(G) = L_1L_2$, e desde que $G$ é uma GLD pelo Teorema \ref{teo:GRD-AFD} tem-se que $L_1L_2$ é uma linguagem regular.
\end{proof}

\begin{lemma}[Fecho da Potência]\label{lema:FechoPotenciaRegular}
	Se $L$ é uma linguagem regular, então para todo $n \in \mathbb{N}$ tem-se que $L^n$ é uma linguagem regular.
\end{lemma}

\begin{proof}
	Inicialmente note que pela Definição \ref{def:PotenciaLinguagens} tem-se para todo $n \in \mathbb{N}$ que $L^n = LL^{n-1}$, e também que $L^0 = \{\lambda\}$, dito isso, assuma que $L$ é uma linguagem regular, agora por indução sobre $n$ será demonstrado que $L^n $ será uma linguagem regular.
	
	\begin{itemize}
		\item \textbf{Base da indução}
		
		Desde que $L^0 = \{\lambda\}$ e como $G = \langle \{S\}, \Sigma, S, P\rangle$ com $P = \{S \rhd \lambda \}$ é uma gramática regular tal que $\mathcal{L}(G) =L^0$, e portanto,  $L^0$ é regular.
		
		\item \textbf{Hipótese indutiva (HI)}: 
		
		Assuma que $L^n$ é uma linguagem regular.
		
		\item \textbf{Passo indutivo:}
		
		Dado a linguagem $L^{n+1}$, sabe-se $L^{n+1} = LL^{n}$, agora desde que por hipótese $L$ é uma linguagem regular e por \textbf{(HI)} $L^n$ é também uma linguagem regular, tem-se pelo Teorema \ref{teo:FechoConcatenacaoRegular} que $LL^{n}$ é uma linguagem regular e, portanto, $L^{n+1}$ é uma linguagem regular.
	\end{itemize}
	Consequentemente para qualquer que seja o $n \in \mathbb{N}$ tem-se que $L^n$ é uma linguagem regular.
\end{proof}

O próximo resultado estabelece que o fecho de Kleene é um fecho para a classe de linguagens regulares, ou seja, para qualquer linguagem regular o seu fecho de Kleene também será uma linguagem regular.

\begin{theorem}[Fecho de Kleene]
	Se $L$ é uma linguagem regular, então $L^*$ é uma linguagem regular.
\end{theorem}

\begin{proof}
	Suponha que $L$ seja uma linguagem regular, assim dado a Definição \ref{def:FechoPositivoKleeneLinguagem} tem-se pelo Lema \ref{lema:FechoPotenciaRegular} e pelo Teorema \ref{teo:FechoUniaoRegular} que $L^*$ é uma linguagem regular.
\end{proof}

A próxima operação a ser estudada aqui será a noção de homomorfismo entre linguagens, definida a seguir.

\begin{definition}[Homomorfismo]
	Sejam $\Sigma_1$ e $\Sigma_2$ dois alfabetos e $f : \Sigma_1^* \rightarrow \Sigma_2^*$ uma função total, um homomorfismo $\widehat{f}$ de $\Sigma_1^*$ em $\Sigma_2^*$ é uma função definida  para todo $w \in \Sigma_1^*, a \in \Sigma_1$ recursivamente como:
	\begin{eqnarray*}
		\widehat{f}(\lambda) & = & \lambda\\
		\widehat{f}(wa) & = & \widehat{f}(w)f(a)
	\end{eqnarray*}
\end{definition}