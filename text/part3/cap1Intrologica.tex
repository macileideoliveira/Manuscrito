\chapter{Introdução à Lógica}\label{cap:IntroducaoLogica}

\section{O que é Lógica?}\label{sec:O-que-e-Logica}

Antes de apresentar uma descrição histórica da lógica, este texto começa pela árdua tarefa de apresentar de forma sucinta uma resposta para a pergunta, \textbf{``o que é a lógica?}''. Como dito em \cite{BenjaV1, copi1981}, a palavra lógica e suas derivações são familiares a quase todas (se não todas) as pessoas, de fato, é comum durante o cotidiano do dia a dia as pessoas recorrerem ao uso do termo lógica ou de seu derivados, sendo que na maioria das vezes seu uso está ligada à ideia de obviedade (ou certeza), por exemplo nas frases:

\begin{itemize}
    \item[(a)] É lógico que vou na festa.
    \item[(b)] É lógico que ciência da computação é um curso difícil.
    \item[(c)] Logicamente o Vasco não pode ganhar o título da primeira divisão nacional em 2021.
    \item[(d)] Logicamente se eu tomar banho, vou ter que me molhar.
\end{itemize}

Essa forma de usar os derivados da palavra lógica enquanto entidades para transmissão de certeza pode ser usada como gatilho ``fácil e preguiçoso'' para enunciar que a lógica se trata de uma ciência (ou disciplina) acerca das certezas sobre os fatos do mundo material. 

Existe outra resposta comumente encontrada na literatura acadêmica (ver \cite{abe2002-logica, BenjaV1, joaoPavao2014}) para o que seria a lógica, esta segunda alternativa de resposta  descreve a lógica como sendo um mecanismo utilizado durante o raciocínio estruturado e correto\footnote{Uma visão semelhante a esta é descrita em \cite{magnus2020}, que diz que a lógica está preocupada com a avaliação de argumentos e a separação dos argumentos bons dos ruins.}, isto é, uma ferramenta do raciocínio que possibilita a inferência de conclusões a partir de premissas \cite{abe2002-logica, copi1981, hodges1997, levin2021}, por exemplo, dado as premissas:

\begin{itemize}
	\item[(a)] Toda quinta-feira é servido peixe no almoço.
	\item[(b)] Hoje é quarta-feira.
\end{itemize}

O raciocínio munido da ``ferramenta de inferência'' contida na lógica permite deduzir a afirmação: \textbf{Amanhã será servido peixo no almoço}, como conclusão. Note que esta segunda resposta estabelece que a lógica é um tipo de procedimento mental capaz de transformar informações de entrada (as premissas) em informações de saída (a conclusão). 

Essas duas formas de encarar a lógica não estão totalmente erradas, entretanto, também não exibem de forma completa o real significado do que seria a lógica em si. Uma terceira resposta para a pergunta ``O que é a lógica?'' aparece na edição de 1953 da Encyclopædia Britannica na seguinte forma: ``\textit{Logic is the systematic study of the structure of propositions and of the general conditions of valid inference by a method which abstracts from the content or matter of the propositions and deals only with their logical form}''. Note que essa resposta utiliza-se de autorreferência\footnote{Autorreferência é um fenômeno que ocorre nas linguagens naturais e formais, tal fenômeno consiste de uma oração ou fórmula que refere-se a si mesma.}, pois a mesma tenta definir o que é a lógica em função do termo ``forma lógica''. 

Apesar dessa definição recursiva\footnote{Em matemática a ideia de definição recursiva está ligada a ideia de uma estrutura que apresenta autorreferência.}, a resposta da Encyclopædia Britannica apresenta duas características muitos marcantes para a apresentação da lógica enquanto ciência (ou disciplina) em nossos dias atuais. A primeira característica é a validade das afirmações derivadas (ou concluídas) pelo mecanismos de inferência. A segunda característica é a importância da forma de representação (a escrita) dos termos lógicos. 

A validade remonta a ideia de um significado dual (verdadeiro e falso) para as afirmações, ou seja, fornece indícios da existência de interpretações das afirmações, e isto significa que existem diferentes significados para as afirmações a depender de um fator que pode ser chamado de contexto, por exemplo considere a seguinte afirmação: 

\begin{center}
	``\textbf{O atual presidente americano é um democrata}''.
\end{center}

Note que o contexto temporal muda drasticamente o valor lógico interpretativo (semântico) dessa afirmação pois no ano de 2021 essa afirmação era interpretada como verdadeira, porém no ano de 2019 a mesma era falsa. Assim os valores interpretativos (semânticos) dentro do universo da lógica não são imutáveis, isto é, os valores das interpretações da lógica são passiveis de mudança a depender do contexto.

Dado então estes componentes sintáticos e semânticos pode-se concluir a partir das definições linguísticas que a \textbf{lógica é uma linguagem}, entretanto, vale salientar que não é uma linguagem natural como o português, como será visto nós próximos capítulos a lógica é uma linguagem formal \cite{benjaLivro2010}, no sentido de que todas as construções linguísticas possuem uma forma precisa e sem ambiguidade determinada por uma gramática geradora \cite{hopcroft2008, linz2006},  pode-se inclusive estabelece que a lógica é a linguagem da ciência da inferência racional, ou seja, a linguagem usada para representar argumentos, inferência e conclusões sobre um certo universo do discurso.

\section{Um Pouco de História}\label{sec:Um-pouco-historia}

A historia do desenvolvimento da lógica remonta até a Grécia antiga e a nomes como: Aristóteles (384-322 a.C.), Sócrates (469-399 a.C.), Zenão de Eléia (490-420 a.C.), Parmenides (515-445 a.C.), Platão (428-347 a.C.), Eudemus de Rodes (350-290 a.C.), Teofrastus de Lesbos (378-287 a.C.), Euclides de Megara (435-365 a.C.) e  Eubulides de Mileto\footnote{Creditado como criador do paradoxo do mentiroso.} (384-322 a.C.). De fato o nome lógica vem do termo grego \textit{logike}, cunhado por Alexandre de Afrodisias no fim do século II depois de Cristo. Como explicado em \cite{abe2002-logica}, os mais antigos registros sobre o estudo da lógica como uma disciplina (ciência) são encontrado exatamente na obra de Aristóteles intitulado como ``$\Gamma$ da metafísica''. Todavia, após seu desenvolvimento inicial dado pelos gregos antigos, a lógica permaneceu quase que intocada\footnote{Aqui não está sendo levando em conta as tentativas de Gottfried Wilhelm Leibniz (1646-1716) de desenvolver uma linguagem universal através da precisão matemática.} por mais de 1800 anos.

Os primeiros a profanar a santidade da lógica de forma contundente, abalando as estruturas da ideia que a lógica era uma ciência completa\footnote{No sentido de que não havia nada novo a se fazer, estudar ou provar.} foram os matemáticos George Boole (1815-1864) e Augustus De Morgan (1806-1871), que introduziram a moderna ideia da lógica como uma ciência simbólica, isto é, eles semearam os conceitos iniciais que depois iriam convergir para as ideias da lógica enquanto linguagem formal apresentadas pelo matemático e filósofo alemão Gottlob Frege (1848-1925), que via a lógica como uma linguagem, que continha em seu interior todo o rigor da matemática.

Ainda no século XIX os maiores defensores das ideias de Frege, os britânicos Alfred Whitehead (1861-1947) e Betrand Russel (1872-1970),  usaram muitas de suas ideias e sua linguagem na publicação monumental em três volumes intitulada ``\textit{Principia Mathematica}'' \cite{russel1910principia}, que é ainda hoje considerada por muitos o maior tratado matemático do século XIX. Como dito em \cite{BenjaV1}, outro influenciado por Frege que apresentou importantes contribuições foi filósofo austríaco Ludwig Wittgenstein (1889-1951), que em seu ``\textit{Tractatus Logico-Philosophicus}'' apresentou pela primeira vez a lógica proposicional através das tabelas verdade. Muitos autores, como é o caso de \cite{abe2002-logica}, consideram que a lógica moderna se iniciou verdadeiramente com a publicação do \textit{Principia}, de fato, alguns usam exatamente a visão de Whitehead que diz: ``A lógica atual está para a lógica aristotélica como a matemática moderna está para a aritmética das tribos primitivas''.

Outra vertente emergente na lógica do século XIX era aquela apoiada puramente por interesses matemáticos, isto é, a visão da lógica não apenas como linguagem, mas também como um objeto algebrizável (um cálculo). Tal escola de lógica encontra alguns de seus expoentes nos nomes de: Erns Zermelo\footnote{Zermelo junto com Fraenkel desenvolveu o sistema formal hoje conhecido como teoria axiomática dos conjuntos.} (1871-1953), Thoralf Skolem (1887-1963), Ludwig Fraenkel-Conrad (1910-1999), John von Neumman (1903-1957), Arend Heyting (1898-1980) entre outros. Uma das grandes contribuições feitas por essa escola foi incluir uma formulação explícita e precisa das regras de inferência no desenvolvimentos de sistemas axiomáticos.

Uma ramificação desta escola ``matemática'' ganhou força na Polônia sobre a tutela e liderança do lógico e filósofo Jan \L{}ukasiewics (1878-1956), o foco da escola polonesa era como dito em \cite{BenjaV1}, analisar os sistemas axiomáticos da lógica proposicional, lógica modal e das álgebras booleanas. Foi esta escola que primeiro considerou interpretações alternativas da linguagem (da lógica) e questões da meta-lógica tais como: consistência, corretude e completude. Por fim, foi na escola polonesa que houve pela primeira vez duas visões separadas sobre a lógica, uma em que a lógica era vista puramente como uma linguagem, e a segunda visão que via a lógica puramente como um cálculo \cite{BenjaV1}.

Instigado pelo problema número dois da lista de Hilbert o jovem matemático e lógico austríaco Kurt Gödel (1906-1978) fez grandes contribuições para a lógica, inicialmente ele provou o teorema da completude para a lógica de primeira ordem em sua tese de doutorado em 1929, tal resultado estabelece que uma fórmula de primeira ordem é dedutível se e somente se ela é universalmente válida \cite{BenjaV1}. Outra contribuição monumental de Gödel são seus teoremas da incompletude \cite{godel1931}, em especial o primeiro que deu uma resposta negativa ao problema número dois da lista Hilbert, de forma sucinta o resultado de Gödel estabelece que não pode haver uma sistematização completa da Aritmética  \cite{abe2002-logica}.

Outros contemporâneos de Gödel também contribuíram fortemente para a lógica, Alfred Tarski (1901-1983) foi o responsável pela matematização do conceito de verdade como correspondência \cite{abe2002-logica, tarski1983}, já o francês Jacques Herbrand (1908-1931) introduziu as funções recursivas e apresentou os resultados hoje chamados de teoria de Herbrand. Entre os resultados de Herbrand se encontra o teorema que relaciona um conjunto insatisfatível de fórmulas da lógica de primeira ordem com um conjunto insatisfatível de fórmulas proposicionais.

Outra enorme revolução matemática do século XX que foi escrita na linguagem da lógica foi a prova da independência entre a hipótese do \textit{continuum}\footnote{A hipótese do \textit{continuum} é uma conjectura proposta por Georg Cantor e que fazia parte da lista inicial de 10 problemas estabelecida por David Hilbert. Esta conjectura consiste no seguinte enunciado: \textbf{Não existe nenhum conjunto com cardinalidade maior que a do conjunto dos números inteiros e menor que a do conjunto dos números reais}.} e o axioma da escolha da teoria de conjuntos de Zermelo–Fraenkel ou teoria dos conjuntos axiomática como também é chamada.

De forma sucinta pode-se então concluir que a lógica uma ciência nascida na Grécia antiga se desenvolveu de forma exponencial após o século XIX, e que seu desenvolvimento foi em boa parte guiado por matemáticos, de fato, pode-se dizer que a lógica contemporânea se caracteriza pela tendência da matematização da lógica \cite{barreto1998}. Muitos outros estudiosos, além dos que foram aqui mencionados, também apresentaram resultados diretos em lógica ou em área correlatas como a teoria da prova e a teoria da recursão, tornando a lógica e suas ramificações e aplicações um dos assuntos dominantes no séculos XX e XXI. 

\section{Argumentos, Proposições e Predicados}\label{sec:Argumento-Proposicao-Predicado}

Como qualquer outra disciplina para entender de fato o que é a lógica deve-se estudar a mesma \cite{copi1981}, antes de qualquer coisa é bom saber que diferente de outras ciências a lógica não apresentar fronteiras bem definidas, na verdade como dito em \cite{joaoPavao2014}, a lógica pode ser compreendida como a tênue linha que separa as ciências da filosofia e da matemática, no que diz respeito a isto, este manuscrito irá se debruçar primariamente sobre os aspectos matemáticos da lógica.

É sabido que para se estudar uma ciência deve-se saber quais são as entidades fundamentais de interesse dessa ciência, no caso da lógica, estas entidades fundamentais são os argumentos dentro de um discurso. Antes de apresentar a noção formal de argumento é conveniente apresentar a ideia de frase declarativa (ou asserção). As frases declarativas usadas para construção de argumentos são aquelas que como dito em \cite{joaoPavao2014}, enunciam como as entidades em um certo discurso são ou poderia ter sido, em outras palavras, as frase declarativas falam sobre as propriedades das entidades.

\begin{example}\label{exe:FrasesDeclarativas}
    As frase:
	\begin{itemize}
		\item A lua é feita de queijo.
		\item O Flamengo é um time carioca.
		\item O uniforme principal da seleção brasileira é azul.
	\end{itemize}
	São ambas frases declarativas. Por outro lado, as frase:
	\begin{itemize}
		\item Que horas são?
		\item Forneça uma resposta para o exercício.
		\item Faça exatamente o que eu mandei.
		\item Cuidado!
	\end{itemize}
	Não são frases declarativas.
\end{example}

Uma forma de identificar se uma frase é declarativa é verificado se a mesma admite ser classificada como verdadeira ou falso.  Na lógica as frase declarativas podem ser ``tipadas'' com dois rótulos: \textbf{proposições} e \textbf{predicados}, formalizados em momentos futuros deste manuscrito. 

\begin{definition}[Argumento]\label{def:Argumento}
	Um argumento é um par formado por dois componentes básicos, a saber:
	\begin{itemize}
		\item[(1)] Um conjunto de frases declarativas, em que cada frase é chamada de premissa.
		\item[(2)] Uma frase declarativa, chamada de conclusão.
	\end{itemize}
\end{definition}

Para representar um argumento pode-se como visto em \cite{copi1981, joaoPavao2014} usar uma organização de linhas, por exemplo, para representar um argumento que possua $n$ premissas primeiro serão distribuídas nas $n$ primeiras linhas as tais premissas do argumento depois na linha $n+1$ é usado o símbolo $\therefore$ para separar as premissas da conclusão, sendo esta última colocada na linha $n+2$.

\begin{example}\label{exe:Argumento}
    São exemplos de argumentos:
    
		\begin{center}
			A sopa foi preparada  sem cebola\\ 
			Toda quarta-feira é servida sopa para as crianças.\\
			Hoje é quinta-feira.\\
			$\therefore$\\
			Ontem as crianças tomaram sopa.
		\end{center}
	
		e
		
		\begin{center}
			A lua é feita de queijo\\
			Os ratos comem queijo\\ 
			$\therefore$\\
			O imperador da lua é um rato.
		\end{center}
\end{example}

A validade de um argumento pode ser analisada em dois aspectos, o semântico e o sintático, este tipo de análise será estudada em capítulos futuros deste manuscrito.

\begin{definition}[Proposição]\label{def:Proposicao}
	Uma proposição é uma frase declarativa sobre as propriedades de indivíduos específicos em um discurso.
\end{definition}

\begin{example}\label{exe:Proposicoes}
    São exemplos de proposições:
	\begin{itemize}
		\item[(a)] $3 < 5$.
		\item[(b)] A lua é feita de queijo.
		\item[(c)] Albert Einstein era francês.
		\item[(d)] O Brasil é penta campeão de futebol masculino.
	\end{itemize}
\end{example}

\begin{definition}[Predicados]\label{def:Predicados}
	Predicados são frase declarativas sobre as propriedades de indivíduos não específicos em um discurso.
\end{definition}

Pela Definição \ref{def:Predicados} pode-se entender que um predicado fala das propriedades de indivíduos sem explicitamente dar nomes a tais indivíduos.

\begin{example}\label{exe:Predicados}
	São exemplos de predicados:
	\begin{itemize}
		\item[(a)] Para qualquer $x \in \mathbb{N}$ tem-se que $x < x + 1$.
		\item[(b)] Para todo $x \in \mathbb{R}$ sempre existem dois números $y_1, y_2 \in \mathbb{R}$ tal que $y_1 < x < y_2$.
		\item[(c)] Existe algum professor cujo nome da mãe é Maria de Fátima.
		\item[(d)] Há um estado brasileiro que não tem litoral.
		\item[(e)] Todo os moradores de Salgueiro são pernambucanos.
	\end{itemize}
\end{example}

Agora note que nas frase (a) e (b) do Exemplo \ref{exe:Predicados} o símbolo $x$ se torna um mecanismo que faz o papel dos números naturais e reais respectivamente, mas sem ser os próprios números em si, o mesmo vale para $y_1$ e $y_2$. Similarmente na frase (c) o termo \textbf{professor} representa todo um conjunto de pessoas, mas nunca sendo uma pessoa em particular, já na frase (d) o termo \textbf{estado brasileiro} representa novamente todos os indivíduos de um conjunto, mas ele nunca é um indivíduo particular. Os termos em um predicado que tem essa capacidade de representação são chamados de \textbf{variáveis do predicado}.

\begin{remark}
    Um predicado que tem suas variáveis substituídas por valores específicos se torna uma proposição. 
\end{remark}

\begin{example}\label{exe:AtribuirVariavelPredicado}
	Considere o predicado: \textbf{Existe algum professor cujo nome da mãe é Maria de Fátima}, se for atribuído o valor  \textbf{Valdigleis} no lugar da variável \textbf{professor} será gerado a proposição: \textbf{O nome da mãe de Valdigleis é Maria de Fátima}.
\end{example}

\section{Conectivos, Quantificadores e Negação}\label{sec:Conectivo-Quantificador-Negacao}

As proposições e os predicados podem ser classificados em duas categorias: simples ou composto. Uma proposição (ou predicado) é dita(o) composta(o) sempre que for possível dividi a (o) proposição (predicado) em proposições (predicados) menores. E no caso contrário é dito que a proposição (ou predicado) é simples (ou atômicas).

\begin{definition}[Conectivos]\label{def:Conectivos}
	Conectivos são termos linguísticos que fazem a ligação entre as proposições ou (e) predicados.
\end{definition}

Os principais conectivos são: a conjunção, a disjunção, a implicação e a bi-implicação.

\begin{remark}
    Dependendo do idioma mais de um termo da linguagem pode representar um determinado conectivo.
\end{remark}

A seguir são listados os termos na língua portuguesa que são conectivos, ressaltamos que o símbolo $\underline{ \ \ \ \ \ \ \ \ \ \ \ \ }$ será usado como uma variável para representar a posição de proposições (ou predicados).

\begin{table}[h]
	\centering
	%\scriptsize
	\begin{tabular}{cl}
		\hline
		\textbf{Conectivo}  & \textbf{Termo em Portugu\^es} \\ \hline
		\multirow{4}{*}{Conjunção}             & $\underline{ \ \ \ \ \ \ \ \ \ \ \ \ }$ e $\underline{ \ \ \ \ \ \ \ \ \ \ \ \ }$\\
		& $\underline{ \ \ \ \ \ \ \ \ \ \ \ \ }$ mas $\underline{ \ \ \ \ \ \ \ \ \ \ \ \ }$\\ 
		& $\underline{ \ \ \ \ \ \ \ \ \ \ \ \ }$ tamb\'em $\underline{ \ \ \ \ \ \ \ \ \ \ \ \ }$\\ 
		&$\underline{ \ \ \ \ \ \ \ \ \ \ \ \ }$ além disso $\underline{ \ \ \ \ \ \ \ \ \ \ \ \ }$\\ \hline
		Disjun\c{c}\~ao             & $\underline{ \ \ \ \ \ \ \ \ \ \ \ \ }$ ou $\underline{ \ \ \ \ \ \ \ \ \ \ \ \ }$\\ \hline
		\multirow{7}{*}{Implicação}
		& Se $\underline{ \ \ \ \ \ \ \ \ \ \ \ \ }$, ent\~ao $\underline{ \ \ \ \ \ \ \ \ \ \ \ \ }$\\
		& $\underline{ \ \ \ \ \ \ \ \ \ \ \ \ }$ implica $\underline{ \ \ \ \ \ \ \ \ \ \ \ \ }$    \\
		& $\underline{ \ \ \ \ \ \ \ \ \ \ \ \ }$ logo, $\underline{ \ \ \ \ \ \ \ \ \ \ \ \ }$\\
		& $\underline{ \ \ \ \ \ \ \ \ \ \ \ \ }$ s\'o se $\underline{ \ \ \ \ \ \ \ \ \ \ \ \ }$\\
		& $\underline{ \ \ \ \ \ \ \ \ \ \ \ \ }$ somente se $\underline{ \ \ \ \ \ \ \ \ \ \ \ \ }$\\
		& $\underline{ \ \ \ \ \ \ \ \ \ \ \ \ }$segue de $\underline{ \ \ \ \ \ \ \ \ \ \ \ \ }$ \\
		& $\underline{ \ \ \ \ \ \ \ \ \ \ \ \ }$ \'e uma condi\c{c}\~ao suficiente para $\underline{ \ \ \ \ \ \ \ \ \ \ \ \ }$\\
		& Basta $\underline{ \ \ \ \ \ \ \ \ \ \ \ \ }$ para $\underline{ \ \ \ \ \ \ \ \ \ \ \ \ }$\\
		& $\underline{ \ \ \ \ \ \ \ \ \ \ \ \ }$\'e uma condi\c{c}\~ao necess\'aria para $\underline{ \ \ \ \ \ \ \ \ \ \ \ \ }$ \\ \hline
		\multirow{2}{*}{Bi-implicação}
		& $\underline{ \ \ \ \ \ \ \ \ \ \ \ \ }$ se, e somente se $\underline{ \ \ \ \ \ \ \ \ \ \ \ \ }$\\
		& $\underline{ \ \ \ \ \ \ \ \ \ \ \ \ }$ \'e condição suficiente e necessária para $\underline{ \ \ \ \ \ \ \ \ \ \ \ \ }$\\ \hline
	\end{tabular}
	\caption{Termos em português que representamos conectivos.}
	\label{tab:ConectivosPT-BR}
\end{table}

%No que diz respeito aos conectivos este manuscrito irá adora o ``$\underline{ \ \ \ \ \ \ \ \ \ \ \ \ }$ e $\underline{ \ \ \ \ \ \ \ \ \ \ \ \ }$'' para a conjunção, ``$\underline{ \ \ \ \ \ \ \ \ \ \ \ \ }$ ou $\underline{ \ \ \ \ \ \ \ \ \ \ \ \ }$'' para a disjunção, ``se $\underline{ \ \ \ \ \ \ \ \ \ \ \ \ }$, então $\underline{ \ \ \ \ \ \ \ \ \ \ \ \ }$'' para a implicação e ``$\underline{ \ \ \ \ \ \ \ \ \ \ \ \ }$ se, e somente se $\underline{ \ \ \ \ \ \ \ \ \ \ \ \ }$'' para a bi-implicação.

\begin{example}\label{exe:Conectivos}
	Usando as proposições do Exemplo \ref{exe:Proposicoes} e os predicados do Exemplo \ref{exe:Predicados} pode-se criar:
	\begin{itemize}
		\item[(a)] $3 < 5$ e para qualquer $x \in\mathbb{N}$ tem-se que $x < x + 1$.
		\item[(b)] Há um estado brasileiro que não tem litoral ou O Brasil é penta campeão de futebol masculino.
		\item[(c)] Se para todo $x \in \mathbb{R}$ sempre existem dois números $y_1, y_2 \in \mathbb{R}$ tal que $y_1 < x < y_2$, então Albert Einstein era francês.
		\item[(d)] Para qualquer $x \in\mathbb{N}$ tem-se que $x < x + 1$ se, e somente se, para todo $x \in \mathbb{R}$ sempre existem dois números $y_1, y_2 \in \mathbb{R}$ tal que $y_1 < x < y_2$.
		\item[(e)] A lua é feita de queijo ou $3 < 5$.
	\end{itemize}
\end{example}

\begin{remark}
    O Exemplo \ref{exe:Conectivos} mostra que os conectivos podem ser usado para combinar proposições com proposições, predicados com predicados, predicados com proposições e vice-versa.
\end{remark}

Como já mencionado antes um predicado não especifica diretamente os indivíduos, em vez disso, usa variáveis para não mencionar os indivíduos especificamente. Essas variáveis por sua vez, estão conectadas a termos da linguagem que determinam a quantidade de elementos que podem vim a ser atribuído a tais variáveis, tais termos são chamados de quantificadores. Os quantificadores por sua vez, podem ser tipados em duas categoria: universais e existenciais. 

Quando uma variável é ligada a um quantificador universal significa que o predicado será verdadeiro se para a atribuição de cada um  dos elementos do universo discurso a proposição gerada com a atribuição é também verdadeira, no caso contrário o predicado é falso. Por outro lado, quando uma variável é ligada a um quantificador existencial significa que tal predicado será verdadeiro se para pelo menos um dos elementos do discurso ao ser atribuído a variável gera uma proposição verdadeira, e no caso contrário o predicado será falso.

De forma similar aos conectivos os quantificadores também são ``representados'' por termos da língua portuguesa como mostrado na tabela a seguir.

\begin{table}[h]
	\centering
	%\scriptsize
	\begin{tabular}{cl}
		\hline
		\textbf{Quantificador}  & \textbf{Termo em Portugu\^es} \\ \hline
		\multirow{4}{*}{Universal}    & Todo(a)s $\underline{ \ \ \ \ \ \ \ \ \ \ \ \ }$\\
		& Para todo(a) $\underline{ \ \ \ \ \ \ \ \ \ \ \ \ }$\\
		& Para qualquer $\underline{ \ \ \ \ \ \ \ \ \ \ \ \ }$\\
		& Para cada $\underline{ \ \ \ \ \ \ \ \ \ \ \ \ }$\\ \hline
		\multirow{4}{*}{Existencial} & Existe $\underline{ \ \ \ \ \ \ \ \ \ \ \ \ }$\\
		& Algum(a) $\underline{ \ \ \ \ \ \ \ \ \ \ \ \ }$\\
		& Para algum $\underline{ \ \ \ \ \ \ \ \ \ \ \ \ }$\\
		& Para um $\underline{ \ \ \ \ \ \ \ \ \ \ \ \ }$\\ \hline
	\end{tabular}
	\caption{Termos em português que representamos quantificadores.}
	\label{tab:QuantificadoresPT-BR}
\end{table}

Anteriormente já foi dito que na lógica as proposições e predicados podem ser interpretados como sendo verdadeiros ou falsos, dito isto, para qualquer proposição ou predicado sempre é possível obter uma proposição ou predicado com um valor de interpretação oposta, isto é, se a proposição (ou predicado) original for verdadeira a proposição (ou predicado) oposta será falsa, ou vice-versa. Esse ``construtor'' que gerar as proposições (ou predicados) opostas(os) é chamado de negação e a tabela a seguir exibe como os termos na língua portuguesa podem ser usados para representar a negação.

\begin{table}[h]
	%\scriptsize
	\centering
	\begin{tabular}{c}
		\hline
		\textbf{Termos em português}\\
		\hline
		Não $\underline{ \ \ \ \ \ \ \ \ \ \ \ \ }$ \\
		É falso que $\underline{ \ \ \ \ \ \ \ \ \ \ \ \ }$\\
		Não é verdade que $\underline{ \ \ \ \ \ \ \ \ \ \ \ \ }$ \\ \hline
	\end{tabular}
	\caption{Termos em português para designar a negação de uma proposição ou predicado.}
	\label{tab:NegacaoPortugues}
\end{table}

\begin{example}\label{exe:Negacaotextual}
	São exemplo de negação:
	\begin{itemize}
		\item[(a)] Não é verdade que a França faz fronteira com o Brasil.
		\item[(b)] Não existe um natural maior que $0$.
		\item[(c)] Não é verdade que todos os números pares são múltiplos de $6$. 
	\end{itemize}
\end{example}

\section{Representação simbólica}\label{sec:Representacao-simbolica}

Ao estudar lógica é comum adotar o uso de símbolos para representar as proposições, os predicados e os conectivos. Nesse sentido o estudo da lógica está interessada na estrutura dos objetos (proposições e predicados) e não nas frases em si \cite{joaoPavao2014}. Nesta seção será apresentado a simbologia básica sem entrar propriamente nos aspectos sintáticos da lógica.

A tabela a seguir apresenta a simbologia dos conectivos, dos quantificadores e também da negação que serão adotados neste manuscrito.

\begin{table}[h]
	%\scriptsize
	\centering
	\begin{tabular}{lc}
		\hline
		\textbf{Objeto} & \textbf{Símbolo}\\
		\hline
		Conjunção & $\land$\\
		Disjunção & $\lor$\\
		Implicação & $\Rightarrow$\\
		Bi-implicação & $\Leftrightarrow$\\
		Negação & $\neg$\\
		Quantificador universal & $\forall$\\
		Quantificador existencial & $\exists$\\
		\hline
	\end{tabular}
	\caption{Símbolos usados na Lógica simbólica.}
	\label{tab:SimbolosLogicos}
\end{table}

\begin{remark}
	No caso do quantificador de existência e unicidade, isto é, aquele que descreve a sentença \textbf{existe um único}, o mesmo é representado simbolicamente por $\exists!$, isto é, é usado o símbolo do quantificador existencial seguido de uma exclamação.
\end{remark}

\begin{definition}[Representação das Proposições]\label{def:RepresentacaoProposicoes}
	As proposições deve ser representadas usando letras maiúsculas do alfabeto latino.
\end{definition}

\begin{example}
	Representando as proposições ``2 > 5'', ``hoje é quarta feira'' e ``Alice é a professor de Introdução à Ciência da Computação'' respectivamente pela letras $P, Q$ e $R$ tem-se que:
	\begin{itemize}
		\item[(a)] $P \land Q$ representa a proposição: ``2 > 5 e hoje é quarta feira''.
		\item[(b)] $P \lor P$ representa a proposição: ``2 > 5 ou 2 > 5''.
		\item[(c)] $R \Rightarrow Q$ representa a proposição: ``Se Alice é a professor de Introdução à Ciência da Computação, então hoje é quarta feira''.
		\item[(d)] $\neg R \Rightarrow  P \lor R$ representa a proposição: ``Se não é verdade que Alice é a professor de Introdução à Ciência da Computação, então 2 > 5 ou  Alice é a professor de introdução à Ciência da Computação''.
		\item[(e)] $P \Leftrightarrow Q$ representa a proposição: ``2 > 5 se, e somente se hoje é quarta feira''.
	\end{itemize}
\end{example}

Agora a representação de um predicado é um pouco mais complexa, primeiro entre parênteses deve-se inserir o simbolo do quantificador e as variáveis ligadas a esse quantificador, se necessário pode-se incluir também o universo a qual essas variáveis pertences. Em seguida, entre colchetes é inserido a representação de sentença que pode ou não conter as variáveis ligadas ao quantificador, ressaltando que as asserções internas aos colchetes devem ser letras maiúscula do alfabeto latino seguidas imediatamente das variáveis do predicado entre parênteses caso necessário, os exemplos a seguir ilustram estas ideias.

\begin{remark}
	Vale ressaltar que os colchetes são símbolos usados para determinar o alcance do quantificador e de suas variáveis.
\end{remark}

\begin{example}
	O predicado: ``Existe um $professor$ que a mãe se chama Fátima'', usando $p$ para representar a variável $professor$ e $MF$ para representar a asserção da mãe do professor se chamar Fátima, pode-se representar tal predicado como: $(\exists p)[MF(p)]$.
\end{example}

\begin{example}
	O predicado: ``Existe uma $pessoa$ tal que a terra é quadrada'', usando $p$ para representar a variável $pessoa$ e $T_P$ para representar a asserção da terra ser plana, pode-se representar tal predicado como: $(\exists p)[T_P]$.
\end{example}

\begin{example}
	O predicado: ``Existe uma tapa, para fechar toda panela'', pode ser representada da seguinte forma, $(\exists t)[(\forall p)[F(t, p)]]$, aqui $t$ representa a variável tampa e $p$ representa a variável panela, por fim, $F(t,p)$ pode-ser interpretado como a asserção de $t$ fechar $p$.
\end{example}

\begin{example}
	O predicado ``Para todo número real, a terra é um planeta''. Pode ser representado por $(\forall n \in \mathbb{R})[P]$ aqui $P$ representa a proposição ``a terra é um planeta''.
\end{example}

\begin{example}
	O predicado ``Todos os homens são mortais''. Pode ser representado por $(\forall h)[M(h)]$ aqui $h$ representa a variável homem e a asserção do homem ser mortal é representado por $M(h)$.
\end{example}

\begin{example}\label{exe:TraducaoPredicadoFormula}
	O predicado: ``Para todo $x$ inteiro e todo $y$ inteiro, existe um número inteiro $z$ tal que $x + y = z$''. Pode ser representado simbolicamente como, $(\forall x, y \in \mathbb{Z})[(\exists z \in \mathbb{Z})[x+ y = z]]$.
\end{example}

\begin{remark}
	Aqui vale ressaltar que não existe nada que proíba representar $x + y = z$ por uma palavra da forma $S(x,y,z)$, e assim a representação exibida no exemplo \ref{exe:TraducaoPredicadoFormula} poderia ser escrita  forma $(\forall x, y \in \mathbb{Z})[(\exists z \in \mathbb{Z})[S(x, y, z)]]$.
\end{remark}


O leitor deve ter notado que $x, y$ no Exemplo \ref{exe:TraducaoPredicadoFormula} estão juntos no mesmo quantificador, isso ocorre pelo fato de ambos pertencerem ao mesmo universo do discurso (conjunto), estarem ligados a quantificadores do mesmo tipo e seus quantificadores são enunciado de forma aninhada direta\footnote{Dois quantificadores estão aninhados diretamente quando um está logo após o outro, ou seja, o escopo de um é seguido do escopo do outro.}, outra forma para presentar tal predicado seria $(\forall x \in \mathbb{Z})[(\forall y \in \mathbb{Z})[(\exists z \in \mathbb{Z})[x + y = z]]]$.

\section{Lógica e Ciência da Computação}\label{sec:Tipos-Logica-Aplicacoes}


Para finalizar este capítulo introdutório é conveniente falar mesmo que de forma superficial sobre os tipos de lógica e suas relações com a Ciência da Computação. A lógica assim como a física pode ser dividida em duas categorias ou tipos bem definidos, a saber, clássicas e as não clássicas. Como mencionado em  \cite{BenjaV1, edgar2002}, as lógicas clássicas são aquelas que apresentam a característica de obedecer os seguintes princípios: 

\begin{itemize}
	\item \textbf{Principio da não contradição}: Qualquer proposição (predicado) não pode ser verdadeira(o) e falsa(o) ao mesmo tempo;
	\item \textbf{Principio do terceiro excluído}: Toda(o) proposição  (predicado) só pode ser falsa(o) ou verdadeira(o), não existe uma terceira possibilidade. 
\end{itemize}

Logo a lógica clássica é bi-valorada \cite{edgar2002}, ou seja, as interpretações sobre as proposições e predicados só podem ser valoradas por dois valores, a saber: verdadeiro ou falso. E por sua vez, a própria lógica clássica é sub-dividida em duas partes, sendo estas: a lógica proposicional e a lógica de primeira ordem (ou lógica dos predicados).

Com respeito a aplicações a lógica clássica tem um papel fundamental e central para a Ciência da Computação, uma vez que, todos computadores são construídos pela combinação de circuitos digitais e estes por sua vez implementam operações da lógica proposicional \cite{abe2002-logica, nunes2008}. Outra área de destaque da aplicação da lógica dentro da Ciência da Computação é no campo de Inteligência Artificial, onde a mesma é o principal formalismo de representação do conhecimento e portanto é muito útil no desenvolvimento de sistemas especialistas e sistemas multi-agentes \cite{BenjaV1}, algumas outras áreas de aplicação da lógica clássica são:

\begin{itemize}
	\item Banco de dados: através descrição de consultas e no relacionamento das
tabelas em bancos de dados dedutivos.
	\item Ontologias web: como uma linguagem para descrever
ontologias e representar o conhecimento.
	\item Engenharia de software: usada como formalismo para especificação e verificação formal das propriedades dos sistemas\footnote{Em especial sistemas criticos como software para controle aéreo são exemplo de sistemas cujas propriedades deve ser especificadas e verificadas com alta precisão matemática dado a importância do mesmo para a manutenção da vida humanas que dependem dele.}.
\end{itemize}

As lógica não clássicas por sua vez, podem se apresentar de duas forma. (1) não obedecem algum dos princípios apresentados acima ou (2) estendem a lógica clássica através de teoremas e meta-teoremas (discutidos nos próximos capítulos) não válidos para as lógicas clássicas. Como exemplos de lógica não clássicas estão: lógica intuicionista \cite{lungarzo1972}, lógica paraconsistente \cite{da2008logica}, lógicas multivaloradas \cite{BenjaV1, magnus2020}, lógicas modais \cite{magnus2020} e lógicas temporais \cite{halpern1983, harel1979, manna1979}. Com respeitos a aplicações relacionadas Ciência da Computação tem-se por exemplo:

\begin{itemize}
	\item A utilização da lógica modal para a verificação da propriedades de sistemas e software \cite{harel1979}.
	\item A lógica temporal usada para especificação e verificação de programas
concorrentes \cite{manna1979} e também para especificar circuitos síncronos \cite{halpern1983}.
	\item As lógicas multi-valoradas usadas para lidar com a simulação e representação de
incertezas presente no raciocínio aproximado \cite{BenjaV1}, principalmente na área de reconhecimento de padrões.
\end{itemize}

Obviamente com dito em \cite{BenjaV1}, existem muitas outras lógicas não clássicas que têm aplicações ou ainda servem de fundamentação para diversas áreas ou disciplinas da computação. Porém, como esta parte do texto é apenas uma introdução não cabe neste escopo se debruçar tão profundamente assim neste assunto, em capítulos futuros as lógicas não clássicas (em especial a modal) serão estudadas mais a fundo. 

\section{Questionário}\label{sec:Questionario1part3}

\begin{problem}\label{prob:IntroLogica1}
    Examine cada uma das frases declarativas abaixo e diga se as mesmas são proposições ou predicados e também diga se são simples ou compostas, justifique suas respostas no caso de proposição (ou predicado) composta(o).
\end{problem}

\begin{exerList}
	\item Existe um gato amarelo.
	\item Alguns patos são marrons.
	\item Não é verdade que o gato de Júlio é amarelo.
	\item O Flamengo joga hoje.
	\item Todos os ratos tem olhos azuis.
	\item 4 é o menor número composto pelo produto de primos.
	\item Meu cachorro é branco, alguns outros são vermelhos.
	\item Alguns gatos são cinza, mas meu gato não é cinza, além disso, Tadeu tem um gato preto ou Sormany tem um rato amarelo.
	\item Os carros são amarelos se, e somente, se eles não são italianos.
	\item Se todos os jogadores da seleção jogam na Europa ou Neymar está machucado, então o Brasil não vence a Argentina.
	\item Basta mais um ponto na carteira de motorista para Sormany perde a aposta ou Juca terá que pagar o almoço.
	\item Se Lucas é irmão de Pedro, então Natalia vai casar com Gabriel se, e somente se, Francisco voltar da Espanha.
	\item Se $\frac{10^2}{50} = 2!$, então o $\pi - 2x = 0$ para todo $x \in \mathbb{C}$.
	\item Se a terra é plana e existe chip nas vacinas, então o Brasil vai conquistar o país de Juvenal. 
	\item A bola é preta.
	\item Eu tirei foto com meu avô hoje.
	\item $3 < 5$ segue do fato que o  voto no papel é mais rápido que voto eletrônico.
	\item O sorvete ser de uva segue do fato de Juliane está grávida.
	\item Bill escreveu o DOS em 1978 é condição necessária de Steven ter lançado o \textit{Apple} 2.
	\item Se o Cruzeiro é um time da primeira divisão, então todo macaco come macarrão ou não é o caso de Patricia ser professora de matemática. 
\end{exerList}

\begin{problem}\label{prob:IntroLogica2}
	Determine as frases simples (ou atômicas) que compõem as proposições e predicados que se seguem.
\end{problem}

\begin{exerList}
	\item Juca não irá a festa, mas Pedro irá ou Flaviana irá.
	\item Fui com a minha família ontem ao parque, e tomei sorvete de chocolate.
	\item Juca vai com família ou irá sozinho, mas se Anabel aparecer no parque, então Juca e Paula não vão ao cinema.
	\item Se Paulo chegou, então ele está na sala. Mas não é verdade que Paulo chegou.
	\item Valdigleis toca clarinete somente se, Katia tocar flauta ou Raíssa sabe desenhar com carvão.
	\item Eu sou aluno da computação somente se, eu passei em Matemática discreta. Mas eu não passei em Introdução à programação ou reprovei todas as disciplinas do primeiro período.
	\item Para qualquer navio no porto, existe um marinheiro bêbado no bar ou todos os soldados estão dormindo na praia.
	\item Se Romero tivesse vindo vê o filme, então Katia teria ido para a sorveteria com ele. Mas Romero foi para a praia com Julinha.
	\item Todos os números primos são números impares ou o número $\pi$ pode ser escrito como produto de dois números.
	\item Vou a padaria se, e somente se, estiver fazendo frio ou se Juliane quiser tomar sopa.
	\item Não é verdade que a terra é esférica se, e somente se, não existe pessoas morando em Recife.
	\item Raíssa e Altair foram para o Chile, mas Luiza também foi.
	\item Basta que eu tire 8 em Matemática discreta para eu ter um noite de festa.
	\item A gripe é uma condição suficiente para ser declarado morto.
	\item Se para todo flor existe um vaso, então os jardins de Jaçanã são cheio de cerejeiras. 
	\item Se ontem choveu a noite toda e hoje é meu aniversário de 14 anos, então Pedro vai a Califórnia se, e somente se, Tom e Frajola forem amigos do Manda-chuva.
	\item Não é verdade que Raíssa sabe desenhar com carvão, mas sabe desenhar com lápis e pincel.
	\item O Catatau comeu o mel somente se, o Puffy saiu para passear com o Jack ou Jerry é vizinho do Mickey.
	\item Para todo homem existe uma mulher que é sua mãe ou Saturno tem várias luas.
	\item Existe um carro amarelo se, e somente se, todas bicicletas são roxas, mas as motos são azuis. 
\end{exerList}

\begin{problem}\label{prob:IntroLogica3}
	Converta para notação simbólica todas as questões as sentenças nos Exercícios \ref{prob:IntroLogica1} e \ref{prob:IntroLogica2}.
\end{problem}

\begin{problem}\label{prob:IntroLogica4}
	Converta para notação simbólica as frase declarativas a seguir.
\end{problem}

\begin{exerList}
	\item Todo número natural é maior ou igual que 0.
	\item Rosas são vermelhas e violetas são azuis.
	\item Rosas são vermelhas e, violetas são azuis ou açúcar é doce.
	\item Julia gosta de doces mas Marcos prefere salgados.
	\item Existe um número real que divide 0.
	\item Todos os mamíferos sabem nadar.
	\item Se a galinha nasceu de um ovo, então todas as aves são dinossauros.
	\item Todos os cisnes são brancos.
	\item Todos os ratos gostam de leite.
	\item Se Abel torce para o Botafogo ou o para o Palmeiras, então ele não sabe o que é ganhar um mundial.
	\item Zico não ganhou uma copa, mas Vampeta ganhou logo, o Vasco vai voltar a primeira divisão.
	\item Todos os alunos passaram em Matemática Discreta, mas só alguns alunos  conseguem tirar 10 em Cálculo logo, não é verdade que todos os alunos reprovaram Administração.
	\item Não é verdade que, Carla come peixe se, e somente se, Thiago lavar a louça.
	\item Se Alfredo é paraibano, então Paulo passou em Física ou não é verdade que Manoel é canhoto. 
\end{exerList}