\chapter{Lógica Proposicional}\label{cap:LogicaProposicional}

\epigraph{``Ou a matemática é muito grande para a mente humana, ou a mente humana é mais do que uma máquina.''}{Kurt Gödel}

\section{Introdução}

Como todo bom material de lógica, este capítulo irá tratar inicialmente de forma separada dos aspectos semânticos e sintáticos da lógica proposicional. Inicialmente será apresentado a estrutura da sintaxe e diverso meta-teoremas da linguagem proposicional, após isso, será apresentada a estrutura semântica e varios resultados sobre a mesma, por fim, será desenvolvido um estudo relacional sobre as duas facetas (sintaxe e semântica) da lógica proposicional, ou seja, serão estudadas as noções de corretude e completude.  

\section{A linguagem proposicional}\label{sec:LingProp}

Este capítulo tem como objetivo apresentar ao leitor o cálculo proposicional, ou seja, o estudo da lógica proposicional, em seus dois aspectos já bem estabelecido por matemáticos e filósofos, isto é,  sua sintaxe e sua semântica\footnote{O aspecto pragmático da lógica, por ainda se encontrar em um estágio primitivo de seu desenvolvimento, do ponto de vista matemático, não será abordado neste texto, para este assunto ver \cite{rodrigues2021, silva2018}.}. Assim este capítulo começa com a formalização da linguagem da lógica proposicional, isto é, a linguagem proposicional. A seguir é apresentado formalmente a noção de alfabeto proposicional.

\begin{definition}[Alfabeto Proposicional]\label{def:AlfProp}
	O alfabeto proposicional corresponde ao conjunto enumerável $\Sigma = \Sigma_s \cup \Sigma_o \cup \Sigma_p \cup \{\bot\}$ onde:
	\begin{itemize}
		\item $\Sigma_s = \{A, \cdots, P, Q, R, P_1, Q_{12}, \cdots\}$ é um conjunto enumerável, chamado conjunto de símbolos proposicionais;
		\item $\Sigma_o = \{\land, \lor, \neg, \Rightarrow\}$ é o conjunto dos símbolos operacionais\footnote{Também é comum encontrar na literatura (ver \cite{joaoPavao2014}) a nomenclatura conjunto de conectivos.};
		\item $\Sigma_p = \{(, )\}$ é o conjunto dos símbolos de pontuação e
		\item $\bot$ é o símbolo do absurdo.
	\end{itemize}
\end{definition}

Qualquer sequência de símbolos do alfabeto proposicional é chamada de palavra, entretanto, nem toda palavra será considerada como sendo parte da linguagem proposicional. A Definição \ref{def:LingProp} a seguir formaliza o conjunto que corresponde a linguagem proposicional.

\begin{definition}[Linguagem Proposicional]\label{def:LingProp}
	Dado o alfabeto proposicional $\Sigma$ a linguagem proposicional, denotada por $L_{Prop}$, é o menor conjunto indutivamente gerado pelas seguintes regras:
	\begin{enumerate}
		\item Para todo $\alpha \in \Sigma_s \cup \{\bot\}$, tem-se que $\alpha \in L_{Prop}$.
		\item Se $\alpha \in L_{Prop}$, então $(\neg\alpha) \in L_{Prop}$.
		\item Se $\alpha, \beta \in L_{Prop}$, então $(\alpha \land \beta), (\alpha \lor \beta), (\alpha \Rightarrow \beta) \in L_{Prop}$. 
	\end{enumerate}
\end{definition}

\begin{example}\label{exe:PalavrasProposicionaisBemFormadas}
	Dado $P, Q, R, S, T \in \Sigma_s \cup \{\bot\}$ tem-se que:
	\begin{itemize}
		\item[(a)] $P$
		\item[(b)] $(P \land Q)$
		\item[(c)] $(R \Rightarrow S)$
		\item[(d)] $((Q \lor S) \Rightarrow T)$
	\end{itemize}
	são todas palavras da linguagem $L_{Prop}$. Por outro lado, as palavras:
	\begin{itemize}
		\item[(e)] $P \land$
		\item[(f)] $\Rightarrow Q$
		\item[(g)] $P \lor \land Q$
	\end{itemize}
	não são palavras da linguagem $L_{Prop}$.
\end{example}

\begin{remark}
	Para simplificar a escrita das palavras de $L_{Prop}$ é comum omitirem-se os parênteses mais exteriores das palavras, por exemplo, é escrito apenas $(Q \lor S) \Rightarrow T$ em vez de $((Q \lor S) \Rightarrow T)$. 
\end{remark}

É possível enriquecer\footnote{No sentido de adicionar mais símbolos operacionais.} a linguagem proposicional adicionando mais símbolos operacionais no alfabeto da mesma, essa introdução é feita utilizando o conceito de abreviação. Uma abreviação na lógica formal consiste na ação de usar um novo símbolo para criar uma nova palavra não presente originalmente na linguagem proposicional, mas que representa uma palavra da linguagem.

\begin{definition}[Abreviatura da Bi-implicação]\label{def:SeSomenteSeAbreviatura}
	Considerando o novo símbolo $\Leftrightarrow$, para todo $\alpha, \beta \in L_{Prop}$ as palavras na forma $\alpha \Leftrightarrow \beta$ são as abreviações das palavras de $L_{Prop}$ na forma $(\alpha \Rightarrow \beta) \land (\beta \Rightarrow \alpha)$, em notação formal tem-se $\alpha \Leftrightarrow \beta \stackrel{abr.}{=} (\alpha \Rightarrow \beta) \land (\beta \Rightarrow \alpha)$.
\end{definition}

De fato, muitos dos símbolos operacionais que foram tomados como símbolos básicos do alfabeto proposicional (Definição \ref{def:AlfProp}) poderiam ser removidos, pois como muito bem explicado em \cite{BenjaV1, joaoPavao2014} a lógica proposicional pode ser definida sobre a linguagem que contém apenas os símbolos operacionais de $\Rightarrow$ e $\neg$, os demais símbolos podem ser obtidos via abreviação sem qualquer perda no estudo da lógica proposicional, para mais detalhes ver \cite{BenjaV1}.

\section{Sobre o Sistema de Dedução Natural}\label{sec:SistemaDedutivo}

A ideia de sistemas dedutivos para a lógica formal remonta aos trabalhos publicados\footnote{Esses trabalhos podem ser encontrados re-editados respectivamente em \cite{gentzen1969} e \cite{jaskowski1934}.} no ano de 1934 pelo matemático e filósofo alemão Gerhard Gentzen (1909-1945) e pelo lógico polonês Stanisław Jaśkowski (1906-1965). Existem diversos sistemas dedutivos para a lógica proposicional, cada um possuindo suas próprias características, vantagens e desvantagens, no entanto, todos os sistemas dedutivos compartilham a característica em comum de possuírem um conjunto finito de regras de inferência, esse conjunto de regras de inferência é também chamado de sistema regras ou sistema de dedução \cite{edgar2002}.

O sistema dedutivo introduzido por Gentzen e Jaśkowski é conhecido por dedução natural, aqui ele será apresentado de forma similar a exposição feita em \cite{joaoPavao2014}. O conjunto de regras de inferência da dedução natural e composto pelas regras: de introdução e eliminação de conectivos, regra de reiteração, introdução de hipóteses e a regra do absurdo. Entretanto, antes de apresentar as regras do sistema de dedução natural e conveniente apresentar o conceito de demonstração, para isso deve-se escolher uma notação para as provas da dedução natural.

Existem diversas formas de se escrever (ou representar) uma demonstração no sistema de dedução natural, entre elas destacam-se as árvores de prova de Gentzen \cite{BenjaV1}, o estilo linear \cite{copi1981, mortari2001} e o estilo de Fitch \cite{joaoPavao2014, fitch1953}. Este último é exatamente o sistema usado para os diagramas de bloco do Capítulo \ref{cap:Demonstracoes}.

Neste texto será adotado o estilo de Fitch como modelo padrão para a escrita das demonstrações do sistema de dedução natural para a lógica proposicional e posteriormente para a lógica de primeira ordem, assim é conveniente apresentar de forma sucinta o estilo de Fitch  \cite{broda2007}.

O estilo de Fitch foi introduzido pelo lógico americano Frederic Brenton Fitch (1908 - 1987) e corresponde a diagramas hierárquicos formados por linhas e barras (verticais e horizontais) que representam o raciocínio para a partir de um conjunto de premissas se obter uma determinada conclusão ou objetivo (em inglês \textit{goal}).

O diagrama de Fitch é organizado por linhas numeradas, onde cada linha $i$ pode conter uma única palavra de $L_{Prop}$, sendo essa palavra uma premissa ou sendo ela obtida pela aplicação de alguma regra de inferência sobre uma ou mais linhas anteriores a linha $i$. As barras verticais nos diagramas de Fitch são usadas de duas formas:
\begin{itemize}
	\item[(1)] Para separar a demonstração em escopos, sendo que um escopo consiste de uma sequencia de várias linhas (ou passos) para demonstrar uma conclusão.
	\item[(2)] Como um mecanismo para saber quais palavras de $L_{Prop}$ estão ativas\footnote{Uma palavra de $L_{Prop}$ está ativa em uma demonstração, enquanto o escopo da mesma está aberto na demonstração.} na prova, como explicado em \cite{joaoPavao2014}. 
\end{itemize}

As barras horizontais no diagrama de Fitch indicam a divisão entre  as  afirmações  que  estão sendo  assumidas  (podendo ser premissas e (ou) hipóteses) e as palavras que se seguem delas, sejam conclusões intermediárias ou o objetivo final. No caso das hipóteses a barra horizontal também cria um novo ``escopo'', isto é, adiciona uma indentação em relação ao escopo anterior, vale salientar que cada escopo é na verdade uma prova para um (sub-)objetivo.

Por fim, é comum na notação dos diagramas de Fitch escrever mais à direita de cada linha a regra de inferência que gerou a palavra na linha, ou o fato da palavra ser uma premissa ou hipótese. Agora pode-se apresentar formalmente o conceito de prova que será adotado neste capítulo.

\begin{definition}[Prova]\label{def:Prova}
	Uma prova para $\alpha \in L_{Prop}$ consiste de um diagrama de Fitch como uma quantidade finita de linhas, de forma que a última linha contém a palavra $\alpha$ e cada linha $i$ anterior contém uma palavra $\beta_i \in L_{Prop}$ tal que $\beta_i$ ou é uma premissa ou é obtida via aplicação de alguma regra de inferência.
\end{definition}

Agora pode-se definir precisamente o conceito que relaciona um conjunto de premissas com uma palavra de $L_{Prop}$, este conceito é conhecido como relação de consequência sintática sobre a linguagem $L_{Prop}$, o mesmo descreve a noção de derivabilidade.

\begin{definition}[Consequência Sintática]\label{def:ConseSintatica}
	Seja $L_{Prop}$ a linguagem proposicional, dado $\alpha \in L_{Prop}$ e $\Gamma \subseteq L_{Prop}$, diz-se que $\alpha$ é consequência sintática de $\Gamma$, denotado por $\Gamma \vdash \alpha$, sempre que existir uma prova de $\alpha$ a partir do conjunto de premissas $\Gamma$. 
\end{definition}

\begin{remark}
	Note que uma instância de consequência sintática pode ser vista como um elemento de $\wp(L_{Prop}) \times L_{Prop}$, isto é, a  consequência sintática $(\vdash)$ pode ser vista como uma relação no sentido usual da teoria ingênua dos conjuntos.
\end{remark}

\begin{note}
	Uma interpretação que se pode ter sobre a relação de consequência sintática $\Gamma \vdash \alpha$ é que a mesma pode ser vista como: a partir das palavras em $\Gamma$ é sintaticamente possível escrever a palavra $\alpha$
\end{note}

A seguir são apresentadas as regras de inferência do sistema de dedução natural. 

\section{Regras de Dedução Natural}

Aqui será feito a apresentação das regras de dedução natural iniciando pelas regras que não envolvem diretamente os símbolo operacionais, isto é, as regras que não agem diretamente para eliminar ou introduzir os elementos de $\Sigma_o$ na demonstração.

\begin{definition}[Regra das premissas]\label{def:RegraPremissas}
	Seja $\Gamma = \{\alpha_1, \cdots, \alpha_n \}$ um conjunto finitos (possivelmente vazio) de premissas, então a regra das premissas fixa que a construção do diagrama de Fitch para uma prova de $\Gamma \vdash \alpha$ dispões nas $n$ primeiras linhas do diagrama as $n$ premissas contidas $\Gamma$, onde na linha $i$ se encontra a premissa $\alpha_i$, além disso, existe uma barra vertical contínua\footnote{Cada linha vertical contínua é um escopo dentro da demonstração.} a esquerda das premissas e após a linha $n$ há uma barra horizontal separando as promissas do resto da prova, ou seja:
	$$
		\begin{nd}
			\have[1]{h}{\alpha_1} \by{Premissa}{}
			\have[\vdots]{skip1}{\vdots} 
			\hypo[n]{atob}{\alpha_n} \by{Premissa}{}
			\have[\vdots]{skip1}{\vdots}
		\end{nd}
	$$
\end{definition}

Seguindo com as regras mais básicas do sistema de dedução natural pode-se agora apresentar a regra chamada de regra da reiteração, repetição, copia ou clonagem. Tal regra estabelece que se já existe uma dedução de $\alpha$ em um diagrama, então pode-se duplicar a fórmula $\alpha$ no diagrama. Neste manuscrito tal regra será denotada apenas por REI.

\begin{definition}[Regra da reiteração]\label{def:RegraRepetição}
	Dado uma palavra $\beta \in L_{Prop}$ que já foi deduzida em uma linha $i$ durante a prova, pode-se escrever novamente $\beta$ em uma linha $j$ com $j > i$, desde que o escopo que contém $\beta$ ainda esteja ativo\footnote{A noção de escopo ativo diz respeito se uma (sub-)prova foi concluída ou ainda está em desenvolvimento, este conceito será melhor trabalhado mais adiante beste manuscrito.}. Na notação de Fitch tem-se o seguinte diagrama:
	$$
		\begin{nd}
			\have[\vdots]{skip1}{\vdots} 
			\have[i]{h}{\beta}
			\have[\vdots]{skip1}{\vdots} 
			\have[n]{atob}{\beta} \by{REI}{h}
			\have[\vdots]{skip1}{\vdots}
		\end{nd}
	$$
\end{definition}

Agora que já foram apresentadas as regras que não agem diretamente sobre os símbolos operacionais pode-se dá sequência no texto apresentando as regras de inferência do sistema de dedução natural que atuam diretamente sobre os símbolos. Como em muitos textos (ver \cite{magnus2020}) será inicialmente apresentado a regra de introdução da conjunção.

\begin{definition}[Regra de introdução da conjunção ($\land$I)]\label{def:RegraIntroducaoE}
	Se em uma prova já foram deduzidas as palavras $\alpha, \beta \in L_{Prop}$ nas linhas $i$ e $j$ respectivamente, então pode-se deduzir a palavra $\alpha \land \beta$ em uma linha $k$ com $i < j < k$, na notação do diagrama de Fitch tem-se:
	$$
		\begin{nd}
			\have[\vdots]{skip1}{\vdots} 
			\have[i]{a}{\alpha}
			\have[\vdots]{skip1}{\vdots} 
			\have[j]{b}{\beta} 
			\have[\vdots]{skip1}{\vdots} 
			\have[k]{c}{\alpha \land \beta} \ai{a, b}
			\have[\vdots]{skip1}{\vdots}
		\end{nd}
	$$
\end{definition}

\begin{remark}
	Para critérios de rigorosidade,  a regra de introdução da conjunção impõe que a palavra que está na linha $i$ seja fixada à esquerda do símbolo $\land$ e a palavra na linha $j$ seja fixada à direita do símbolo $\land$.
\end{remark}

A próxima regra de dedução natural a ser apresentada é a eliminação da conjunção, tal regra possui duas formas de uso, o que contrasta com a regra da introdução da conjunção que possui apenas uma única forma. Primeiramente note que o operador $\land$ combina duas palavras $\alpha, \beta \in L_{Prop}$, assim quando tal operador for removido deve-se optar por qual das duas palavras será mantida como uma conclusão (intermediária ou final) da prova. A seguir é definida formalmente a regra de eliminação de conjunção.

\begin{definition}[Regra de eliminação da conjunção ($\land E$)]\label{def:RegraEliminacaoE}
	Se em uma prova for deduzida a palavra $\alpha \land \beta$ na linha $i$, então pode-se deduzir a palavra $\alpha$ ou então a palavra $\beta$ em uma linha $j$ com $i < j$, na notação do diagrama de Fitch tem-se:
	
	\begin{minipage}{.45\textwidth} %
		$$
			\begin{nd}
				\have[\vdots]{skip1}{\vdots}  
				\have[i]{a}{\alpha \land \beta}
				\have[\vdots]{skip1}{\vdots}  
				\have[j]{b}{\alpha} \ae{a}
				\have[\vdots]{skip1}{\vdots} 
			\end{nd}
		$$
	\end{minipage} %
	ou
	\begin{minipage}{.45\textwidth} %
		$$
			\begin{nd}
				\have[\vdots]{skip1}{\vdots}  
				\have[i]{a}{\alpha \land \beta}
				\have[\vdots]{skip1}{\vdots}  
				\have[j]{b}{\beta} \ae{a}
				\have[\vdots]{skip1}{\vdots} 
			\end{nd}
		$$
	\end{minipage}
\end{definition}

Agora será aberto um parêntese na apresentação das regras de inferência dos símbolos operacionais para que possa ser discutido neste texto a noção de prova hipotética. As provas hipotéticas são muito importantes dentro do sistema de dedução natural, tais provas com dito em \cite{joaoPavao2014}, podem ser pensadas como sendo um ambiente (ou escopo) de sub-prova em que além das premissas que iniciaram a prova são assumidas outras informações adicionais na forma de hipóteses. 

Como argumentado em \cite{copi1981, joaoPavao2014}, uma prova hipotética surge quando a regra de introdução hipótese é aplicada, e ao se introduzir essa nova hipótese na prova é gerado um novo escopo dentro da prova que se estava demonstrando, isto é, é criada uma sub-prova que terá seu próprio objetivo. 

\begin{definition}[Regra de introdução de hipótese]\label{def:RegraHipotese}
	Dado uma demonstração com $n$ passos, se for necessário assumir uma hipótese $\beta \in L_{Prop}$ no passo $n+1$, então é inserida a hipótese $\beta$ junto com uma barra vertical de escopo, e abaixo de $\beta$ é inserida a barra horizontal de separação para destacar a hipótese, aqui será usado a palavra \textbf{Assuma} para referenciar a regra de introdução de hipótese\footnote{Na literatura em língua inglesa é comum o uso do termo \textit{Assumption}.}.
	$$
		\begin{nd}
			\have[\vdots]{skip1}{\vdots}  
			\have[n]{a}{\vdots}
			\open
			\hypo[n+1]{b}{\beta} \by{Assuma}{}  
			\have[\vdots]{c}{\vdots}
			\close
		\end{nd}
	$$
\end{definition}

Como dito em \cite{edgar2002}, uso da regra de inferência de introdução de hipótese está intimamente ligada ao uso da regra de introdução da implicação definida a seguir, por isso a necessidade de apresenta-lá antes da regra de introdução da implicação. 

\begin{definition}[Introdução da implicação ($\Rightarrow$I)]\label{def:RegraIntroImplicacao}
	Se partindo de uma suposição hipotética $\alpha$ na linha $m$ for possível deduzir um certo $\beta$ na linha $n$ com $m < n$, então no escopo externo da prova hipotética é concluído na linha $n+1$ que $\alpha \Rightarrow \beta$, na notação dos diagrama de Fitch tem-se:
	
	$$
		\begin{nd}
			\have[\vdots]{skip1}{\vdots}  
			\open
			\hypo[m]{a}{\alpha} \by{Assuma}{}  
			\have[\vdots]{b}{\vdots}
			\have[n]{c}{\beta}
			\close
			\have[n+1]{d}{\alpha \Rightarrow \beta} \ii{a--c}
			\have[\vdots]{skip1}{\vdots} 
		\end{nd}
	$$
\end{definition}

\begin{remark}
	Note que a regra de introdução da implicação pode ser vista como um mecanismo que desativa um escopo de prova, isto é, quando a mesma é aplicada um escopo de prova terá sido completado e assim estará desativado (ou fechado).
\end{remark}

Vale destacar que ao desativar um escopo de prova todas as palavras contidas unicamente entre as linhas $i$ e $j$ que foram o escopo, não podem mais ser utilizadas na sequência da demonstração, isso ocorre pela razão de tais palavras só existirem no escopo ``local'' da sub-prova que foi concluída.

Prosseguindo com a apresentação das regras de inferência do sistema de dedução natural agora será definida formalmente a regra de eliminação da implicação, também conhecida como \textit{modus ponens}, que surge da expressão em latin, \textit{modus ponendo ponens}\footnote{Que em português pode ser traduzido como: \textbf{o modo de afirmar, afirmando}}. 

\begin{definition}[Regra da eliminação da implicação ($\Rightarrow$E)]\label{def:EliminacaoImplicacao}
	Se em uma prova na linha $i$ existe uma palavra $\alpha$ e em uma linha $j$ existe uma palavra $\alpha \Rightarrow \beta$ com $i < j$, então na linha $k$ tal que $j < k$ é possível deduzir a palavra $\beta$, em diagrama tem-se:
	$$
		\begin{nd}
			\have[i]{a}{\alpha}
			\have[\vdots]{skip1}{\vdots}  
			\have[j]{b}{\alpha \Rightarrow \beta}
			\have[\vdots]{skip1}{\vdots} 
			\have[k]{c}{\beta} \ie{a,b}
			\have[\vdots]{skip1}{\vdots}
		\end{nd}
	$$
\end{definition}

\begin{remark}
	Novamente para manter a rigorosidade sintática a Definição \ref{def:EliminacaoImplicacao} especifica que o termo hipotético $\alpha$ deve aparecer na prova antes do termo condicional $\alpha \Rightarrow \beta$, para que se possa aplicar a regra $\Rightarrow$E.
\end{remark}

A próxima regra de inferência do sistema de dedução natural que será apresentada neste texto é chamada de regra de introdução do absurdo, a mesma é utilizada para introduzir na demonstração o símbolo do absurdo\footnote{É comum na literatura em língua inglesa principalmente na área de lógica algébrica achar o símbolo do absurdo sendo chamado  \textit{bottom}.} $(\bot)$.

\begin{definition}[Regra de introdução do absurdo ($\bot $I)]\label{def:IntroducaoDoAbsurdo}
	Se na linha  $i$ de uma prova existe uma palavra $\beta$ e no mesmo escopo de prova na linha $j$ existe uma palavra $\neg \beta$ com $i < j$, então na linha $k$ desta prova é deduzido $\bot$ com $j < k$, em diagrama tem-se:
	$$
		\begin{nd}
			\have[\vdots]{skip1}{\vdots} 
			\have[i]{a}{\beta}
			\have[\vdots]{skip1}{\vdots} 
			\have[j]{b}{\neg \beta} 
			\have[\vdots]{skip1}{\vdots} 
			\have[k]{c}{\bot} \by{$\bot$ I}{a,b}
			\have[\vdots]{skip1}{\vdots} 
		\end{nd}
	$$
\end{definition}

\begin{remark}
	Em alguns texto também é mencionado que o absurdo é na verdade uma abreviação para a palavra $\alpha \land \neg \alpha$.
\end{remark}

\begin{remark}
	O leitor deve ficar atendo ao fato da Definição \ref{def:IntroducaoDoAbsurdo} estabelecer que a palavra $\beta$ deve vim antes da palavra $\neg \beta$ no escopo da prova.
\end{remark}

Agora utilizando a regra de introdução do absurdo é gerada uma nova regra do sistema de dedução natural, tal regra é responsável por introduzir a negação. Assim como a regra de introdução da implicação ela também é realizada em uma estrutura de sub-prova.

\begin{remark}
	O leitor atento notará a seguir que a introdução da negação sobre uma palavra $\alpha$ não é nada além de uma abreviação para a palavra $\alpha \Rightarrow \bot$
\end{remark}

\begin{definition}[Regra de introdução da negação ($\neg$I)]\label{def:IntroducaoNegacao}
	Se existe uma sub-prova iniciada com a hipótese $\alpha$ na linha $i$ que deduz $\bot$ em uma linha $j$  tal que $i < j$, então na linha $k$ com $j < k$ o escopo da sub-prova é fechado e é escrito a palavra $\neg \alpha$. No que diz respeito ao diagrama de Fitch tem-se:
	
	$$
		\begin{nd}
			\have[\vdots]{skip1}{\vdots}
			\open
			\hypo[i]{a}{\alpha}
			\have[\vdots]{skip1}{\vdots}
			\have[j]{b}{\bot}
			\close
			\have[\vdots]{skip1}{\vdots}
			\have[k]{c}{\neg \alpha} \ni{a--b}
			\have[\vdots]{skip1}{\vdots}
		\end{nd}
	$$
\end{definition}

De forma dual a regra a seguir descreve um mecanismo para a eliminação da negação, tal regra pode ser interpretada como a ideia de que negar uma palavra (argumento) duas vezes é o mesmo que afirmar tal palavra (argumento).

\begin{definition}[Regra de eliminação da negação ($\neg E$)]\label{def:EliminacaoDaNegacao}
	Sempre que existir uma palavra $\neg \neg \alpha$ em uma linha $i$, então em uma linha $j$ pode-se deduzir $\alpha$ com $i < j$. Em notação de diagrama tem-se:
	$$
		\begin{nd}
			\have[\vdots]{skip1}{\vdots}
			\have[i]{a}{\neg \neg \alpha}
			\have[\vdots]{skip1}{\vdots}
			\have[j]{b}{\alpha} \ne{a}
			\have[\vdots]{skip1}{\vdots}
		\end{nd}
	$$
\end{definition}

Por fim, serão agora apresentadas as regras de introdução e eliminação para a disjunção para o sistema de dedução natural.

\begin{definition}[Regra de introdução da disjunção ($\lor$I)]\label{def:IntroducaoDisjuncao}
	Se em uma prova aparece na linha $i$ uma palavra $\alpha$, então em uma linha $j$ tal que $i < j$ pode-se deduzir para algum $\beta \in L_{Prop}$ uma das seguintes palavras: $\alpha \lor \beta$ ou $\beta \lor \alpha$. Na notação do diagrama de Fitch tem-se:
	
	\begin{minipage}{.40\textwidth} %
		$$
		\begin{nd}
		\have[\vdots]{skip1}{\vdots}  
		\have[i]{a}{\alpha}
		\have[\vdots]{skip1}{\vdots}  
		\have[j]{b}{\alpha \lor \beta} \oi{a}
		\have[\vdots]{skip1}{\vdots} 
		\end{nd}
		$$
	\end{minipage} %
	ou
	\begin{minipage}{.40\textwidth} %
		$$
		\begin{nd}
		\have[\vdots]{skip1}{\vdots}  
		\have[i]{a}{\alpha}
		\have[\vdots]{skip1}{\vdots}  
		\have[j]{b}{\beta \lor \alpha} \oi{a}
		\have[\vdots]{skip1}{\vdots} 
		\end{nd}
		$$
	\end{minipage}
\end{definition}

A regra de eliminação da disjunção é um pouco mais complicada, pois para ser realizada a mesma invoca duas sub-provas hipotéticas, formalmente tal regra é definida como se segue.

\begin{definition}[Regra de eliminação da disjunção ($\lor$E)]\label{def:EliminacaoDisjuncao}
	Sempre que existe uma palavra $\alpha \lor \beta$ na $i$-ésima linha da prova e for possível deduz $\gamma$ a partir de sub-provas iniciadas com $\alpha$ e $\beta$ como hipótese, então na linha $n$ tal que $i < n$ é possível deduzir a palavra $\gamma$. Na notação de diagramas tem-se que:
	$$
	\begin{nd}
	\have[i]{a}{\alpha \lor \beta}
	\have[\vdots]{skip1}{\vdots}
	\open
	\hypo[j]{b}{\alpha}
	\have[\vdots]{skip1}{\vdots}  
	\have[j + l_1]{c}{\gamma}
	\close
	\open
	\hypo[k]{d}{\beta}
	\have[\vdots]{skip1}{\vdots}  
	\have[k + l_2]{e}{\gamma}
	\close
	\have[\vdots]{skip1}{\vdots}  
	\have[n]{f}{\gamma} \oe{a, (b--c, d--e)}
	\end{nd}
	$$
\end{definition}

Neste ponto do texto tem-se que foram apresentadas todas as regras básicas de inferência para o sistema de dedução natural da linguagem proposicional que foi apresentada na Definição \ref{def:LingProp}. Entretanto, pode-se pensar em extensões da linguagem proposicional baseado na ideia de abreviações (Definição \ref{def:SeSomenteSeAbreviatura}), assim é natural que os novos símbolos criados também possuam suas regras de introdução e eliminação, para o caso da bi-implicação que foi descrita neste texto (Definição \ref{def:SeSomenteSeAbreviatura}) suas regras de introdução e remoção foram muito bem detalhadas em \cite{carmo2013}.

\section{Construção de Demonstrações em Dedução Natural}\label{sec:ProvasDN}

\section{Propriedades do Sistema de Dedução Natural}\label{sec:ResultadosDN}

\section{Sistemas Axiomáticos ao Estilo de Hilbert}\label{sec:SistemaAxiomatico}

\section{A Semântica da Linguagem $L_{Prop}$}\label{sec:SistemaSemantico}

\section{Propriedades do Sistema Semântico}\label{sec:ResultadosSemanticos}