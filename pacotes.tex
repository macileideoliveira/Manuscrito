% ===================================================
% Arquivo para importação de pacotes básicos
% ===================================================

% ===================================================
% Pacotes matemáticos
% =================================================== 
\usepackage{amssymb}

% ===================================================
% Pacote para e configurações para desenhar
% ===================================================
\usepackage{tikz}
\usepackage{tikz-qtree}

\usetikzlibrary{positioning, calc, chains, fit, shapes, automata, trees}

% ======================================================
% Pacote para subfigura
% ======================================================
\usepackage{subfig}

% ========================================================
% Pacote para dividir na diagonal as cédulas de uma tabela
% ========================================================
\usepackage{slashbox}

% ========================================================
% Pacotes e configuração de algoritmos
% =========================================================
\usepackage[portuguese,ruled,lined, linesnumbered]{algorithm2e}
\usepackage{algorithmic}

% ========================================================
% Pacotes para colocar epígrafes nas páginas de partes
% ========================================================
\usepackage{epigraph}
\usepackage{titlesec}
\usepackage{xpatch}

% ========================================================
% Pacotes para caputar a data e hora atual
% ========================================================
\usepackage[yyyymmdd,hhmmss]{datetime}

% ========================================================
% Pacotes para provas em caixas no style de Fitch
% ========================================================
\usepackage{logicproof}

% ========================================================
% Pacotes para criar novos tipos de lista
% ========================================================
\usepackage{enumitem}