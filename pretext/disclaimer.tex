%----------------------------------------------------------------------------------------
%	Página de Disclaimer
%----------------------------------------------------------------------------------------
\begingroup
\thispagestyle{empty}
\begin{center}
	{\normalfont\fontsize{20}{20}\sffamily\selectfont \textbf{\textit{Disclaimer}}}\par
\end{center}

\vspace{1cm}

Este manuscrito está sendo construído tendo como base diversas notas de aula que eu preparei para cursos de:

\begin{multicols}{2}
	\begin{fieldsList}
		\item Matemática Discreta
		\item Lógica para Ciência da Computação
		\item Linguagens Formais e Autômatos
		\item Análise de Algoritmos
		\item Computabilidade
		\item Teoria dos Números para Computação
	\end{fieldsList}
\end{multicols}	

Uma vez que este manuscrito ainda é um projeto em andamento e possivelmente sua escrita nunca será realmente concluída com total aprovação de seu autor, é claro que você poderá encontrar diversos erros, que com toda certeza você leitor irá me enviar e-mails\footnote{E-mail do autor: \url{valdigleis.costa@univasf.edu.br}} ou \textit{issues}\footnote{Páginas de \textit{issues}: \url{https://github.com/valdigleis/Manuscrito/issues}} com reports de tais erros.

%Com respeito a organização do manuscrito é interessante ressaltar que ele não tem uma ordem geral para a leitura de capítulos, de fato, existem diversas ordem de leituras. Assim o autor recomenda a leitura na ordem com que o leitor for necessitando dos conteúdos, isto é, realizando \textbf{pulos}, \textbf{adiamentos} e \textbf{retomadas} de leitura entre os capítulos e seções.

%Além das básicas estruturas de definições, teoremas, lemas, exercícios, exemplos e etc, este manuscrito ainda conta com três tipos ``novos'' de ambientes sendo: as explicações Snuffles, observações do Rick e perguntas do Morty.

%\begin{SnufflesInfo}
%	O Snuffles é um cachorrinho que ficou super inteligente,  ele gosta de se gabar dando explicações ao leitor sobre temas do manuscrito.
%\end{SnufflesInfo}

%\begin{RickInfo}
%	O Rick como é um grande cientista conhece muito de tudo então durante o texto ele vai dando ao leitor observações de diversos tipos sobre os temas e mostrando outras formas de entender o mesmo assunto.
%\end{RickInfo}

%\begin{MortyInfo}
%	O Morty ainda é um adolescente que está crescendo e aprendendo, assim ele faz diversas perguntas a você leitor, já que o Rick está bêbado demais para responder e o Snuffles prefere fazer planos (para a revolução canina) do que explicar coisas ao Morty. Se você estiver com a mesma dúvida do Morty, pergunte ao seu professor em sala de aula, assim você sana a dúvida sua e do Morty.
%\end{MortyInfo}

\endgroup
\newpage